%% It is just an empty TeX file.
%% Write your code here.

\chapter{El problema de la adecuación}

\noindent Hay un problema conceptual cuando se habla de adecuación en biología evolutiva. Esto nos lleva a que hay al menos una cuestión filosófica importante cuando se trata la discusión sobre la adecuación. Este es el problema de la tautología. Dicho en palabras de Diane Paul ``[s]i adecuación está definido en términos de éxito de sobrevivencia y éxito reproductivo, entonces el enunciado de que sobrevive el más apto, está aparentemente vació de contenido'' \footnote{If fitness is defined as success in surviving and reproducing, the statement that the fittest survive is apparently emptied of content.} \citeyear{Paul1992}

Cuando decimos que los organismos con más adecuación son aquellos que dejan más descendencia y los organismos más adecuados son quienes dejan más descendencia obtenemos una tautología. El concepto de adecuación es central en las explicaciones en biología evolutiva, por lo que, si es un enunciado tautológico, entonces Si queremos explicar cómo es que los organismos con más \emph{fitness} son los que dejan más descendencia, mientras que a su vez definimos y medimos al \emph{fitness} en términos de dejar más descendencia, entonces este enunciado es trivialmente verdadero: los organismos que dejan más descendencia son aquellos que dejan más descendencia \cite{Bouchard2004}. Según Kenneth Waters \citeyear{Waters1986}, para solucionar este problema hay que mostrar como es que podemos definir al \emph{fitness} de manera tal que sea independiente a la medida de descendencia que deja un organismo. Creemos que podemos hacer esto ofreciendo una definición en términos causales del \emph{fitness}.

¿Por qué es importante desarrollar un concepto de \emph{fitness} que no sea tautológico? Porque el concepto de \emph{fitness} es central a la teoría de la evolución por selección natural. Si el concepto es una tautología, entonces no se puede comprobar empíricamente. Nosotros defenderemos que el concepto de \emph{fitness} es empírico y que, por tanto no es tautológico. Para entender por qué este concepto es central a la teoría de la selección natural es porque decimos que hay selección natural cuando los organismos más aptos son los que dejan más descendencia. En contraste, la deriva génica es el proceso en el cual no se involucra a los más aptos, sino que apelamos a una explicación por deriva génica cuando los datos esperados no son los que se reflejan en el estudio, por lo que el concepto de \emph{fitness} no está involucrado en este tipo de explicación.

Para tratar de solucionar el problema de la tautología, parece intuitivo comenzar por leer causalmente la oración anterior: que un organismo tenga más \emph{fitness} causa que tenga más descendencia. Si esto es así, entonces hay que ofrecer una nueva definición de \emph{fitness} que nos permita hacer esto, al mismo tiempo nos permite resolver el problema de la tautología.

Una de las soluciones para este problema es tomar al \emph{fitness} como un primitivo de la teoría y de esta manera salvarnos del problema de la tautología. Esta solución haría que \emph{fitness} no estuviera definido en la teoría y que desaparezca el problema de la tautología. Sin embargo, queremos que el concepto nos sea de utilidad en explicaciones sobre selección natural. Un primitivo sin definir no ayuda para usar el concepto de forma que expliquemos por qué los más aptos son los que dejan más descedencia. Por tanto, necesitamos otra solución a este problema. Tres opciones famosas restan para atacar este problema. Podemos interpretar al \emph{fitness} como una propensión, como una frecuencia relativa, o bien interpretar al fitness como un concepto ecológico.

El concepto de \emph{fitness} como frecuencia relativa es algo que aparece en el artículo de Walsh y compañía \citeyear{Walsh2002}. Ellos mencionan que medir al \emph{fitness} en términos frecuentistas: como la razón entre aquellos que sobreviven sobre el número total de organismos, nos da una manera clara de fijar las probabilidades de que un organismo con cierta característica deje más descendencia. Sin embargo, esta solución tiene algunos defectos. El primero de ellos es que regularmente definimos a las frecuencias como un límite al que se tiende cuando el número de organismos es infinito. Pero las poblaciones que se estudian en biología no son infinitas. Por tanto, las frecuencias relativas no son de utilidad en este caso.

El concepto de \emph{fitness} como propensión trata de resolver el problema anterior. En lugar de definir al \emph{fitness} como el límite de las frecuencias relativas, nos centramos en las propiedades físicas de los organismos y es a partir de las propiedades físicas que determinamos las frecuencias relativas \cite{Mills1979}. Esta es la propuesta de Mills y Beatty. En particular, Mills y Beatty relativizan el \emph{fitness} a una variable ambiental y a las diferencias físicas que hay entre organismos. Por supuesto, no se pueden tomar en cuenta todas las posibles variables que hay en un medio ambiente. Es por ello que Mills y Beatty hablan en términos de un medio ambiente hipotético.

Pensemos, por ejemplo, en una analogía. Un dado con 6 lados tiene las propiedades físicas que nos permiten decir que cada cara tiene $1/6$ de caer en una tirada. Esta probabilidad dadas las propiedades físicas del dado es la propensión que ostenta este aparato en particular. Esta estrategia consiste, entonces, en volver al \emph{fitness} una propiedad disposicional en conjunto a una variable medioamebiental. Las propiedades disposicionales son aquellas que sólo se expresan cuando se realiza un proceso. Por ejemplo, el azúcar es soluble en agua, sin embargo, no observamos la solubilidad en el agua a menos que de hecho pongamos el azúcar en agua. En analogía, los organismos tendrán la propensión de dejar $n$ número de descendientes sin que de hecho se reproduzcan. Esta probabilidad estaría definida a partir de las catracterísticas físicas del organismo (o tipo de organismo) en cuestión. La definición de \emph{fitness} según los propensionistas es: $x$ es más apto que $y$ en un ambiente $E =_{df}$ $x$ tiene una probabilidad más alta de dejar más descendencia que $y$. Esta solución sugiere que es de la composición física que se desprenden las frecuencias que observamos cuando el límite tiende a infinito.

Esta definición nos libra de la carga tautológica que tiene la definición original de \emph{fitness}. Sin embargo, en esta definición queda por aclarar exactamente cómo medir la probabilidad de que $x$ deje mayor descendencia que $y$. Es decir, como podríamos determinar las frecuencias relativas sólo a partir de la composición fisiológica de los organismos. En el caso del dado su composición nos indica cuál será la frecuencia, pero un organismo difícilmente se compara con un dado debido a la gran cantidad de variables que están involucradas. Una opción es volver a la definición frecuentista y medir la frecuencia relativa en un número muy grande de generaciones. La ley de los números grandes nos dirá cuál será la probabilidad a ``la larga'' de que un organismo deje más descendencia que otro. Esto nos hace regresar al problema anterior. Esta lectura no es apta para ayudarnos a explicar, porque nos importa que podamos medir el \emph{fitness} en generaciones finitas y no que la propensión dependa de lo que suceda ``a la larga''. Si la propensión depende de lo que pasará a ``la larga'' y esto es potencialmente infinito, entonces no tendremos acceso epistémico a dicha propensión\footnote{Una segunda opción, que no discutiremos en este trabajo, es optar por una interpretación subjetiva de la probabilidad y asignar a cada organismo una medida según le parezca al biólogo evolutivo. Esto no es totalmente arbitrario, podemos argumentar que un investigador con suficiente información puede asignar una medida de probabilidad subjetiva que refleje las probabilidades de un organismo de dejar cierto número de descendientes. Para un trabajo más detallado de estas tesis, véase \cite{Suarez2021}.}.

Otra opción, que es la que nosotros creemos correcta, es colapsar la definición propensionista a una caracterización ecológica del fitness. Al hacer esto, podríamos medir el fitness en términos de cómo ciertas características se comportan en el medio ambiente y resuelven problemas de diseño. Creemos que la variable medioambiental es necesaria porque un mismo tipo de organismo puede tener diferentes medidas de \emph{fitness} dependiendo del medio ambiente en el que se encuentre. Entonces, para poder salvar al \emph{fitness} de la carga tautológica y aclarar cómo medir esta propensión de los organismos, quisiéramos hacer una propuesta: que podemos definir causalmente al \emph{fitness} utilizando la maquinaria que nos ofrece Woodward. Esto lo haremos apoyándonos en el argumento que exponen Bouchard y Rosenberg \citeyear{Bouchard2004}. Bouchard y Rosenberg defienden una noción de \emph{fitness} ecológico. Si bien en algunos casos se presenta al \emph{fitness} ecológico y al \emph{fitness} como propensión como contradictorias \cite{sep-fitness}, creemos que podemos tomar una parte de ambas nociones a través de utilizar a la teoría de Woodward como modelo causal.

Antes de continuar con el debate acerca del \emph{fitness} y pasar a la interpretación de la teoría de la selección natural, quisiéramos motivar la tesis de la naturaleza dinámica exponiendo algunas explicaciones causales en biología. Nos parece que hacer esto es necesario si es que queremos definir causalmente un término que está al centro de la teoría de la evolución por selección natural. A esto dedicamos en el siguiente apartado.