Este ejemplo es evidencia que se ha estado trabajando en la tesis que afirma que el medio ambiente es un factor que hace la diferencia en los fenotipos. No sólo hay evidencia a favor de esto, sino que además hay evidencia que estas relaciones entre medio ambiente e individuos son un factor relevante para la evolución por selección natural \cite{Jablonka2020, Dayan2020, MacColl2011}. Esto no es lo único importante, si queremos ofrecer explicaciones causales de otros fenómenos biológicos como Eco-Evo-Devo \cite{PfenningEco-Evo-Devo}, Plasticidad fenotípica \cite{WESTEBERHARD20082701}, CGV \cite{CVG}, entonces hay que hablar de causas próximas en biología evolutiva.

Estos problemas con la distinción que hizo Mayr dan entrada a que podamos hablar de relaciones causales en las explicaciones evolutivas. Por lo que nos atrevemos a afirmar que las causas próximas sí están presentes en las explicaciones evolutivas por selección natural.

Esto de nuevo lleva a pensar que la distinción hecha por Mayr deja de ser útil para explicar fenómenos evolutivos\footnote{Los límites de la distinción de Mayr han sido explorados recientemente en \cite{Uller2020, Dayan2020, Laland2011}.}, al mismo tiempo que es compatible con la metodología de Woodward.


Dado que podemos hablar de causas próximas, nos dedicamos en la siguiente sección a hablar en particular de cómo lo dicho hasta ahora está relacionado con el debate acerca de la adecuación.

Este ejemplo cumnple con la noción de intervención que Woodward presenta. Más aún lo que este ejemplo pretende mostrar es que se pueden recrear diferentes condiciones en el laboratorio. Por ejemplo, supongamos que un observador cayó en cuenta de que sus plantas florecían más rápido cuando estaban bajo una sombra que al sol directo. Esto podría ayudar a diseñar condiciones en las que el observador replique lo que accidentalmente observó. Puede entonces crear condiciones en las que mantenga la temperatura igual y que modifique la luz que llega a la planta. O bien puede mantener la luz fija y variar la temperatura. Si todo esto además cumple con las características que pide Woodward, podemos concluir que hay una relación causal entre la floración y/o la temperatura.
