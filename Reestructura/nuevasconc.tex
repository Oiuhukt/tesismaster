\section{Poblaciones o individuos}

\noindent Hasta este momento hemos apoyado la tesis de que el \emph{fitness} es causal en su naturaleza. Para argumentar a favor de la naturaleza estadística, Ariew, Lewens y Walsh argumentan que la selección natural es a nivel de poblaciones y apelamos a las propiedades estadísticas de la población para explicar.

En contra de estos autores, se argumentó que las teorías dinámicas y las teorías estadísticas no son excluyentes en especial, el modelo de Woodward nos ayuda a extraer información causal a partir de la tendencia estadística de una población de organismos. Dijimos además que la selección natural no es indiscriminada, sino que de hecho hay un cierto tipo de selección. Esta noción de \emph{fitness} nos permite separar al proceso evolutivo por selección natural del proceso evolutivo por deriva génica. Esto por supuesto apoya a la tesis de que la selección natural es de naturaleza dinámica.

Ahora bien, queda pendiente si esta interpretación dinámica ocurre a nivel de poblaciones o a nivel de los organismos. Millstein \citeyear{Millstein2006} argumenta que la selección natural es un proceso causal a nivel de poblaciones. El argumento que presenta Millstein a favor de que la selección natural es un proceso causal es que si sólo fuera un proceso estadístico, entonces no podríamos elegir entre diferentes hipótesis. La pura distribución estadística no nos permite distinguir entre si el proceso se debe a deriva génica o bien a selección natural.

Todo lo anterior está en el tenor de lo que hemos argumentado hasta aquí. Que causalidad no es necesariamente determinista y que el modelo de Woodward es útil para elucidar el concepto de causa a través de intervenir en un ambiente de laboratorio. Esto además casa con la metodología utilizada por los biólogos evolutivos. Si asumimos, como hemos hecho hasta aquí que el modelo de Woodward es útil para todo esto, entonces es compatible con lo que la misma Millstein acerca de la interpretación dinámica de la causalidad.

Sin embargo, una vez que argumentamos en favor de una interpretación dinámica de la selección natural, queda decir en qué nivel actúa dicho proceso: si es a nivel de individuos o es a nivel de poblaciones. Millstesin argumenta que la selección natural es a nivel de poblaciones.

El argumento de Millstein comienza con definición de evolución por selección natural como el cambio en la frecuencia genética de una generación a otra. Millstein procede entonces a separar dos tipos de defensas del individualismo: individualismo ingenuo e individualismo sofisticado. El individualismo ingenuo nos dice que hay que rastrear toda la cadena causal de un organismo particular para saber si hay selección natural. Millstein nos dice que esto no nos ayudaría a distinguir hipótesis, porque no podemos discriminar entre si un individuo sobrevive por deriva génica o bien por selección natural. Pensemos en el siguiente ejemplo: un árbol sobrevive a un incendio forestal. Ahora pensemos en dos opciones: 1) con respecto a los demás árboles, el árbol que sobrevivió lo hizo por una característica heredable; 2) el árbol que sobrevivió lo hizo por cuestión de suerte. Si sólo siguiéramos la historia causal de ése árbol en particular no podríamos discriminar entre 1) y 2). Para poder hacer dicha discriminación es necesario hacer un muestreo de la población y una comparación de los genotipos para determinar entre 1) y 2).

Contra el individualismo sofisticado, su argumento es que la tesis termina reduciéndose a la tesis poblacional. En primer lugar, la tesis del individualismo sofisticado es la tesis de Bouchard y Rosenberg expuesta anteriormente: se hace una comparación dos a dos de individuos y se estima cómo algunos de los individuos resuelven problemas de diseño que otros no resuelven. Millstein argumenta que esta comparación dos a dos de individuos se tiene que hacer para absolutamente todos los individuos de la población debido a la existencia de poblaciones no transitivas\footnote{Las poblaciones no transitivas son aquellas en que si bien un organismo A tiene más \emph{fitness} que B y C tiene más \emph{fitness} que A, no necesariamente C tiene más fitness que B. Esto se puede observar en el ciclo de los lagartos estudiados en \cite{Sinervo1996}. De manera muy esquemática, hay tres tipos de lagartos A, B y C. Los lagartos A son muy agresivos y tienden a acaparar a las hembras. A pesar de estro, los lagartos de tipo C, que son muy parecidos a las hembras, se escabullen para aparearse y generan más descendencia que los de tipo A. Ahora que hay más lagartos de tipo C que no acaparan a las hembras, los lagartos de tipo B pueden aparearse y dejar más descendencia que los de tipo C. En este esquema A tiene más \emph{fitness} que B, B más \emph{fitness} que C y C más \emph{fitness} que A}. Millstein dice que una vez que hacemos la jerarquía de cuáles individuos resuelven mejor ciertos problemas de diseño impuestos por el medio ambiente, entonces ya no estamos hablado de individuos, sino de poblaciones.

Tal como dice la misma Millstein, una vez que hacemos la jerarquía de individuos y la resolución de problemas, ya no estamos hablando de individuos, sino de un conjunto de individuos. En esto último detectamos sólo un problema. Como se argumentó en el primer capítulo, hay que hacer una diferencia entre los aspectos epistémicos y metafísicos de la causalidad. Millstein quiere extraer una conclusión metafísica: que selección natural es de naturaleza dinámica y poblacional, a partir de una premisa epistémica: que la única manera en la que podemos obtener información sobre si actúa selección natural es cuando hablamos de un sampleo. Sin embargo, esto cae en el error de reducir los métodos para obtener información causal, con la naturaleza misma de la causalidad.

Sobre esto queremos decir sólo dos cosas. La primera es que lo que dicen Millstein, Bouchard y Rosenberg no es incompatible. Esta comparación dos a dos y la parte que el ambiente juega en esta relación es importante para distinguir entre selección natural y deriva génica. Sin embargo, lo que dice Millstein es insuficiente para concluir que la selección natural es a nivel de poblaciones, lo más sensato dada la distinción entre aspectos epistémicos y ontológicos es que no sabemos en qué nivel actúa. Lo que sabemos es que podemos observar la tendencia de una población a través de un diseño experimental que nos permita intervenir en las variables, que el \emph{fitness} se ve reflejado en la estadística, que no es incompatible la información estadística con la información causal y que no hay problema en hablar de causalidad en biología evolutiva porque no estamos comprometidos con el determinismo causal.
