
\chapter{El problema de la adecuación}

\section{Problema clásico de la adecuación}

\noindent La adecuación es un concepto central en la teoría de la evolución por selección natural. Cuando explicamos un fenómeno de evolución por selección natural, apelamos a que algunos de los organismos de una población tienen características que les ofrecen una ventaja con respecto a otros individuos de la población. Esta característica que les ofrece ventaja tiene que ser una característica heredable para que la selección natural haga su trabajo. Es entonces que decimos que los individuos con esta característica están más adecuados a su medio ambiente.

A pesar de ser un concepto central en la teoría de la evolución por selección natural, hay al menos una cuestión filosófica importante sobre el concepto de adecuación. Esta cuestión es el problema de la tautología que parece tener la noción clásica de adecuación. El problema reside en que si definimos a los organismos más aptos como aquellos que sobreviven o tienen más descendencia, mientras que adecuación está definido en términos de dejar más descendencia o sobrevivencia, entonces obtenemos que los organismos que tienen más descendencia son aquellos que tienen más descendencia. Dicho en palabras de Diane Paul ``[s]i adecuación está definido en términos de éxito de sobrevivencia y éxito reproductivo, entonces el enunciado de que sobrevive el más apto, está aparentemente vacío de contenido'' \footnote{If fitness is defined as success in surviving and reproducing, the statement that the fittest survive is apparently emptied of content.} \citeyear{Paul1992}

Para solucionar este problema Kenneth Waters \citeyear{Waters1986}, nos dice que hay que mostrar como es que podemos definir a la adecuación de manera tal que sea independiente a la medida de descendencia que deja un organismo. Creemos que podemos hacer esto ofreciendo una definición en términos causales de la adecuación, que además apoya a una noción ecológica. ¿Por qué es importante desarrollar un concepto de adecuación que no sea tautológico? Porque el concepto de adecuación es central a la teoría de la evolución por selección natural. Si el concepto es una tautología, entonces no se puede comprobar empíricamente. Nosotros defenderemos que el concepto de adecuación sí es empírico y que, por tanto no es tautológico.

%Para tratar de solucionar el problema de la tautología, parece intuitivo comenzar por leer causalmente la oración anterior: que un organismo sea más adecuado en su ambiente es lo que causa que tenga más descendencia. Si esto es así, entonces hay que ofrecer una nueva definición de adecuación que nos permita hacer esto.

Una de las soluciones para este problema es tomar a la adecuación como un primitivo de la teoría y de esta manera salvarnos del problema de la tautología. Esta solución haría que adecuación no estuviera definido en la teoría y que desaparezca el problema de la tautología. Sin embargo, queremos que el concepto nos sea de utilidad en explicaciones sobre selección natural. Un primitivo sin definir no ayuda para usar el concepto de forma que expliquemos por qué los más aptos son los que dejan más descendencia. Por tanto, necesitamos otra solución a este problema. Tres opciones famosas restan para atacar este problema. Podemos interpretar al adecuación como una propensión, como una frecuencia relativa, o bien interpretar a la adecuación como un concepto ecológico.

El concepto de adecuación como frecuencia relativa es algo que aparece en el artículo de Walsh y compañía \citeyear{Walsh2002}. Ellos mencionan que hay que medir la adecuación en términos frecuentistas: como la razón entre aquellos que sobreviven sobre el número total de organismos, nos da una manera clara de fijar las probabilidades de que un organismo con cierta característica deje más descendencia. Sin embargo, esta solución tiene algunos defectos. El primero de ellos es que definimos a las frecuencias como un límite al que se tiende cuando el número de organismos es infinito. Pero las poblaciones que se estudian en biología no son infinitas. Por tanto, las frecuencias relativas no son de utilidad en este caso.

Otra opción es hacer del concepto de adecuación una propensión. Esto trata de resolver el problema anterior. En lugar de definir la adecuación como el límite de las frecuencias relativas, nos centramos en las propiedades físicas de los organismos y es a partir de las propiedades físicas que determinamos las frecuencias relativas \cite{Mills1979}. Esta es la propuesta de Mills y Beatty. En particular, Mills y Beatty relativizan la adecuación a una variable ambiental y a las diferencias físicas que hay entre organismos. Por supuesto, no se pueden tomar en cuenta todas las posibles variables que hay en un medio ambiente. Es por ello que Mills y Beatty hablan en términos de un medio ambiente hipotético.

Pensemos, por ejemplo, en una analogía. Un dado con 6 lados tiene las propiedades físicas que nos permiten decir que cada cara tiene $1/6$ de probabilidad de caer en una tirada. Esta probabilidad y dadas las propiedades físicas del dado nos da la propensión que ostenta este aparato en particular. La estrategia de Mills y Beatty consiste, entonces, en volver la adecuación una propiedad disposicional en conjunto a una variable medioamebiental. Esto quiere decir que los organismos ostentan una probabilidad de dejar cierto número de descendientes de acuerdo a su configuración fisiológica y el medioambiente en el que se desarrollan los organismos.

Hay un par de cocnepttos que hay que desempacar, por un lado, las propiedades disposicionales son aquellas que sólo se expresan cuando se realiza un proceso. Por ejemplo, el azúcar es soluble en agua, sin embargo, no observamos la solubilidad en el agua a menos que de hecho pongamos el azúcar en agua. En analogía, los organismos tendrán la propensión de dejar $n$ número de descendientes sin que de hecho se reproduzcan. Esta probabilidad estaría definida a partir de las catracterísticas físicas del organismo (o tipo de organismo) en cuestión. La definición de adecuación según los propensionistas es: $x$ es más apto que $y$ en un ambiente $E =_{df}$ $x$ tiene una probabilidad más alta de dejar más descendencia que $y$. Esta solución sugiere que es de la composición física que se desprenden las frecuencias que observamos cuando el límite tiende a infinito.

Esta definición nos libra de la carga tautológica que tiene la definición original de adecuación. Sin embargo, en esta definición queda por aclarar exactamente cómo medir la probabilidad de que $x$ deje mayor descendencia que $y$. Es decir, como podríamos determinar las frecuencias relativas sólo a partir de la composición fisiológica de los organismos. En el caso del dado su composición nos indica cuál será la frecuencia, pero un organismo difícilmente se compara con un dado debido a la gran cantidad de variables que están involucradas en la naturaleza. Una opción es volver a la definición frecuentista y medir la frecuencia relativa en un número muy grande de generaciones. La ley de los números grandes nos dirá cuál será la probabilidad a ``la larga'' de que un organismo deje más descendencia que otro. Esto nos hace regresar al problema anterior. Esta lectura no es apta para ayudarnos a explicar, porque nos importa que podamos medir la adecuación de un organismo en generaciones finitas y no que la propensión dependa de lo que suceda ``a la larga''. Si la propensión depende de lo que pasará a ``la larga'' y esto es potencialmente infinito, entonces no tendremos acceso epistémico a dicha propensión\footnote{Una segunda opción, que no discutiremos en este trabajo, es optar por una interpretación subjetiva de la probabilidad y asignar a cada organismo una medida según le parezca al biólogo evolutivo. Esto no es totalmente arbitrario, podemos argumentar que un investigador con suficiente información puede asignar una medida de probabilidad subjetiva que refleje las probabilidades de un organismo de dejar cierto número de descendientes. Para un trabajo más detallado de estas tesis, véase \cite{Suarez2021}.}.

Otra opción, que es la que nosotros creemos correcta, es colapsar la definición propensionista a una caracterización ecológica de la adecuación. Al hacer esto, podríamos medir la adecuación en términos de cómo ciertas características se comportan en el medio ambiente y resuelven problemas de diseño\cite{sep-fitness}. Creemos que la variable medioambiental es necesaria porque un mismo tipo de organismo puede tener diferentes medidas de adecuación dependiendo del medio ambiente en el que se encuentre.  Esta noción de adecuación nos permite resolver el problema de la tautología y tiene la ventaja de que no depende de utilizar frecuencias relativas y que, por tanto, no dependa de que haya poblaciones infinitas.

Sin embargo, esta noción de adecuación está lejos de estar libre de problemas. Uno de ellos siendo que depende de una metáfora poco clara: ¿exactamente qué queremos decir con ``resolver problemas de diseño''? Para poder salvar la adecuación de la carga tautológica y defender una noción ecológica de adecuación, quisiéramos hacer una propuesta: que podemos definir causalmente la adecuación de un organismo utilizando la maquinaria que nos ofrece Woodward. Con esto queremos decir que en un ambiente controlado, podemos individuar el problema de diseño que resuelve, de acuerdo a intervenir en las variables ambientales y ver si el cambio generado apunta a lo que vemos en la naturaleza.

La maquinaria ofrecida por Woodward, decimos nosotros, puede aclarar cuáles son los ``problemas de diseño'' que resuelven los organismos al poder cambiar una variable en el laboratorio y observar lo que vemos en la naturaleza. Esto lo haremos apoyándonos en el argumento que exponen Bouchard y Rosenberg \citeyear{Bouchard2004} donde defienden una noción ecológica de adecuación. sin embargo, ellos defienden que para poder evaluar la adecuación, hay que hacer una comparación dos a dos de todos los individuos de la población. Debido a que hay poblaciones no transitivas, como las descritas por Sinervo y Lively \cite{Sinervo1996}, esta comparación dos a dos tiene que hacerse para todos y cada uno de los individuos de la población \cite{Millstein2006}. Creemos que adoptar el modelo de Wooidward tiene la ventaja de no necesitar hacer una comparación dos a dos de todos los individuos en la población, al mismo tiempo que nos permite distinguir entre evolución por selección natural y evolución por deriva génica.

Antes de continuar con la defensa de un criterio ecológico de adecuación, queremos presentar el modelo de Woodward. Creemos que el modelo de Woodward es consistente con cómo se trabaja en biología evolutiva. Nos apoyamos en un par de ejemplos para decir que la metodología del modelo de explicación causal de Woodward  nos ofrece un criterio de explicación cercano a los experimentos realizados en biología evolutiva.

\section{El modelo de explicación causal de Woodward}

En la postura de Woodward, son importantes las nociones de ``invarianza'' e ``intervención''. Estas son importantes porque a partir de estas nociones se define una generalización que, a pesar de no ser una ley de la naturaleza, es explicativa. Si el modelo de Woodward logra hacer lo que se propone, entonces tendremos un modelo de explicación que no apele a leyes y, por tanto, más adecuado para ciencias como la biología. Además, retomando causalidad, evitamos la simetría de la que pecaba el modelo de Hempel y resolvemos los problemas de correlación del modelo de Salmon. Además, es claro que muchas de nuestras explicaciones exitosas (si bien no todas las explicaciones exitosas) son causales.

\textit{Grosso modo} Woodward nos dice que explicar tiene que ver con hacer explícitas las relaciones causales entre dos variables: sean ``$A$'' y ``$B$'' dos eventos cualquiera, decimos que el evento $A$ explica al evento $B$ cuando hay una relación causal que liga la ocurrencia de $A$ con la ocurrencia de $B$. Esto es sólo cuando al manipular $A$, el valor de $B$ cambia en consecuencia. El hecho de que haya una intervención no implica que tenga que haber un agente. Intervención está definido de manera que fenómenos naturales en los que sucede un evento $A$ y esto a su vez modifica el valor de otro evento $B$ cuenta como una intervención en el modelo de Woodward.

 Para asegurar que esta relación es causal, este cambio en $B$ debe estar relacionado sólamente con los cambios en $A$ y no deberíamos poder explicar el cambio en $B$ por intervención directa. De manera más esquemática la noción de explicación de Woodward es definir una relación $R$ tal que $R<A, B>$ esté constreñida por las siguientes características: i) cambios en el valor de $B$ deben estar directamente relacionado con cambios en el valor de $A$ de manera que sin cambios en $A$, no habría habido cambios en $B$\footnote{Este criterio es evidentemente modal}; ii) mediante $R$ debemos ser capaces de hacer una ``generalización'' tal que dicha generalización nos describe el comportamiento del sistema en los casos donde la relación es invariante (que son casos en los que bajo ciertas restricciones si ocurre $A$, entonces ocurre $B$)\footnote{Supongamos por ejemplo que quiero saber bajo qué condiciones un vaso que se cae, se rompe. Podríamos variar la altura de la caída, así como el material sobre el que cae. Si, por ejemplo, tiráramos un vaso en un colchón, a 10 cm. de altura, no se romperá; si lo tiráramos de una altura de dos metros en un piso de concreto, seguro se romperá. Esto quiere decir que la relación entre tirar un vaso y que se rompa en consecuencia es invariante bajo algunos valores de altura y del material sobre el que cae.}.  iii) $A$ hace un cambio en $B$ y el cambio en $B$ no debe darse por ninguna otra ruta; iv) No hay causas diferentes a $A$ que cambien a $B$ (ya sea una causa común o alguna otra razón), por último todo debe estar acotado a i-iv \cite[p. 201]{Woodward2000}. Todo esto constituye la noción de intervención. Si diseñamos una manera en la que podamos intervenir en $A$, que cambié el valor de $B$ en consecuencia y dicha intervención cumple las características i-iv, entonces tenemos una explicación \textbf{causal} de la ocurrencia de $B$.

 Pongamos un ejemplo: la luz y la temperatura, por separado, afectan el proceso de floración en plantas del género \emph{Arabidopsis}. En el artículo \cite{AusinEnviro}, los autores desarrollan un modelo experimental en el que observaron que hay una relación entre la temperatura y la luz afectan cómo florecen las plantas del género \emph{Arabidopsis}. Para el experimento se sometió a los especímenes a temperaturas bajas (aunque no al punto de congelamiento) y observaron cómo el proceso de floración se acelera en consecuencia. Esto según la metodología de Woodward nos permite concluir que hay una relación causal entre la temperatura y la floración. En el caso de la luz, se observó que los especímenes reaccionan a la luz roja, a la luz roja lejana (longitudes de onda entre 700 y 750 nm.) y a la luz azul. Cuando hay bajos niveles de estas tres, se promueve la floración. Esto indica que hay una relación causal entre los bajos niveles de este tipo de luces y la floración. Para esto, se interviene en las condiciones de temperatura y de luz para poder concluir que hay una relación causal.

 \begin{center}
   \begin{tikzpicture}[scale=2]
     \node (X) at (1,1){Floración};
     \node (A) at (0,0){Temperatura};
     \node (B) at (2,0){Luz};
     \path[-angle 90,font=\scriptsize]
     (A) edge    (X)
     (B) edge    (X);
     \end{tikzpicture}
 \end{center}

 Este diseño experimental en el que somos capaces de intervenir variables es compatible con la metodología de Woodward. Algo parecido sucede en el caso de \emph{A. sagrei} y \emph{L. Carinatus}, se interviene la variable depredador y se observa que hay un cambio en el tamaño de las extremidades de \emph{A. sagrei}. Que haya un nuevo depredador en el medio de \emph{A. sagrei} causa que se seleccionen los organismos que tienden a tener extremidades más largas. Por lo que podemos fijarnos en como el fenotipo interactúa con el ambiente.

Pensemos en un ejemplo más: El experimento realizado por Amarillo Suárez y Fox \citeyear{Amarillo-Suarez2006}. Hay insectos que se desarrollan dentro de un hospedero. Se tiene evidencia que el hospedero en el que se desarrollan las crías tiene influencia en el tamaño de los insectos. En el artículo de Amarillo-Suárez y Fox, se explora cómo el hospedero del \emph{Stator limbatus} que puede hospedarse en dos tipos de árbol: \emph{Acacia greggi y Pseudosamanea guachapele}, tiene consecuencias en su desarrollo. Las particularidades de estos árboles es que el \emph{Acacia greggi} tiene unas semillas más grandes que el \emph{Pseudosamanea Guachapele}. Se analizó cómo varía el tamaño de los insectos cuando el hospedero es un árbol u otro. El resultado experimental mostró que cuando este insecto se hospeda en el árbol con las semillas más grandes, los organismos son de mayor tamaño. Este mayor tamaño es independiente al tamaño de los progenitores. Según las autoras del artículo esto indica plasticidad fenotípica. El diseño experimental se encarga de mantener a la población fija, es decir, sortea de manera aleatoria en qué hospedero pondrán a qué organismos. Al hacer esto, controlamos por el tamaño de los organismos. Esto quiere decir que hay al menos dos maneras de intervenir en el tamaño de los organismos. La más obvia es el tamaño de los padres, la segunda es las semillas que tiene el hospedero. En el diseño se controla por el tamaño para que no afecte como variable. Al eliminar como factor el tamaño de los padres, podemos intervenir directamente en qué hospedero poner a los organismos.

 \begin{center}
   \begin{tikzpicture}[scale=2]
     \node (X) at (1,1){Tamaño de los organismos};
     \node (A) at (0,0){Tamaño de los padres};
     \node (B) at (2,0){Semillas};
     \path[-angle 90,font=\scriptsize]
     (A) edge   (X)
     (B) edge   (X);
     \end{tikzpicture}
 \end{center}


 Para poder controlar por el tamaño de los padres, se hace una selección aleatoria de una población y se asigna en los dos diferentes hospederos. Lo que nos permite intervenir directamente en el tamaño de las semillas al alocar a los organismos en los dos respectivos árboles.

 \begin{center}
   \begin{tikzpicture}[scale=2]
     \node (X) at (1,1){Tamaño de los organismos};
     \node (A) at (0,0){Tamaño de los padres};
     \node (B) at (2,0){Semillas};
     \path[-angle 90,font=\scriptsize]
     (A) edge    (X)
     (B) edge [dotted]   (X);
     \end{tikzpicture}
 \end{center}

Si después se observa que a pesar de que las alturas de una generación eran, en promedio, iguales en los dos árboles y que las nuevas generaciones de organismos son más grandes en el árbol con las semillas más grandes, entonces podemos afirmar que hay una relación causal en términos de la metodología de Woodward. En términos de lo que afirma la teoría de Woodward, hay una relación causal entre el medio ambiente y los organismos que lo habitan. Más aún, hay intentos de exponer que un enfoque manipulabilista puede ser de utilidad en las explicaciones por selección natural \cite{MacColl2011}.

Con respecto a la noción de invarianza, Woodward nos dice que cualquier generalización que describa una relación entre dos o más variables es invariante si se sostiene aún cuando se modifican otras condiciones. Esta noción de invarianza es lo que permite hacer generalizaciones de la relación entre dos variables. Porque si hay una relación causal entre $A$ y $B$ y dicha relación se sostiene aún cuando otras variables se modifican, entonces podemos decir que para cualquier $A$ y $B$ habrá la misma relación causal. Cuando esto se cumple, tenemos un indicio de que es posible manipular y controlar la variable independiente para ver qué cambios hay en la variable dependiente \cite{Woodward2000}.

Sin duda el modelo de Woodward tiene muchas virtudes. Primero tiene una aplicación clara para las ciencias especiales ya que no parte de la noción de ley, sino que construye generalizaciones como ``invarianza bajo intervenciones''. Woodward resuelve los problemas que tenía el modelo de Salmon al poner más restricciones en lo que deberíamos hacer cuando buscamos relaciones de dependencia causal. Otra virtud es que la noción de intervención encaja con el hecho de que en las investigaciones se llevan a cabo experimentos y que es a partir de ello que obtenemos información que indica si hay o no una relación entre variables.

\section{Adecuación, un concepto causal}

\noindent Como dijimos anteriormente la adecuación es un concepto central de la teoría de la evolución por selección natural. En particular, decimos que la adecuación es medido en términos de descendencia y esto quiere decir que la selección natural opera en este tipo\footnote{En este caso hablamos de \textbf{tipos} de organismos, esto es, que compartan cierta característica física.} de organismo cuando, en ausencia de otros factores (por ejemplo, deriva génica) un organismo deja más descendencia que otro.

El concepto de adecuación como propensión tiene el problema de cómo tenemos acceso epistémico a dichas propensiones. Mencionan Mills y Beatty \citeyear{Mills1979} que esto se puede hacer de acuerdo a las características físicas de un organismo, tomando como variable el medio ambiente. Esto sugiere que un organismo que tenga un nivel de adecuación $x$ tendrá un nivel diferente en otro medio ambiente. Pero es en cómo determinamos este nivel de adecuación que está el problema porque no podemos controlar todas las variables que pueden actuar en el medio ambiente en el que se desarrolla el organismo.

 Nosotros queremos decir que el nivel de adecuación de un organismo se puede determinar sólo en los resultados de un diseño experimental. Esto es crucial para poder incorporar la noción de intervención que requiere el modelo de Woodard. Es decir en el medioambiente del laboratorio. Esto nos daría acceso a las propensiones de los organismos, resolviendo el problema de la definición propensionista. La parte causal en esta noción de adecuación estaría definida de la mano del intervencionismo de Woodward al interactuar con las variables dentro del laboratorio. Si todo esto es correcto, entonces tenemos una noción empírica de adecuación, al mismo tiempo que nos deshacemos del problema de la tautología. Nos deshacemos de este problema porque los más aptos son aquellos que resuelven mejor problemas de diseño y que, como resultado, son los que dejan más descendencia.

 El modelo de Woodward está relacionado con el debate sobre la interpretación de adecuación. Mostramos en la sección anterior cómo el modelo nos ayuda para extraer información causal a través de los diseños experimentales y la manera en que podemos intervenir en las variables y observar cómo se modifican otras variables. Al poder hacer esto, podemos definir la adecuación a través de estos modelos experimentales. Es en estos modelos en dónde podemos observar cómo se desarrollan los organismos al interactuar con el medio ambiente. Al intervenir en las diferentes variables, podemos determinar cuál es la causa de que un tipo de organismos resulte ventajoso. Todo esto en un ensamble de laboratorio y sin necesidad de hacer una comparación dos a dos de cada uno d elos individuos. En este sentido la adecuación de un tipo de organismo se sigue midiendo en tanto número de descendientes. La ventaja es que en el diseño experimental podemos ver la tendencia del organismo, determinando así la característica ventajosa. Al ser una intervención, esta explicación cuenta como causal en términos de la maquinaria que nos ofrece Woodward.

 Esto nos permite decir exactamente cuál es el organismo más apto dependiendo del medio ambiente en el que se desarrolla, lo que tiene la ventaja de que tenemos una medida de adecuación que explica por qué unos organismos tienen ciertas características y si estas características son mejores para resolver problemas de ``diseño'' que otras características en la población. Esto nos deja con una definición de adecuación que no  es tautológica y que es  explicativa. El valor explicativo se desprende de estar inserto en el modelo de Woodward. El modelo también nos permite que la definición sea causal. Por lo anterior nos parece que esta definición de adecuación es mejor que la propensionista.

 Aceptar esta definición de adecuación tiene las ventajas de dar un concepto más claro. Este concepto está relativizado a variables medioambientales. Al mismo tiempo nos permite leer causalmente la adecuación al decir exactamente cuál es la característica ventajosa que permitió resolver los problemas de diseño impuestos por el medioambiente. Esta definición además nos permite explicar cuál es el factor relevante para la sobrevivencia de los organismos.

 Tiene además la consecuencia de que la única manera en que podemos medir qué organismos son más aptos es cuando ya hay una cierta adaptación en el diseño experimental. Pensemos, por ejemplo, en el caso de \emph{A. sagrei} y \emph{L. carinatus}. En este diseño experimental se observa una tendencia al crecimiento de extremidades que les permitan escalar. Los organismos que mejor resuelven este problema y que, por ello, dejan más descendencia son aquellos con extremidades más largas. Aquí tenemos una explicación apelando a la selección natural. Pero es sólo en este diseño experimental y una vez observada esta tendencia que podemos decir que los más aptos, en este medioambiente particular, son los que desarrollan extremidades más largas. Es sólo cuando tenemos esta información que podemos utilizarla para explicar.


 \section{Adecuación y la interpretación dinámica de la Selección natural}

 \noindent Como también mencionamos, hay un debate entre cuál es la mejor manera de interpretar la teoría de la evolución por selección natural. Por un lado Bouchard y Rosenberg argumentan en favor de una interpretación dinámica (es decir que involucra fuerzas) \citeyear{Bouchard2004}; por otro lado Walsh, Lewens y Ariew \citeyear{Walsh2002} argumentan que la teoría no es sobre fuerzas, sino sobre consecuencias puramente estadísticas. Si aceptamos la noción de adecuación en términos de lo dicho anteriormente, ¿qué consecuencias hay para el debate de la interpretación de la selección natural?

 Walsh y compañía argumentan a favor de una interpretación estadística de la teoría de la selección natural. Pensemos, por ejemplo, en un grave al que dejamos caer de cierta altura. En este caso, podemos describir las fuerzas que hacen que caiga y podemos predecir el lugar en el que el grave de hecho va a caer. En el caso de las monedas, el hecho de que una $x$ cantidad de monedas caiga en cara y una $y$ cantidad caiga en cruz, no depende de las fuerzas actuando en cada moneda particular. Lo que observamos es consecuencia de la estructura de la población.

 En ambos casos hay dos tipos diferentes de error. El error en el caso de un grave que cae dependerá de que no tomamos en cuenta todas las fuerzas actuando para ser capaces de predecir el lugar de caída. En el caso de las monedas, el error es intrínseco a la probabilidad de las monedas. Debido a esta diferencia entre teorías dinámicas y teorías estocásticas, cabe la pregunta de cómo interpretamos a la teoría de la evolución.

 Walsh y compañía asumen que ambos tipos de interpretación son excluyentes. El argumento de Walsh y compañía descansa en que cuando buscamos una explicación en selección natural, apelamos a las estadísticas de la población que estamos observando. Para argumentar en favor de una interpretación estadística de la selección natural, primero hacen un caso a favor de la interpretación estadística de la deriva génica. Para esto, argumentan que la noción de deriva génica implica una noción de azar. Aunado a lo anterior, los autores asumen que cuando hablamos de fuerzas, caemos directamente en el determinismo. Debido a que hay un factor de azar en cómo se estudian las poblaciones, entonces la teoría de la selección natural no es una teoría sobre fuerzas. Si esto es verdad, ``Si todos los hechos fueran conocidos, cualquier proceso que cause un cambio en las frecuencias de una característica contaría como un proceso de selección. En consecuencia, no habría un proceso llamado `deriva génica' '' \citeyear[p. 457]{Walsh2002} \footnote{If all the facts were known, any process that causes a change in trait frequencies would be counted as a selection process. Consequently, there is no such process or force as ``random drift''}.

 Pero el consecuente en esta afirmación, continúa el argumento, es falso. Si esto es verdad, entonces la deriva génica no es una explicación que apele a fuerzas y, por tanto es una teoría estadística. Para el caso a favor de la interpretación estadística de la selección natural, Walsh y compañía argumentan que cuando buscamos una explicación por selección natural, apelamos a propiedades estadísticas de la población y no a las muertes y nacimientos individuales. Esto asume además que la única manera en que pueden estar involucradas las fuerzas en la selección natural es al nivel de individuos particulares.

 Hay razones para rechazar algunos supuestos que hacen los autores. La primera razón es que no es excluyente la distnición entre una teoría dinámica y una teoría estadística. Al menos es claro por el modelo de Woodward que podemos apelar a propiedades de una población, intervenir en las variables y extraer información causal acerca de las fuerzas que llevaron a que la población en general tuviera cierta tendencia. Esto quiere decir que podemos extraer relaciones causales a partir de una tendencia estadística y a través de un diseño experimental. En el diseño experimental tomamos información de la estadística de la población y extraemos una conclusión causal. Por tanto, no son exluyentes ambos tipos de teorías.

Por otro lado, dado lo dicho en la sección anterior, decimos que cuando actúa selección natural en una población, esperamos que algunos de los organismos sean más aptos que otros en un medio ambiente al resolver mejor los problemas de diseño impuestos por el ambiente. Cuando decimos que deriva génica actúa en una población es porque la población de alguna manera se ha desviado de lo que habíamos esperado fueran los organismos más aptos. Por decirlo de otra manera, cuando observamos una tendencia en nuestro diseño experimental y lo que observamos se desvía de lo que de hecho ocurre naturalmente, entonces apelamos a una explicación por deriva génica. En esto estamos de acuerdo con Walsh y compañia, que deriva génica no es una fuerza, sino una consecuencia estadística. Estamos también de acuerdo en lo que argumenta Lange \citeyear{Lange2013}, cuando pedimos una explicación en términos de selección natural buscamos una explicación causal. En cambio, cuando esta explicación causal falla debido a algún evento y la expectativa que teníamos del crecimiento de la población se ve afectada, entonces apelamos a una explicación por deriva génica. Esta es una Explicación realmente estadística porque no pedimos una explicación que apele a causas. Es decir que es una consecuencia púramente estadística.

 En la sección anterior dijimos por qué es ventajoso aceptar una interpretación causal de adecuación. Entre las razones estuvo que es un concepto de adecuación que es explicativo, pero además nos permite distinguir entre deriva génica y selección natural en el sentido explicado anteriormente. Retomemos por un momento lo que argumentan Bouchard y Rosenberg \citeyear{Bouchard2004} nos dicen que la definición tradicional de adecuación tiene tres problemas. El primer problema está relacionado con las frecuencias si adecuación tuviera una interpretación puramente estadística, entonces decimos que ``a la larga'' el organismo más apto dejará más descendencia. Sin embargo, nos interesa que el tiempo sea finito, porque de otra manera no tendríamos acceso a dichas frecuencias. Nosotros resolvimos esto al medir la adecuación de un organismo en términos de lo que se observa en el laboratorio, por lo que no es necesario adoptar el frecuentismo. El segundo problema que tiene una definición de este tipo es que es una tautología. Se supone que los organismos más aptos, esto es, con más adecuación son aquellos que dejan más descendencia. Pero si definimos el grado de adecuación de un organismo como aquél que deja más descendencia, entonces los organismos que dejan más descendencia son aquellos organismos que dejan más descendencia. Para resolver este problema, nosotros apelamos a una definición ecológica de adecuación donde medimos qué tan apto es un organismo en términos de un diseño experimental, que además nos permita medirlo a través de la tendencia que tiene la población al someterla a presiones selectivas. El último problema es de naturaleza puramente biológica: no siempre el organismo que deja más descendencia es el organismo más apto.

 Nosotros, para resolver el segundo problema, apelamos a una nodión ecológica de adecuación. Sin embargo, hay al menos dos conceptos poco claros en la definición anterior. La comparación dos a dos en una población puede volverse complicada tomando en cuenta el tamaño de la población. Para poblaciones grandes este proceso podría tardar un tiempo indefinido. Creemos que la definición propuesta por nosotros salva este problema al permitir explicar cuál es el organismo más apto recreando las presiones ambientales en nuestro diseño experimental.

 El segundo concepto poco claro es el diseño. Esto puede sugerir que la selección natural está guiada por un diseñador. Sin embargo, podemos apoyarnos en lo que dice Ayala. Ayala argumenta que uno de los grandes aportes de Darwin es haber descrito un mecanismo en el cual podemos hablar de diseño sin que haya un diseñador \cite{Ayala2004}. Si podemos hablar en estos términos, entonces hablar de diseño en los organismos nos permitirá, en la propuesta de Bouchard y Rosenberg y en la nuestra, hacer una evaluación en términos de cómo los organismos resuelven problemas de diseño. Esto nos da una pauta para hablar de que el medio ambiente es un factor causal en las explicaciones evolutivas. Esto, en conjunción con lo que proponemos es un buen método para obtener información causal para sustentar hipótesis, nos permite hacer una distinción entre deriva génica y selección natural. Además, el método de Woodward para rastrear relaciones causales nos permite evitar la comparación dos a dos del concepto ecológico que presentan Bouchard y Rosenberg. Este proceso queda cubierto por el diseño experimental en el que recreamos las condiciones ambientales que sospechamos hacen una diferencia en la supervivencia de los organismos. Es cuando los organismos resuelven estos problemas de diseño que apelamos a una explicación por selección natural. Cuando hacemos el conteo poblacional y no observamos lo que se esperaba, entonces apelamos a una explicación por deriva génica. Esto, creemos, resuelve los problemas diagnosticados por Bouchard y Rosenberg. Sin embargo, hay algunos problemas al hablar de causalidad en biología. Estos problemas están expuestos por Mayr \cite{Mayr1998}. La siguiente sección la discusión de lo señalado por Mayr


\section{Causas próximas y causas últimas}

Hay algunos problemas al hablar de causalidad en biología. Lo que haría que un modelo de explicación causal como el de Woodward parezca inadecuado a la hora de aplicarlo al caso de la biología evolutiva. Por lo que vale la pena hacer un par de distinciones. Estos problemas están identificados en el artículo de Ernst Mayr \citeyear{Mayr1998} donde habla de causalidad en biología evolutiva. En primer lugar, Mayr señala que hay dos grandes campos de la biología: la biología funcional y la biología evolutiva.

En segundo lugar, Mayr nos dice que el biólogo funcional trata a su objeto de estudio como un solo individuo y su método es principalmente la experimentación. El biólogo evolutivo por su parte, trabaja con fenómenos extendidos temporalmente y no se preocupa del ambiente próximo de los organismos.

Desde esta distinción, Mayr argumenta que no es lo mismo estudiar la causalidad en ambas áreas: mientras que el biólogo funcional se ocupa de causas próximas, el biólogo evolutivo se ocupa de causas últimas. Con estas distinciones, Mayr afirma que ``[...] las causas próximas son las que gobiernan las respuestas de los individuos (y sus órganos) a factores inmediatos del ambiente, mientras que las causas últimas son responsables de la evolución del programa de información del ADN [...]'' (p. 86).

Mayr presenta a estos dos dominios como complementarios aunque excluye del ámbito del biólogo evolutivo las causas próximas. Dice de la biología funcional que ``La técnica principal del biólogo funcional es el experimento y su aproximación es esencialmente la misma que la del físico o la del químico'' (p.83). Mientras que la biología evolutiva es autónoma de las ciencias físicas o quúimicas.

Pero la tesis de Mayr no está lejos de ser controvertida. Recientemente han habido críticas a la distinción trazada por él. El negar que haya causas próximas en biología evolutiva excluye aspectos de la práctica biológica que son intuitivamente relaciones causales próximas. Por ejemplo, cuando hay un cambio en el medio ambiente y los organismos responden a este cambio de manera que se modifica el fenotipo para poder adaptarse, como pasa en el fenómeno CVG\footnote{Entre otros fenómenos en los que hay causas próximas y son fenómenos evolutivos están la plasticidad fenotípica \cite{WESTEBERHARD20082701} y el Eco-Evo-Devo \cite{PfenningEco-Evo-Devo}} \cite{CVG}. Además, deja de lado toda la nueva literatura que ha explorado la epigenética, por ejemplo en el fenómeno de la construcción de nicho. En este fenómeno los organismos modifican su medio ambiente. Este medio ambiente se hereda a la progenie, ya que es el lugar en el que habitarán. Este medio heredado es una causa próxima. Si eliminamos las causas próximas del ámbito evolutivo, estaríamos excluyendo estos fenómenos. Esto está relacionado con admitir nuevos mecanismos de herencia que no son heredados por la vía sexual.

Endler en su libro \citeyear{Endler1986} ofrece una caracterización más amplia de herencia. Endler nos dice que no necesariamente el fenómeno de la herencia depende de la herencia genética. El autor menciona que debe haber una relación consistente entre que algún organismo tenga una cierta característica que le permita una mejor habilidad de apareamiento, o bien una mejor habilidad de fertilización, o bien una mejor fecundidad y/o sobrevivencia (siempre con miras a que estas características no son necesariamente genéticas).

Ahora bien, la síntesis moderna fue bastante reticente en aceptar cualquier otro mecanismo de herencia que no fuera la herencia genética. Por lo que se excluyeron otros tipos de explicación y mecanismos no genéticos, o bien que no tuvieran una reducción a la genética. Ya desde hace varios años se ha explorado si es necesario incorporar nuevos mecanismos de herencia \cite{Jablonka2020}. De ser esto correcto, estos nuevos mecanismos de herencia y su incorporación a la teoría tienen consecuencias importantes en qué evidencia debemos tomar para justificar hipótesis en selección natural y por supuesto ponen en tela de juicio la distinción que Mayr propone.

En este trabajo no vamos a discutir las consecuencias filosóficas y biológicas que tiene el integrar nuevos mecanismos de herencia. Lo que queremos hacer es motivar el que haya una aplicación del modelo de explicación causal de Woodward al ofrecer ejemplos de biología evolutiva donde hay una relación inmediata con el ambiente en el que se desarrollan los organismos. Es por ello que afirmamos que si hay herencia no genética, entonces podemos incorporar causas próximas en las explicaciones evolutivas. Más aún, si podemos incorporar causas próximas para explicaciones en selección natural, entonces no es controvertido decir que necesitamos un concepto de ``causalidad'' que permita incorporar dichas causas próximas. Como vimos en la sección 1.2 motivamos al modelo de Woodward como modelo causal para explicaciones evolutivas al ofrecer un concepto de adecuación en consonancia con el modelo que propone Woodward.

Una advertencia antes de continuar: el que debamos integrar una dimensión ambiental es aún un debate abierto. Por lo que en un futuro la investigación puede encontrar una manera de reducir los factores ambientales a factores genéticos, lo que haría que debamos excluir la dimensión ambiental de las explicaciones evolutivas. Esta estrategia es la que siguen algunos de los biólogos que aún defienden algún tipo de neo-darwinismo. Esto sin duda es posible, pero la investigación actual en biología apunta en la otra dirección  \cite{Bateson2014}. Hay investigaciones que de hecho incorporan una dimensión ambiental y que afirman que hay otros medios de herencia no necesariamente genéticos. Esto es indicio de que podemos hablar de causas próximas en biología evolutiva. Por lo que en lo que resta de este trabajo supondremos que este es el caso.

A pesar de la reticencia de algunos biólogos, se han investigado sistemas de herencia distintos que no son herencia genética. Por ejemplo en el libro de Jablonka y Lamb \citeyear{Jablonka2020}, las autoras argumentan que hay diferentes mecanismos de herencia que no necesariamente están al nivel genético. Entre ellos se encuentran los mecanismos epigenéticos y la imitación del comportamiento. La construcción de nicho también es un caso en el que puede haber herencia no genética. Aún con la evidencia experimental, las autoras comentan que se ha relegado este tipo de mecanismos con el argumento de que son triviales porque no hacen una diferencia en términos evolutivos. Sin embargo, las autoras argumentan que hay evidencia de lo contrario.

En el artículo de Uller y compañia \citeyear{Uller2019} encontramos un ejemplo concreto que pretende mostrar que las causas próximas no están excluídas del ámbito del biólogo evolutivo. Estudiando a las ballenas asesinas, los investigadores dan cuenta de que las ballenas han adaptado su dieta localmente y han desarrollado técnicas de caza particulares. Los estudios muestran que estas diferencias no se deben a variación genética, sino al aprendizaje social. El aprendizaje social es un caso de causa próxima tal como lo describe Mayr. Entonces las causas próximas sí son relevantes para la biología evolutiva ya que estas diferencias en técnicas de caza son características que pueden ser seleccionadas.

Una estrategia para mantener la distinción entre causas próximas y causas últimas es cambiar el foco de atención, alguien podría sugerir que no es interesante el hecho de por qué diferentes poblaciones de ballenas han desarrollado distintos métodos de caza por aprendizaje social, sino que la pregunta interesante es por qué las ballenas asesinas desarrollaron la característica de ser capaces de aprender socialmente.

Sin embargo, esta solución tiene problemas. En primer lugar esta solución cambia el \emph{explanandum} de la pregunta original. El \emph{explanandum} de la pregunta original es por qué los diferentes métodos de caza de las ballenas asesinas han llevado a variación entre las distintas poblaciones. Nos interesa saber por qué las diferentes poblaciones tienen estos métodos particulares, no por qué las ballenas tienen la característica de aprendizaje social, que es una pregunta diferente.

Un ejemplo que podría ilustrar aún más esto que mencionamos es el experimento de las mariposas \textit{Bycyclus anynana}. En \cite{Frankino2007}, los investigadores prueban la hipótesis de que el tamaño de las alas de las mariposas se debe en mayor medida a selección natural y no a restricciones de desarrollo del organismo. Aunque las restricciones de desarrollo guían el tipo de alometrías posibles, la selección natural actúa para favorecer un tipo sobre otro según los resultados del artículo.

Para ver qué tanto afecta selección natural en la alometría de \emph{Bicyclus anynana}, seleccionaron artificialmente a los individuos para guiar el desarrollo de las alometrías que tuvieran las alas posteriores más grandes y las alas anteriores más chicas; además también seleccionaron artificialmente a aquellos individuos con las alas posteriores más chicas y las alas anteriores más grandes. Llegado un punto, se dieron cuenta de que estas mariposas podían tener una variación de tamaños exagerada, diferente a los que vemos en la naturaleza. Por lo que sugierenm que el medioambiente se encarga de hacer que las maripoisas tengan las alometrías que vemos en la naturaleza.

Debido a que estas alometrías no son imposibles y que no están completamente determinadas por restricciones de desarrollo, la selección natural es el proceso principal que determina las alometrías que observamos en el entorno natural. Esta explicación es en términos de causas próximas ya que apelan al medioambiente inmediato.

Estos ejemplos no sólo pretenden ilustrar que la teoría de la causalidad de Woodward es compatible con el quehacer del biólogo evolutivo, sino además ofrecer evidencia para lo que argumentan Jablonka y Lamb, a saber, que hay una importante injerencia del medio en el que se desarrollan los organismos y que hay herencia no necesariamente genética.  Esto nos lleva, contra la suposición de Mayr, a poder incorporar causas próximas en explicaciones de selección natural.

Hasta este momentro hemos utilizado el modelo de Woodward sin más justificación que algunos ejemplos en los que se pretende mostrar que es consistente con la práctica biológica. El siguiente capítulo está dedicado a motivar cómo el modelo de Woodward resuelve algunos problemas de otros modelos de explicación científica. 
