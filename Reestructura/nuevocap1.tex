
\chapter{El problema de la adecuación}

\section{Problema clásico de la adecuación}


\noindent La adecuación es un concepto central en la teoría de la evolución por selección natural. Cuando explicamos un fenómeno de evolución por selección natural, apelamos a que algunos de los organismos de una población tienen características que les ofrecen una ventaja con respecto a otros individuos de la población. Esta característica que les ofrece ventaja tiene que ser una característica heredable para que la selección natural haga su trabajo. Es entonces que decimos que los individuos con esta característica están más adecuados a su medio ambiente.

A pesar de ser un concepto central en la teoría de la evolución por selección natural, hay al menos una cuestión filosófica importante sobre el concepto de adecuación. Esta cuestión es el problema de la tautología que parece tener la noción clásic de adecuación. El problema reside en que si definimos a los organismos más aptos como aquellos que sobreviven o tienen más descendencia, mientras que adecuación está definido en términos de dejar más descendencia o sobrevivencia, entonces obtenemos que los organismos que tienen más descendencia son aquellos que tienen más descendencia. Dicho en palabras de Diane Paul ``[s]i adecuación está definido en términos de éxito de sobrevivencia y éxito reproductivo, entonces el enunciado de que sobrevive el más apto, está aparentemente vacío de contenido'' \footnote{If fitness is defined as success in surviving and reproducing, the statement that the fittest survive is apparently emptied of content.} \citeyear{Paul1992}

Para solucionar este problema Kenneth Waters \citeyear{Waters1986}, nos dice que hay que mostrar como es que podemos definir a la adecuación de manera tal que sea independiente a la medida de descendencia que deja un organismo. Creemos que podemos hacer esto ofreciendo una definición en términos causales de la adecuación, que además apoya a una noción ecológica de adecuación. ¿Por qué es importante desarrollar un concepto de adecuación que no sea tautológico? Porque el concepto de adecuación es central a la teoría de la evolución por selección natural. Si el concepto es una tautología, entonces no se puede comprobar empíricamente. Nosotros defenderemos que el concepto de adecuación sí es empírico y que, por tanto no es tautológico.

Para tratar de solucionar el problema de la tautología, parece intuitivo comenzar por leer causalmente la oración anterior: que un organismo sea más adecuado en su ambiente es lo que causa que tenga más descendencia. Si esto es así, entonces hay que ofrecer una nueva definición de adecuación que nos permita hacer esto.

Una de las soluciones para este problema es tomar a la adecuación como un primitivo de la teoría y de esta manera salvarnos del problema de la tautología. Esta solución haría que adecuación no estuviera definido en la teoría y que desaparezca el problema de la tautología. Sin embargo, queremos que el concepto nos sea de utilidad en explicaciones sobre selección natural. Un primitivo sin definir no ayuda para usar el concepto de forma que expliquemos por qué los más aptos son los que dejan más descendencia. Por tanto, necesitamos otra solución a este problema. Tres opciones famosas restan para atacar este problema. Podemos interpretar al adecuación como una propensión, como una frecuencia relativa, o bien interpretar a la adecuación como un concepto ecológico.

El concepto de adecuación como frecuencia relativa es algo que aparece en el artículo de Walsh y compañía \citeyear{Walsh2002}. Ellos mencionan que hay que medir la adecuación en términos frecuentistas: como la razón entre aquellos que sobreviven sobre el número total de organismos, nos da una manera clara de fijar las probabilidades de que un organismo con cierta característica deje más descendencia. Sin embargo, esta solución tiene algunos defectos. El primero de ellos es que definimos a las frecuencias como un límite al que se tiende cuando el número de organismos es infinito. Pero las poblaciones que se estudian en biología no son infinitas. Por tanto, las frecuencias relativas no son de utilidad en este caso.

Otra opción es hacer del concepto de adecuación una propensión. Esto trata de resolver el problema anterior. En lugar de definir la adecuación como el límite de las frecuencias relativas, nos centramos en las propiedades físicas de los organismos y es a partir de las propiedades físicas que determinamos las frecuencias relativas \cite{Mills1979}. Esta es la propuesta de Mills y Beatty. En particular, Mills y Beatty relativizan la adecuación a una variable ambiental y a las diferencias físicas que hay entre organismos. Por supuesto, no se pueden tomar en cuenta todas las posibles variables que hay en un medio ambiente. Es por ello que Mills y Beatty hablan en términos de un medio ambiente hipotético.

Pensemos, por ejemplo, en una analogía. Un dado con 6 lados tiene las propiedades físicas que nos permiten decir que cada cara tiene $1/6$  de probabilidad de caer en una tirada. Esta probabilidad y dadas las propiedades físicas del dado nos da la propensión que ostenta este aparato en particular. Esta estrategia consiste, entonces, en volver la adecuación una propiedad disposicional en conjunto a una variable medioamebiental. Las propiedades disposicionales son aquellas que sólo se expresan cuando se realiza un proceso. Por ejemplo, el azúcar es soluble en agua, sin embargo, no observamos la solubilidad en el agua a menos que de hecho pongamos el azúcar en agua. En analogía, los organismos tendrán la propensión de dejar $n$ número de descendientes sin que de hecho se reproduzcan. Esta probabilidad estaría definida a partir de las catracterísticas físicas del organismo (o tipo de organismo) en cuestión. La definición de adecuación según los propensionistas es: $x$ es más apto que $y$ en un ambiente $E =_{df}$ $x$ tiene una probabilidad más alta de dejar más descendencia que $y$. Esta solución sugiere que es de la composición física que se desprenden las frecuencias que observamos cuando el límite tiende a infinito.

Esta definición nos libra de la carga tautológica que tiene la definición original de adecuación. Sin embargo, en esta definición queda por aclarar exactamente cómo medir la probabilidad de que $x$ deje mayor descendencia que $y$. Es decir, como podríamos determinar las frecuencias relativas sólo a partir de la composición fisiológica de los organismos. En el caso del dado su composición nos indica cuál será la frecuencia, pero un organismo difícilmente se compara con un dado debido a la gran cantidad de variables que están involucradas en la naturaleza. Una opción es volver a la definición frecuentista y medir la frecuencia relativa en un número muy grande de generaciones. La ley de los números grandes nos dirá cuál será la probabilidad a ``la larga'' de que un organismo deje más descendencia que otro. Esto nos hace regresar al problema anterior. Esta lectura no es apta para ayudarnos a explicar, porque nos importa que podamos medir la adecuación de un organismo en generaciones finitas y no que la propensión dependa de lo que suceda ``a la larga''. Si la propensión depende de lo que pasará a ``la larga'' y esto es potencialmente infinito, entonces no tendremos acceso epistémico a dicha propensión\footnote{Una segunda opción, que no discutiremos en este trabajo, es optar por una interpretación subjetiva de la probabilidad y asignar a cada organismo una medida según le parezca al biólogo evolutivo. Esto no es totalmente arbitrario, podemos argumentar que un investigador con suficiente información puede asignar una medida de probabilidad subjetiva que refleje las probabilidades de un organismo de dejar cierto número de descendientes. Para un trabajo más detallado de estas tesis, véase \cite{Suarez2021}.}.

Otra opción, que es la que nosotros creemos correcta, es colapsar la definición propensionista a una caracterización ecológica de la adecuación. Al hacer esto, podríamos medir la adecuación en términos de cómo ciertas características se comportan en el medio ambiente y resuelven problemas de diseño\cite{sep-fitness}. Creemos que la variable medioambiental es necesaria porque un mismo tipo de organismo puede tener diferentes medidas de adecuación dependiendo del medio ambiente en el que se encuentre. Esta noción de adecuación nos permite resolver el problema de la tautología y tiene la ventaja de que no depende de utilizar frecuencias relativas y que, por tanto, no dependa de que haya poblaciones infinitas.

 Sin embargo, esta noción de adecuación está lejos de estar libre de problemas. Uno de ellos siendo que depende de una metáfora poco clara: ¿exactamente qué queremos decir con ``resolver problemas de diseño''? Para poder salvar la adecuación de la carga tautológica y defender una noción ecológica de adecuación, quisiéramos hacer una propuesta: que podemos definir causalmente la adecuación de un organismo utilizando la maquinaria que nos ofrece Woodward. Esto lo haremos apoyándonos en el argumento que exponen Bouchard y Rosenberg \citeyear{Bouchard2004} donde defienden una noción ecológica de adecuación.

 Sin embargo, antes de hablar de lleno acerca de resolver el problema de la adecuación

\section{Causas próximas y causas últimas}

Hay algunos problemas al hablar de causalidad en biología. Lo que haría que un modelo de explicación causal como el de Woodward parezca inadecuado a la hora de aplicarlo al caso de la biología evolutiva. Por lo que vale la pena hacer un par de distinciones antes de continuar con un modelo de explicación causal. Estos problemas están identificados en el artículo de Ernst Mayr \citeyear{Mayr1998} donde habla de causalidad en biología evolutiva. En primer lugar, Mayr señala que hay dos grandes campos de la biología: la biología funcional y la biología evolutiva.

En segundo lugar, Mayr nos dice que el biólogo funcional trata a su objeto de estudio como un solo individuo y su método es principalmente la experimentación. El biólogo evolutivo por su parte, trabaja con fenómenos extendidos temporalmente y no se preocupa del ambiente próximo de los organismos.

Desde esta distinción, Mayr argumenta que no es lo mismo estudiar la causalidad en ambas áreas: mientras que el biólogo funcional se ocupa de causas próximas, el biólogo evolutivo se ocupa de causas últimas. Con estas distinciones, Mayr afirma que ``[...] las causas próximas son las que gobiernan las respuestas de los individuos (y sus órganos) a factores inmediatos del ambiente, mientras que las causas últimas son responsables de la evolución del programa de información del ADN [...]'' (p. 86).

Mayr presenta a estos dos dominios como complementarios aunque excluye del ámbito del biólogo evolutivo las causas próximas. Dice de la biología funcional que ``La técnica principal del biólogo funcional es el experimento y su aproximación es esencialmente la misma que la del físico o la del químico'' (p.83). Mientras que la biología evolutiva es autónoma de las ciencias físicas o quúimicas.

Pero la tesis de Mayr no está lejos de ser controvertida. Recientemente han habido críticas a la distinción trazada por él. El negar que haya causas próximas en biología evolutiva excluye aspectos de la práctica biológica que son intuitivamente relaciones causales próximas. Por ejemplo, cuando hay un cambio en el medio ambiente y los organismos responden a este cambio de manera que se modifica el fenotipo para poder adaptarse, como pasa en el fenómeno CVG\footnote{Entre otros fenómenos en los que hay causas próximas y son fenómenos evolutivos están la plasticidad fenotípica \cite{WESTEBERHARD20082701} y el Eco-Evo-Devo \cite{PfenningEco-Evo-Devo}} \cite{CVG}. Además, deja de lado toda la nueva literatura que ha explorado la epigenética, por ejemplo en el fenómeno de la construcción de nicho. En este fenómeno los organismos modifican su medio ambiente. Este medio ambiente se hereda a la progenie, ya que es el lugar en el que habitarán. Este medio heredado es una causa próxima. Si eliminamos las causas próximas del ámbito evolutivo, estaríamos excluyendo estos fenómenos. Esto está relacionado con admitir nuevos mecanismos de herencia que no son heredados por la vía sexual.

Endler en su libro \citeyear{Endler1986} ofrece una caracterización más amplia de herencia. Endler nos dice que no necesariamente el fenómeno de la herencia depende de la herencia genética. El autor menciona que debe haber una relación consistente entre que algún organismo tenga una cierta característica que le permita una mejor habilidad de apareamiento, o bien una mejor habilidad de fertilización, o bien una mejor fecundidad y/o sobrevivencia (siempre con miras a que estas características no son necesariamente genéticas).

Ahora bien, la síntesis moderna fue bastante reticente en aceptar cualquier otro mecanismo de herencia que no fuera la herencia genética. Por lo que se excluyeron otros tipos de explicación y mecanismos no genéticos, o bien que no tuvieran una reducción a la genética. Ya desde hace varios años se ha explorado si es necesario incorporar nuevos mecanismos de herencia \cite{Jablonka2020}. De ser esto correcto, estos nuevos mecanismos de herencia y su incorporación a la teoría tienen consecuencias importantes en qué evidencia debemos tomar para justificar hipótesis en selección natural y por supuesto ponen en tela de juicio la distinción que Mayr propone.

En este trabajo no vamos a discutir las consecuencias filosóficas y biológicas que tiene el integrar nuevos mecanismos de herencia. Lo que vamos a hacer es ofrecer ejemplos que apoyen la hipótesis de que los mecanismos de herencia no son todos genéticos al ofrecer un par de ejemplos. Es por ello que afirmamos que si hay herencia no genética, entonces podemos incorporar causas próximas en las explicaciones evolutivas. Más aún, si podemos incorporar causas próximas para explicaciones en selección natural, entonces no es controvertido decir que necesitamos un concepto de ``causalidad'' que permita incorporar dichas causas próximas. Motivaremos que el modelo de Woodward como modelo causal para explicaciones evolutivas al ofrecer un concepto de \emph{fitness} en consonancia con el modelo que propone Woodward.

Una advertencia antes de continuar: el que debamos integrar una dimensión ambiental es aún un debate abierto. Por lo que en un futuro la investigación puede encontrar una manera de reducir los factores ambientales a factores genéticos, lo que haría que debamos excluir la dimensión ambiental de las explicaciones evolutivas. Esta estrategia es la que siguen algunos de los biólogos que aún defienden algún tipo de neo-darwinismo. Esto sin duda es posible, pero la investigación actual en biología apunta en la otra dirección  \cite{Bateson2014}. Hay investigaciones que de hecho incorporan una dimensión ambiental y que afirman que hay otros medios de herencia no necesariamente genéticos. Esto es indicio de que podemos hablar de causas próximas en biología evolutiva. Por lo que en lo que resta de este trabajo supondremos que este es el caso.

A pesar de la reticencia de algunos biólogos, se han investigado sistemas de herencia distintos que no son herencia genética. Por ejemplo en el libro de Jablonka y Lamb \citeyear{Jablonka2020}, las autoras argumentan que hay diferentes mecanismos de herencia que no necesariamente están al nivel genético. Entre ellos se encuentran los mecanismos epigenéticos y la imitación del comportamiento. La construcción de nicho también es un caso en el que puede haber herencia no genética. Aún con la evidencia experimental, las autoras comentan que se ha relegado este tipo de mecanismos con el argumento de que son triviales porque no hacen una diferencia en términos evolutivos. Sin embargo, las autoras argumentan que hay evidencia de lo contrario.

En el artículo de Uller y compañia \citeyear{Uller2019} encontramos un ejemplo concreto que pretende mostrar que las causas próximas no están excluídas del ámbito del biólogo evolutivo. Estudiando a las ballenas asesinas, los investigadores dan cuenta de que las ballenas han adaptado su dieta localmente y han desarrollado técnicas de caza particulares. Los estudios muestran que estas diferencias no se deben a variación genética, sino al aprendizaje social. El aprendizaje social es un caso de causa próxima tal como lo describe Mayr. Entonces las causas próximas sí son relevantes para la biología evolutiva ya que estas diferencias en técnicas de caza son características que pueden ser seleccionadas.

Una estrategia para mantener la distinción entre causas próximas y causas últimas es cambiar el foco de atención, alguien podría sugerir que no es interesante el hecho de por qué diferentes poblaciones de ballenas han desarrollado distintos métodos de caza por aprendizaje social, sino que la pregunta interesante es por qué las ballenas asesinas desarrollaron la característica de ser capaces de aprender socialmente.

Sin embargo, esta solución tiene problemas. En primer lugar esta solución cambia el \emph{explanandum} de la pregunta original. El \emph{explanandum} de la pregunta original es por qué los diferentes métodos de caza de las ballenas asesinas han llevado a variación entre las distintas poblaciones. Nos interesa saber por qué las diferentes poblaciones tienen estos métodos particulares, no por qué las ballenas tienen la característica de aprendizaje social, que es una pregunta diferente.

Un ejemplo que podría ilustrar aún más esto que mencionamos es el experimento de las mariposas \textit{Bycyclus anynana}. En \cite{Frankino2007}, los investigadores prueban la hipótesis de que el tamaño de las alas de las mariposas se debe en mayor medida a selección natural y no a restricciones de desarrollo del organismo. Aunque las restricciones de desarrollo guían el tipo de alometrías posibles, la selección natural actúa para favorecer un tipo sobre otro según los resultados del artículo.

Para ver qué tanto afecta selección natural en la alometría de \emph{Bicyclus anynana}, seleccionaron artificialmente a los individuos para guiar el desarrollo de las alometrías que tuvieran las alas posteriores más grandes y las alas anteriores más chicas; además también seleccionaron artificialmente a aquellos individuos con las alas posteriores más chicas y las alas anteriores más grandes. Llegado un punto, se dieron cuenta de que estas mariposas podían tener una variación de tamaños exagerada, diferente a los que vemos en la naturaleza. Por lo que sugierenm que el medioambiente se encarga de hacer que las maripoisas tengan las alometrías que vemos en la naturaleza.

Debido a que estas alometrías no son imposibles y que no están completamente determinadas por restricciones de desarrollo, la selección natural es el proceso principal que determina las alometrías que observamos en el entorno natural. Esta explicación es en términos de causas próximas ya que apelan al medioambiente inmediato.

Estos ejemplos no sólo pretenden ilustrar que la teoría de la causalidad de Woodward es compatible con el quehacer del biólogo evolutivo, sino además ofrecer evidencia para lo que argumentan Jablonka y Lamb, a saber, que hay una importante injerencia del medio en el que se desarrollan los organismos y que hay herencia no necesariamente genética.  Esto nos lleva, contra la suposición de Mayr, a poder incorporar causas próximas en explicaciones de selección natural.

Nuestra suposición, a saber, que podemos incorporar causas próximas en explicaciones de fenómenos evolutivos, nos invita a ofrecer una caracterización de causa que sea compatible con el quehacer de los biólogos evolutivos. Lo que falta ahora es motivar cómo este modelo nos permite ofrecer una definición causal de adecuación. Antes de continuar, nos proponemos a exponer en qué consiste el modelo de Woodward.

\section{la noción ecológica de adecuación}

La noción ecológica de adecuación nos dice que un organismo es más adecuado a su medio ambiente cuando resuelve mejor problemas de diseño que otros individuos de la población \cite{sep-fitness}.

\section{el modelo de explicación causal de Woodward}

En la postura de Woodward, son importantes las nociones de ``invarianza'' e ``intervención''. Estas son importantes porque a partir de estas nociones se define una generalización que, a pesar de no ser una ley de la naturaleza, es explicativa. Si el modelo de Woodward logra hacer lo que se propone, entonces tendremos un modelo de explicación que no apele a leyes y, por tanto, más adecuado para ciencias como la biología. Además, retomando causalidad, evitamos la simetría de la que pecaba el modelo de Hempel y resolvemos los problemas de correlación del modelo de Salmon. Además, es claro que muchas de nuestras explicaciones exitosas (si bien no todas las explicaciones exitosas) son causales.

\textit{Grosso modo} Woodward nos dice que explicar tiene que ver con hacer explícitas las relaciones causales entre dos variables: sean ``$A$'' y ``$B$'' dos eventos cualquiera, decimos que el evento $A$ explica al evento $B$ cuando hay una relación causal que liga la ocurrencia de $A$ con la ocurrencia de $B$. Esto es sólo cuando al manipular $A$, el valor de $B$ cambia en consecuencia. El hecho de que haya una intervención no implica que tenga que haber un agente. Intervención está definido de manera que fenómenos naturales en los que sucede un evento $A$ y esto a su vez modifica el valor de otro evento $B$ cuenta como una intervención en el modelo de Woodward.

Para asegurar que esta relación es causal, este cambio en $B$ debe estar relacionado sólamente con los cambios en $A$ y no deberíamos poder explicar el cambio en $B$ por intervención directa. De manera más esquemática la noción de explicación de Woodward es definir una relación $R$ tal que $R<A, B>$ esté constreñida por las siguientes características: i) cambios en el valor de $B$ deben estar directamente relacionado con cambios en el valor de $A$ de manera que sin cambios en $A$, no habría habido cambios en $B$\footnote{Este criterio es evidentemente modal}; ii) mediante $R$ debemos ser capaces de hacer una ``generalización'' tal que dicha generalización nos describe el comportamiento del sistema en los casos donde la relación es invariante (que son casos en los que bajo ciertas restricciones si ocurre $A$, entonces ocurre $B$)\footnote{Supongamos por ejemplo que quiero saber bajo qué condiciones un vaso que se cae, se rompe. Podríamos variar la altura de la caída, así como el material sobre el que cae. Si, por ejemplo, tiráramos un vaso en un colchón, a 10 cm. de altura, no se romperá; si lo tiráramos de una altura de dos metros en un piso de concreto, seguro se romperá. Esto quiere decir que la relación entre tirar un vaso y que se rompa en consecuencia es invariante bajo algunos valores de altura y del material sobre el que cae.}.  iii) $A$ hace un cambio en $B$ y el cambio en $B$ no debe darse por ninguna otra ruta; iv) No hay causas diferentes a $A$ que cambien a $B$ (ya sea una causa común o alguna otra razón), por último todo debe estar acotado a i-iv \cite[p. 201]{Woodward2000}. Todo esto constituye la noción de intervención. Si diseñamos una manera en la que podamos intervenir en $A$, que cambié el valor de $B$ en consecuencia y dicha intervención cumple las características i-iv, entonces tenemos una explicación \textbf{causal} de la ocurrencia de $B$.

Por ejemplo, supongamos que una nueva píldora minimiza los dolores de cabeza. Esta píldora actúa disminuyendo la sensibilidad de dolor en todo el cuerpo, entonces si la tomara, disminuirá mi dolor. El cambio en el dolor de cabeza debe estar directamente relacionado con la toma de la píldora y no con que, por ejemplo, haya tenido un accidente que cercenó mi cabeza (algo que seguramente habría eliminado mi dolor). También tiene que ver con el posible evento en el que si no me hubiera tomado la píldora, entonces no hubiera disminuido mi dolor de cabeza.

Un ejemplo más elaborado de esto y que pone de manifiesto que no es necesario un agente que intervenga es el caso de cómo la luz y la temperatura afecta el proceso de floración en plantas del género \emph{Arabidopsis}. En \cite{AusinEnviro}, los autores argumentan a favor de la hipótesis que afirma que el proceso de florecimiento es un proceso altamente plástico. Los datos arrojan que en el caso de \emph{Arabidopsis}, hay al menos dos factores que modifican la velocidad con la que este género florece: temperatura y luz. En el caso de la temperatura se ha observado que si se somete a los especímenes a temperaturas bajas (aunque no al punto de congelamiento), el proceso de florecimiento se acelera. En el caso de la luz, los especímenes reaccionan a la luz roja, a la luz roja lejana (longitudes de onda entre 700 y 750 nm.) y a la luz azul. Cuando hay bajos niveles de estas tres, se promueve el florecimiento.

Lo que este ejemplo pretende mostrar es que no necesariamente debe haber un agente interviniendo directamente en los factores relevantes. Sin duda, se pueden recrear diferentes condiciones en el laboratorio. Por ejemplo, supongamos que un observador cayó en cuenta de que sus plantas florecían más rápido cuando estaban bajo una sombra que al sol directo. Esto podría ayudar a diseñar condiciones en las que el observador replique lo que accidentalmente observó. Puede entonces crear condiciones en las que mantenga la temperatura igual y que modifique la luz que llega a la planta. O bien puede mantener la luz fija y variar la temperatura. Si todo esto además cumple con las características que pide Woodward, podemos concluir que hay una relación causal.

Cabe aclarar que dichas intervenciones deben ser posibles. Por ejemplo puedo preguntarme qué pasaría en el caso en el que no me tomara la píldora. Si realmente es la píldora lo que hace que cese mi dolor de cabeza, entonces me seguirá doliendo en el caso en el que no me la tome. Puedo preguntarme también qué pasaría en caso de que el componente de la píldora fuese diferente al que de hecho es, etc. Puedo también preguntarme por qué un cuervo es negro, y puedo preguntarme qué debería cambiar para que el cuervo tuviera un color diferente. Pero sería absurdo preguntarme qué pasaría si en lugar de ser ``este'' cuervo fuera un cardenal. Puedo hacer que en el cuarto haya un cardenal y no un cuervo, pero no puedo hacer que ``este'' cuervo se convierta en un cardenal. Son casos como los anteriores los que acotan las intervenciones posibles.

Con respecto a la noción de invarianza, Woodward nos dice que cualquier generalización que describa una relación entre dos o más variables es invariante si se sostiene aún cuando se modifican otras condiciones. Esta noción de invarianza es lo que permite hacer generalizaciones de la relación entre dos variables. Porque si hay una relación causal entre $A$ y $B$ y dicha relación se sostiene aún cuando otras variables se modifican, entonces podemos decir que para cualquier $A$ y $B$ habrá la misma relación causal. Cuando esto se cumple, tenemos un indicio de que es posible manipular y controlar la variable independiente para ver qué cambios hay en la variable dependiente \cite{Woodward2000}.

Sin duda el modelo de Woodward tiene muchas virtudes. Primero tiene una aplicación clara para las ciencias especiales ya que no parte de la noción de ley, sino que construye generalizaciones como ``invarianza bajo intervenciones''. Woodward resuelve los problemas que tenía el modelo de Salmon al poner más restricciones en lo que deberíamos hacer cuando buscamos relaciones de dependencia causal. Otra virtud es que la noción de intervención encaja con el hecho de que en las investigaciones se llevan a cabo experimentos y que es a partir de ello que obtenemos información que indica si hay o no una relación entre variables.

Pongamos un ejemplo: la luz y la temperatura, por separado,  afectan el proceso de floración en plantas del género \emph{Arabidopsis}. En el artículo \cite{AusinEnviro}, los autores desarrollan un modelo experimental en el que observaron que hay una relación entre la temperatura y la luz afectan cómo florecen las plantas del género \emph{Arabidopsis}. Para el experimento se sometió a los especímenes a temperaturas bajas (aunque no al punto de congelamiento) y observaron cómo el proceso de floración se acelera en consecuencia. Esto según la metodología de Woodward nos permite concluir que hay una relación causal entre la temperatura y la floración. En el caso de la luz, se observó que los especímenes reaccionan a la luz roja, a la luz roja lejana (longitudes de onda entre 700 y 750 nm.) y a la luz azul. Cuando hay bajos niveles de estas tres, se promueve la floración. Esto indica que hay una relación causal entre los bajos niveles de este tipo de luces y la floración. Para esto, se interviene en las condiciones de temperatura y de luz para poder concluir que hay una relación causal.

\begin{center}
  \begin{tikzpicture}[scale=2]
    \node (X) at (1,1){Floración};
    \node (A) at (0,0){Temperatura};
    \node (B) at (2,0){Luz};
    \path[-angle 90,font=\scriptsize]
    (A) edge    (X)
    (B) edge    (X);
    \end{tikzpicture}
\end{center}

Este diseño experimental en el que somos capaces de intervenir variables es compatible con la metodología de Woodward. Algo parecido sucede en el caso de \emph{A. sagrei} y \emph{L. Carinatus}, se interviene la variable depredador y se observa que hay un cambio en el tamaño de las extremidades de \emph{A. sagrei}. Que haya un nuevo depredador en el medio de \emph{A. sagrei} causa que se seleccionen los organismos que tienden a tener extremidades más largas. Por lo que podemos fijarnos en como el fenotipo interactúa con el ambiente. Esto de nuevo lleva a pensar que la distinción hecha por Mayr deja de ser útil para explicar fenómenos evolutivos\footnote{Los límites de la distinción de Mayr han sido explorados recientemente en \cite{Uller2020, Dayan2020, Laland2011}.}, al mismo tiempo que es compatible con la metodología de Woodward.

Pensemos en un ejemplo más: El experimento realizado por Amarillo Suárez y Fox \citeyear{Amarillo-Suarez2006}. Hay insectos que se desarrollan dentro de un hospedero. Se tiene evidencia que el hospedero en el que se desarrollan las crías tiene influencia en el tamaño de los insectos. En el artículo de Amarillo-Suárez y Fox, se explora cómo el hospedero del \emph{Stator limbatus} que puede hospedarse en dos tipos de árbol: \emph{Acacia greggi y Pseudosamanea guachapele}, tiene consecuencias en su desarrollo. Las particularidades de estos árboles es que el \emph{Acacia greggi} tiene unas semillas más grandes que el \emph{Pseudosamanea Guachapele}. Se analizó cómo varía el tamaño de los insectos cuando el hospedero es un árbol u otro. El resultado experimental mostró que cuando este insecto se hospeda en el árbol con las semillas más grandes, los organismos son de mayor tamaño. Este mayor tamaño es independiente al tamaño de los progenitores. Según las autoras del artículo esto indica plasticidad fenotípica. El diseño experimental se encarga de mantener a la población fija, es decir, sortea de manera aleatoria en qué hospedero pondrán a qué organismos. Al hacer esto, controlamos por el tamaño de los organismos. Esto quiere decir que hay al menos dos maneras de intervenir en el tamaño de los organismos. La más obvia es el tamaño de los padres, la segunda es las semillas que tiene el hospedero. En el diseño se controla por el tamaño para que no afecte como variable. Al eliminar como factor el tamaño de los padres, podemos intervenir directamente en qué hospedero poner a los organismos.

\begin{center}
  \begin{tikzpicture}[scale=2]
    \node (X) at (1,1){Tamaño de los organismos};
    \node (A) at (0,0){Tamaño de los padres};
    \node (B) at (2,0){Semillas};
    \path[-angle 90,font=\scriptsize]
    (A) edge   (X)
    (B) edge   (X);
    \end{tikzpicture}
\end{center}


Para poder controlar por el tamaño de los padres, se hace una selección aleatoria de una población y se asigna en los dos diferentes hospederos. Lo que nos permite intervenir directamente en el tamaño de las semillas al alocar a los organismos en los dos respectivos árboles.

\begin{center}
  \begin{tikzpicture}[scale=2]
    \node (X) at (1,1){Tamaño de los organismos};
    \node (A) at (0,0){Tamaño de los padres};
    \node (B) at (2,0){Semillas};
    \path[-angle 90,font=\scriptsize]
    (A) edge    (X)
    (B) edge [dotted]   (X);
    \end{tikzpicture}
\end{center}

Si después se observa que a pesar de que las alturas de una generación eran, en promedio, iguales en los dos árboles y que las nuevas generaciones de organismos son más grandes en el árbol con las semillas más grandes, entonces podemos afirmar que hay una relación causal en términos de la metodología de Woodward.

Este ejemplo evidencia que se ha estado trabajando en la tesis que afirma que el medio ambiente es un factor que hace la diferencia en los fenotipos. No sólo hay evidencia a favor de esto, sino que además hay evidencia que estas relaciones entre medio ambiente e individuos son un factor relevante para la evolución por selección natural \cite{Jablonka2020, Dayan2020, MacColl2011}. Esto no es lo único importante, si queremos ofrecer explicaciones causales de otros fenómenos biológicos como Eco-Evo-Devo \cite{PfenningEco-Evo-Devo}, Plasticidad fenotípica \cite{WESTEBERHARD20082701}, CGV \cite{CVG}, entonces hay que hablar de causas próximas en biología evolutiva.

Estos problemas con la distinción que hizo Mayr dan entrada a que podamos hablar de relaciones causales en las explicaciones evolutivas. Por lo que nos atrevemos a afirmar que las causas próximas sí están presentes en las explicaciones evolutivas por selección natural.

Esto además casa bien con la metodología ofrecida por Woodward. En términos de lo que afirma esta teoría, hay una relación causal entre el medio ambiente y los organismos que lo habitan. Más aún, hay intentos de exponer que un enfoque manipulabilista puede ser de utilidad en las explicaciones por selección natural \cite{MacColl2011}. Dado que podemos hablar de causas próximas, nos dedicamos en la siguiente sección a hablar en particular de cómo lo dicho hasta ahora está relacionado con el debate acerca de la adecuación.

\section{\emph{fitness}, un concepto causal}

\noindent Ahora que ya tenemos una teoría de la explicación causal que es compatible con el quehacer del biólogo evolutivo y que hemos afirmado que podemos incorporar causas próximas en fenómenos evoluivos, cabe defender la noción ecológica de adecuación. Como dijimos anteriormente la adecuación es un concepto central de la teoría de la evolución por selección natural. En la esquematización que dimos anteriormente en \ref{causal}, la tercera línea captura el hecho de que los organismos más aptos son aquellos que dejan más descendencia. En particular, decimos que el \emph{fitness} es medido en términos de descendencia y esto quiere decir que la selección natural opera en este tipo\footnote{En este caso hablamos de \textbf{tipos} de organismos, esto es, que compartan cierta característica física.} de organismo cuando, en ausencia de otros factores (por ejemplo, deriva génica) un organismo deja más descendencia que otro.

El concepto de \emph{fitness} como propensión tiene el problema de cómo tenemos acceso epistémico a dichas propensiones. Mencionan Mills y Beatty \citeyear{Mills1979} que esto se puede hacer de acuerdo a las características físicas de un organismo, tomando como variable el medio ambiente. Esto sugiere que un organismo que tenga un nivel de \emph{fitness} $x$ tendrá un nivel diferente en otro medio ambiente. Pero es en cómo determinamos este nivel de \emph{fitness} que está el problema.

En consonancia con la noción propensionista, nosotros queremos decir que el nivel de \emph{fitness} se puede determinar sólo en los resultados de un diseño experimental. Esto es crucial para poder incorporar la noción de intervención que requiere el modelo de Woodard. Es decir en el medioambiente del laboratorio. Esto nos daría acceso a las propensiones de los organismos, resolviendo el problema de la definición propensionista. La parte causal en esta noción de \emph{fitness} estaría definida de la mano del intervencionismo de Woodward al interactuar con las variables dentro del laboratorio. Si todo esto es correcto, entonces tenemos una noción empírica de \emph{fitness}, al mismo tiempo que nos deshacemos del problema de la tautología. Nos deshacemos de este problema porque los más aptos no necesariamente son los que dejan más descendencia, sino aquellos que mejor resuelven problemas de diseño impuestos por el medio ambiente.

Habiendo mencionado que podemos incorporar causas próximas en nuestras explicaciones por selección natural, queda decir cómo el modelo de Woodward está relacionado con el debate sobre la interpretación de \emph{fitness}. Mostramos en la sección anterior cómo el modelo nos ayuda para extraer información causal a través de los diseños experimentales y la manera en que podemos intervenir en las variables y observar cómo se modifican otras variables. Al poder hacer esto, podemos definir al \emph{fitness} a través de estos modelos experimentales. Es en estos modelos en dónde podemos observar cómo se desarrollan los organismos al interactuar con el medio ambiente. Al intervenir en las diferentes variables, podemos determinar cuál es la causa de que un tipo de organismos resulte ventajoso. Todo esto en un ensamble de laboratorio. En este sentido el \emph{fitness} de un tipo de organismo se sigue midiendo en tanto número de descendientes. La ventaja es que en el diseño experimental podemos ver la tendencia del organismo, determinando así la característica ventajosa. Al ser una intervención cuenta como causal en términos de la maquinaria que nos ofrece Woodward.

Esto nos permite decir exactamente cuál es el organismo más apto dependiendo del medio ambiente en el que se desarrolla, lo que tiene la ventaja de que tenemos una medida de \emph{fitness} que explica por qué unos organismos tienen ciertas características y si estas características son mejores para resolver problemas de ``diseño'' que otras características en la población. Esto nos deja con una definición de fitness no tautológica y explicativa. El valor explicativo se desprende de estar inserto en el modelo de Woodward. El modelo también nos permite que la definición sea causal. Como defendimos en el capítulo 2 de este trabajo, no es necesario que causalidad implique determinismo. Por lo que parece que esta definición de \emph{fitness} es mejor que la propensionista, además de ser más apta que la sola versión ecológica.

Aceptar esta definición de \emph{fitness} tiene las ventajas de dar un concepto más claro. Este concepto está relativizado a variables medioambientales. Al mismo tiempo nos permite leer causalmente al \emph{fitness} al decir exactamente cuál es la característica ventajosa que permitió resolver los problemas de diseño impuestos por el medioambiente. Esta definición además nos permite explicar cuál es el factor relevante para la sobrevivencia de los organismos.

Tiene además la consecuencia de que la única manera en que podemos medir qué organismos son más aptos es cuando ya hay una cierta adaptación en el diseño experimental. Pensemos, por ejemplo, en el caso de \emph{A. sagrei} y \emph{L. carinatus}. En este diseño experimental se observa una tendencia al crecimiento de extremidades que les permitan escalar. Los organismos que mejor resuelven este problema y que, por ello, dejan más descendencia son aquellos con extremidades más largas. Aquí tenemos una explicación apelando a la selección natural. Pero es sólo en este diseño experimental y una vez observada esta tendencia que podemos decir que los más aptos, en este medioambiente particular, son los que desarrollan extremidades más largas. Es sólo cuando tenemos esta información que podemos utilizarla para explicar.





































\noindent La síntesis moderna en biología evolutiva fue un cambio enorme para la teoría darwinista. En pocas palabras, lo que se hizo durante este periodo fue acercar la genética mendeleiana y la teoría de la selección natural de Darwin. En su investigación, Darwin había descubierto que hay variación entre organismos de una misma población. Además, se podía observar que algunas de estas variaciones se mantenían entre los padres y su progenie. Más aún, algunas de estas variaciones son ventajosas para los organismos. Estas variaciones ventajosas hacen que ciertos individuos tiendan a tener más descendencia como resultado.

De manera esquemática la teoría de la selección natural sostiene que\footnote{1. There is variation in morphological, physiological, and behavioural traits among members of a species (the principle of variation), 2. The variation is in part heritable, so that individuals resemble their relations more than they resemble other individuals and, in particular, offsprings resemble their parents (the principle of heredity), 3. Different variants leave different numbers of offspring either in inmediate or remote generations (the principle of differential \emph{fitness})}:

\begin{enumerate}
  \item Hay variación en las características morfológicas, fisiológicas y de comportamiento entre los miembros de una especie.
  \item La variación es en parte heredable, de manera que los individuos se parecen más a sus padres que a otros miembros de la especie.
  \item Diferencias en las variaciones tienen como consecuencia diferencias en el número de descendientes, bien en generaciones inmediatas, o bien en generaciones más distantes\cite{Godfrey-Smith2013}.
\end{enumerate}


El esquema anterior deja ver que medimos el \emph{fitness} de acuerdo al número de descendencia que deja un organismo. Nosotros no queremos negar que esta es la manera en la que medimos si un organismo es más apto que otro, lo que sí buscamos es ofrecer una caracterización que nos permita decir cuáles son los organismos que, dada la información que tenemos, son los que dejarán más descendencia.


%Si bien en algunos casos se presenta la adecuación ecológica y al \emph{fitness} como propensión como contradictorias \cite{sep-fitness}, creemos que podemos tomar una parte de ambas nociones a través de utilizar a la teoría de Woodward como modelo causal.
