\section{Análisis filosóficos de la explicación}

\noindent De manera cotidiana explicamos por qué suceden fenómenos. Nos interesa saber por qué el agua está fría mientras nos bañamos, por qué la puerta rechina cuando la abrimos, por qué nuestro automóvil hace ruidos extraños, etc.

En algunas ocasiones, nos interesa explicar fenómenos más complejos\footnote{Con esto no queremos implicar que las explicaciones de fenómenos más complejos son a su vez más valiosas que las de fenómenos más simples. Tampoco queremos implicar que haya una distinción de tipo entre ambos conjuntos de ejemplos, el problema acerca de si tal diferencia es o bien de grados o bien de tipo no es algo que vayamos a discutir en este trabajo.}: ¿por qué la capa de ozono tiene orificios?, ¿por qué los seres humanos somos bípedos?, ¿por qué se cumple que la suma de los cuadrados de los lados de un triángulo rectángulo, son iguales al cuadrado de la hipotenusa?, etc.

Intentos de hacer claro el concepto de explicación en ciencia se remontan al menos hasta los empiristas lógicos\footnote{Esto en la filosofía moderna. Aristóteles también dedico parte de su trabajo en los analíticos posteriores a esclarecer el concepto de explicación \cite{Aristoteles2009}}. Hempel y Oppenheim desarrollaron --quizás el más famoso-- modelo de explicación: ND. ND hace de una explicación un tipo de argumento deductivo de la siguiente manera: en las premisas debe haber una ley de la naturaleza. Esto nos lleva al trabajo de poder disitnguir a las generalizaciones accidentales, por ejemplo, el hecho de que todas las monedas en mi bolsillo sean de \$ 5 y una generalización legaliforme como que los cuerpos en caída libre tienen un aumento de velocidad de 9.8 m por segundo cuadrado.

Nagel nos ofreció algunos criterios que debía cumplir una ley de la naturaleza, que presumiblemente no cumplen las generalizaciones accidentales: i) debe ser un universal irrestricto, es decir, que funcionara en todo momento para una cantidad potencialmente infinita de objetos; ii) en su formulación sólo debe haber vocabulario puramente cualitativo, esto con el fin de evitar que se hiciera referencia a objetos en un espacio y tiempo determinados y iii) debe haber algún tipo de relación tal que el antecedente haga necesario el consecuente. Toda esta discusión se puede revisar en el capítulo IV de \cite{Nagel2006}. Indicamos algunas condiciones iniciales observadas y derivamos el fenómeno a explicar como una consecuencia tanto de la ley y de las condiciones iniciales. Un criterio adicional de ND es que las leyes sean verdaderas. Deben ser verdaderas porque en el caso contrario el antecedente es falso y el condicional asociado es trivialmente verdadero. De manera que si tenemos un antecedente falso, no es claro cómo explica el fenómeno que nos interesa.

%Además de ND Hempel y Oppenheim desarrollaron un modelo Inductivo Estadístico y un modelo Deductivo Estadístico. Las particularidades de ambos es que en el deductivo estadístico partimos de una ley estadística y derivamos que el hecho va a ocurrir. Lo que tiene que suceder para que esto cuente como una explicación es que la ley estadística tiene que hacer más probable el evento. Una motivación para el desarrollo del modelo inductivo estadístico es que Hempel pretendía explicar eventos particulares utilizando probabilidades. Por ejemplo, en el caso de que tome un analgésico para calmar mi dolor de cabeza, la probabilidad de que disminuya dependerá de la frecuencia de las personas que han tomado la pastilla y han mejorado. En el deductivo estadístico se usa una ley estadística para deducir una uniformidad estadística más estrecha. Mientras que el modelo inductivo estadístico tomamos muestras individuales para hacer una generalización estadística para una población y una propiedad determinada.

Supongamos, para hacer un ejemplo, que hay una ``ley'' que nos dice que: para todos los analgésicos y todos los dolores de cabeza, si cualquier sujeto tomara un analgésico, entonces calmará su dolor $\forall{x}\forall{y}\forall{z}((Ax\&Dzy\&Tzx)\supset Czy)$. Supongamos que de hecho yo tengo dolor de cabeza y me tomo un analgésico $(Aa \& Doc \& Toa)$. Por tanto se resuelve mi problema de dolor $Coc$. En este ejemplo, el hecho de que mi dolor de cabeza desaparezca, depende de la ley establecida y de las condiciones iniciales, que en este caso es el hecho de que yo tengo un dolor de cabeza particular y que me tomo un analgésico. El fenómeno de que cese mi dolor de cabeza se explica a través de la ley, ya que resulta un caso particular. Es decir que esta ``cubierto'' por la ley.

Cabe resaltar que ND tiene algunos valores y que rescata algunas intuiciones que vale la pena señalar: que los científicos explican utilizando leyes, que hace de la explicación un argumento y es claro que en las discusiones científicas hay intercambio de argumentos.

Pero aquí ya hay varias preguntas que hacer al respecto de este modelo. Por ejemplo, algunos de los problemas que tuvo este modelo tenían que ver con la noción de ley de la naturaleza. No es obvio que todas las ciencias trabajen con leyes. La biología es un caso bastante comentado porque ofrece explicaciones, pero no es claro que lo haga utilizando leyes \cite{Brandon1997}.

Esta aserción podría ser debatida. Alguien podría decir que las condiciones de Nagel no son indicadoras de una ley, y que deberíamos relajar los compromisos\footnote{Este debatiente ficticio tendría qué decirnos cuál criterio debemos relajar o eliminar}. Supongamos, por ejemplo, que rechazamos el criterio de que debe ser un universal irrestricto. Relajando este criterio, podríamos decir que la ``ley de segregación''\footnote{Esta ley, a grandes rasgos, nos dice que en un organismo diploide, cada uno de los progenitores tiene la mitad de su código genético almacenado en sus gametos. Por lo que hay un 50\% de probabilidad de encontrar cualquiera de los genes de los progenitores en el código genético del descendiente.} de Mendel es un buen candidato para una ley en biología. Esto se cumple para los organismos diploides, pero no para todos los organismos.

Sin embargo, aún así no sería suficiente para salvar a ND de todos los problemas con que carga. Cartwright defiende que ni siquiera es obvio que las leyes de la física sean verdaderas ni es claro que cumplan con las condiciones que presenta Nagel \cite{Cartwright1983}: entre ello que sean un universal irrestricto. Según Cartwright sólo son verdaderas cuando imponemos las condiciones necesarias para que lo sean. Pero sin lugar a dudas los físicos ofrecen explicaciones. Esto indica que debemos relajar aún más los criterios.

Aún más que el problema de las leyes, ND tiene otros problemas. Las explicaciones tienen una dirección porque no sirve de nada una explicación circular, de manera que la conclusión pueda explicar a las premisas y viceversa. Un ejemplo famoso que muestra este defecto es el del asta bandera\footnote{El ejemplo es el siguiente, supongamos que un asta bandera proyecta una sombra de cierta longitud $x$. Si conocemos el largo del asta bandera y sabemos el ángulo del sol, podemos calcular la longitud de la sombra. Por lo que la altura del asta explica la longitud de la sombra. Sin embargo, si sabemos la longitud de la sombra y el ángulo del sol, podemos obtener la altura del asta. Pero es absurdo pensar que el largo de la sombra explica el alto del asta. El alto del asta depende de las intenciones de quién la construyó.}. Entonces ND no sólo tiene el problema de que se necesitan leyes para la explicación, sino que además no respeta la asimetría de la explicación.

En el caso particular de la biología, si aceptamos que esta disciplina explica, entonces o bien hay que encontrar enunciados en la biología que tengan el estatus de ley, o bien desechamos el modelo de Hempel y desarrollamos nuevos modelos que permitan esclarecer cómo explicamos en biología. Muchos filósofos han desarrollado modelos alternativos y nuestro interés es explorar cuál es más adecuado para la biología.

%Los otros modelos de Hempel presentados también presentan algunos problemas: para el inductivo estadístico es necesario tener eventos con alta probabilidad de tal manera que hagan más probable a la conclusión. Dado este requisito este modelo no puede incluir casos en los que esperamos que suceda con menor probabilidad un evento. El modelo deductivo estadístico es una instancia de ND así que hereda los problemas de ND.  Además no siempre tenemos a nuestra disposición eventos con un alto índice de probabilidad y necesitamos dar cuenta de casos donde la probabilidad de hecho es baja.

\section{Relevancia Estadística}

%Kitcher expone de manera clara la discusión en \citeyear{Kitcher2002}, enmarcando a los diferentes modelos de explicación bajo el modelo pragmático de van Fraassen. El modelo de explicación pragmático de van Fraassen consiste en armar una tripleta ordenada cuyos elementos son una clase de contraste, el tema de la pregunta, y la relación de relevancia \cite{Fraassen1977}. Kitcher nos dice que el modelo de van Fraassen es útil para exponer claramente cuál debería ser el objetivo de cualquier modelo de explicación. Este objetivo consiste en decir cuál es la condición de relevancia adecuada\footnote{van Fraassen acepta que su modelo es compatible con los modelos de Salmon y Hempel al momento de evaluar si una respuesta es correcta. Esto sucede porque la relación de relevancia se puede definir de varias maneras\cite{Fraassen}.}.

\noindent El modelo de Salmon estaba motivado por resolver algunos problemas de ND. Salmon desarrolla su modelo de explicación basado en la noción de relevancia estadística, además de señalar explícitamente que en su modelo las explicaciones no son argumentos. Salmon expone su modelo como una solución a los problemas que tiene ND. En el modelo de Salmon, no necesita haber un aumento en la probabilidad de ocurrencia de un fenómeno para explicar por qué ocurre. Además, de acuerdo a cómo formula su modelo, la explicación deductiva es un caso límite (cuando la probabilidad de ocurrencia es igual a 1) de la relación de relevancia estadística. Por lo que es más parsimonioso que el modelo de Hempel \cite{Salmon1970}.

En la sección 4 de su artículo, Salmon motiva su discusión al tratar de resolver el problema del caso único. Este problema parece funesto para la interpretación frecuentista de la probabilidad (que es la interpretación que Salmon favorece a lo largo de su artículo) y su uso en explicación. Salmon cree que no es así. Para justificar esto, desarrolla un aparato teórico para tratar con dicho problema. Una parte importante de su solución es hacer una distinción entre la clase de referencia y la clase de atributo. La clase de referencia es aquella que nos ayuda a partir el espacio de posibilidades en lo que sospechamos es lo que hace una diferencia en la estadística. La clase de atributo está asignada por la pregunta que hacemos. Por ejemplo, supongamos que queremos explicar porqué hoy hubo una tormenta. La clase de atributo es el conjunto de las tormentas el día de hoy. La clase de referencia puede ser el que los barómetros marquen un cierto número o bien la caída en la presión atmosférica en los últimos días. Si al condicionar nuestro evento a una de estas dos clase de referencia, descubrimos que una de ellas es relevante para el hecho de que hoy haya llovido, entonces esto explica por qué hoy hubo una tormenta.

Para evitar la vaguedad de la noción intuitiva, esta diferencia está definida en términos probabilísticos de la siguiente manera: si $P(A|B) \neq P(A|B\&C)$ entonces C hace una diferencia para la ocurrencia del evento A y es gracias a este ``hacer una diferencia'' que podemos explicar la ocurrencia del evento A a partir de la ocurrencia del evento C.

Este modelo tiene algunos detalles importantes: debemos ser capaces de hacer una ``partición homogénea'' del evento o fenómeno a explicar. Que sea una partición homogénea significa que dado un objeto $O$ y un evento $E$, debemos ser capaces de seccionar todas las propiedades relevantes de $O$ con respecto a $E$. Esta partición debe ser exhaustiva, es decir, que no falte alguna partición y no podamos agregar nuevas propiedades al conjunto. Además, cada una de estas particiones debe ser mutuamente excluyente \cite{Woodward2019}.

Para ejemplificar esto, retomemos el caso de mi dolor de cabeza: llamemos ``A'' al evento en el que disminuye mi dolor de cabeza y llamemos ``B'' al evento en el que tomo la pastilla. Ahora queremos explicar por qué disminuye mi dolor dado que tomo la pastilla $P(A|B)$, esto lo lograremos haciendo una partición homogénea del evento ``B''. Buenos candidatos para ello son: $C_{1}$ = ``no soy alérgico al medicamento que me tomé '', $C_{2}$ = ``que la pastilla de hecho sea un analgésico y no una menta'', [...], $C_{n}$ = ``X''. Si alguna de estas propiedades modifica la probabilidad en la ocurrencia de ``A'', entonces tenemos una explicación de ``A''. En palabras de Galavotti ``En esta perspectiva lo que cuenta para la explicación no es la alta probabilidad, como lo requería Hempel, sino estar en posición de afirmar que la distribución de probabilidad asociada con el \textit{explanandum} refleja la información más completa y detallada que tenemos disponible''\footnote{In this perspective what counts for the sake of explanation is not high probability, as required by Hempel, but being in a position to assert that the probability distribution associated with the explanandum reflects the most complete and detailed information attainable.} \cite{Galavotti2018}.

Este modelo resuelve algunos de los problemas que tenía el modelo de explicación que desarrolló Hempel: la noción de explicación que desarrolla Salmon no echa mano de leyes y, por tanto, no tiene el problema de distinguir entre leyes y generalizaciones accidentales\footnote{Aunque sí echa mano de generalizaciones estadísticas.}. Evitar hablar de leyes es más adecuado para usar dicho modelo con aquellas ciencias en las que no es obvio que las haya. Por otro lado tenemos que desechar la intuición de que explicar es ofrecer un argumento (aunque el costo no parece muy alto).

Con todo esto, el modelo aún tiene problemas relacionados con nuestra noción de causa. Salmon reconoce que este método es sensible al problema de hacer pasar las correlaciones por relaciones de dependencia. Salmon menciona algo acerca de cómo utilizar su método para rastrear relaciones causales, apelando a los estados de baja entropía y a la relación temporal entre el evento a explicar y la clase de referencia. Por ejemplo, si vemos un estado de baja entropía, podemos asumir que este estado es el temporalmente previo porque es muy poco probable que la entropía disminuya. Esto porque no son frecuentes (muy cercano a 0) los estados naturales de baja entropía. Los estados macro se comportan de manera análoga a los estados micro, por lo que la ocurrencia de un evento altamente improbable indica que hay un estado temporalmente previo que lo causó.

Pero aún restan problemas que resolver. Es común el eslogan de ``correlación no implica causalidad'' y el problema con la teoría de Salmon es que hace que las correlaciones que tienen una causa común sean explicativas o que no seamos capaces de rastrear la causa que explica el fenómeno. Por ejemplo, si mi dolor de cabeza se debe a que no he tomado agua y sin darme cuenta me tomo una pastilla del frasco que es un dulce entonces cuando lo tomé con agua quizás mi dolor disminuya. Que mi dolor disminuya está correlacionado con mi tomar la pastilla, pero no es su causa.

En el caso de causa común, el ejemplo del barómetro es ilustrativo: sabemos que siempre que una tormenta se avecina, hay un cambio en el barómetro. Pero no decimos que el hecho de que haya un cambio en el barómetro causa que una tormenta se avecine, esto tiene que ver con el cambio en la presión atmosférica.

Para tratar de resolver estos problemas, podemos comprometernos con causalidad (que es metafísica y epistémicamente problemático) o bien desechar el modelo sin más. Woodward tomó la primer estrategia: si algo es explicativo, es porque rastrea las causas de un fenómeno.

Problemas de correlación llevaron no sólo a Woodward, sino también a Salmon a decidir que una noción de causalidad era necesaria si queríamos ofrecer un buen análisis de la explicación \footnote{Salmon, por ejemplo, desarrolla un modelo alternativo: su modelo causal-mecánico \cite{Salmon1994}.}. Como nos dice Psillos ``Si las relaciones de Relevancia Estadística son explicativas, entonces tienen que capturar las dependencias causales correctas entre el \textit{explanandum} y el \textit{explanans}''\footnote{For if the relevant SR [Statistical Relevance] relations are to be explanatory, they have to capture the right causal dependencies between the \textit{explanandum} and the \textit{explanans}} \cite[p. 255]{Psillos2009}. En la siguiente sección nos dedicamos a exponer la solución que sugiere Woodward

\section{Retomando la causalidad}

\noindent El paso para asumir causalidad como la condición de relevancia adecuada, ayuda a solventar problemas como el de asimetría. Es claro que el asta explica el largo de la sombra porque el asta causa la sombra y no viceversa. Además, en el caso del barómetro, podemos indicar que hay una causa común al cambio del barómetro y al hecho de que haya una tormenta, esto es, la baja presión atmosférica, por lo que evitamos casos de correlación espuria. Esto incentiva a asumir la tesis de que la causalidad puede ser una clase de relevancia adecuada para resolver estos problemas.

Sin embargo, antes de continuar explorando a la causalidad como una solución a estos problemas, vale la pena hacer una distinción. La primera distinción, defendida por Anjum y Mumford en \citeyear{Anjum2018}, es la de mantener separados los aspectos epistémicos de los aspectos ontológicos de la causalidad. Si bien, las preguntas que son englobadas por el conjunto de los problemas ontológicos, \textit{e. g.}, ¿es la causalidad una relación o un proceso? y si bien es una relación ¿cuáles son los objetos \textit{relata} que pueden entrar en esta relación?, son preguntas que nos dan una pista de esta distinción, pero está lejos de ser una distinción completamente tajante. No es tajante porque dependiendo de cómo respondamos a ``¿qué es la causalidad?'' tendrá repercusiones en cómo respondemos a ``¿cómo obtenemos evidencia de relaciones causales?'' y viceversa.

Según Anjum, una razón para mantener separados ambos dominios es que es claro que en el ámbito legal esta distinción es importante. Su ejemplo del crimen perfecto ilustra esta distinción: pensemos en un crimen perfecto. Por definición un crimen perfecto es aquel que no deja ninguna evidencia. Sin embargo no podemos pasar de que no hay evidencia del crimen al hecho de que no hubo crimen en absoluto.

Una razón más para esta distinción es que al colapsar los problemas epistémicos con los problemas ontológicos, podemos caer en un tipo de operacionalismo. Por ejemplo, sabemos que diferentes métodos en ciencia tratan de buscar si hay causalidad: doble ciegos en medicina, aumento de probabilidades en economía, etc. Si cada uno de los métodos que tenemos para buscar causalidad tiene contraejemplos y no hacemos la distinción entre los rasgos epistémicos y ontológicos de la causalidad, entonces podemos llegar a afirmar que no hay una única cosa que sea la causalidad, sino que son varias. Woodward además de una teoría de la explicación causal, nos ofrece una teoría acerca de qué es la causalida. No queremos negar la tesis del pluralismo causal\footnote{Una defensa del pluralismo causal nos la ofrece Cartwright \citeyear{Cartwright2007}.}, y tampoco queremos afirmar la tesis del monismo causal. No queremos afirmar que el modelo de Woodward captura a todas las relaciones causales, sólo nos comprometemos a afirmar que captura algunas relaciones causales. Queda pendiente una defensa de si las relaciones causales que captura son todas las relaciones causales. Es decir, este trabajo es agnóstico con respecto a ese problema.

Esto parece una salida muy fácil al problema. Sin embargo, en esta sección queremos argumentar que el modelo de Woodward tiene valores que podemos rescatar y que son útiles a cómo de hecho se trabaja en ciencia. Aún si la teoría de Woodward no captura todos los fenómenos causales, sin duda nos ayuda a explicar una parte de estos, tal como la misma Cartwright concede \citeyear[cap. 7]{Cartwright2007}. Esto significa que podemos estar seguros de que captura un subconjunto de las relaciones causales. Sin afirmar que es la única manera, ni afirmando que el complemento difiere en tipo. Esto es una razón suficiente para tratar de aplicarlo a lo que nos importa en este trabajo.

%La segunda distinción importante es si la causalidad involucra al determinismo. Esto será discutido con más cuidado en el siguiente capítulo. Hume famosamente argumentó que las cuestiones de hecho están relacionadas por causa y efecto. La relación de causa y efecto viene de la experiencia. Ahora, ¿cómo justificamos las inferencias basadas en experiencia? Es aquí donde Hume argumenta que no hay manera en que podamos concluir que eventos similares seguirán ocurriendo en un futuro, ya que esta inferencia ni está basada en la experiencia, ni es demostrable. Por lo que nuestras inferencias basadas en la relación de causa y efecto no están justificadas \cite{Hume2017}.

%Hume resuelve este problema apelando a que estas inferencias están basadas en el hábito de ver que de un eventos ocurriendo, otro evento ocurre. Hume nos dice que la idea de causalidad engloba al menos cuatro propiedades: la causa precede temporalmente al efecto, contigüidad, regularidad sin conexión necesaria y la causa es diferente al efecto. Más aún, la tesis de Hume es reduccionista. Es reduccionista en el sentido en el que reduce a la causalidad al hábito. Este hábito tiene ciertas propiedades y excluye a la necesidad del análisis\footnote{Hay que notar que Hume está reduciendo los aspectos ontológicos de la causalidad a los aspectos epistémicos. Debido a que la única evidencia que tenemos es el hábito de que a ciertos eventos les siguen ciertas causas y esto no es evidencia de que haya conexión necesaria, entonces la causalidad no implica necesidad. En el capítulo 2 ofrecemos un argumento a favor de que la causalidad no implica conexiones necesarias, por tanto, excluyendo el determinismo}.

Continuando con la presentación del modelo de explicación manipulabilista de Woodward, hay que hablar de David Lewis, quien hizo en  \citeyear{Lewis1973a} un análisis contrafáctico de la causalidad. El análisis de Lewis es a primera vista problemático por dos razones: la primera razón es que depende de la noción de mundos posibles cercanos. Esta noción es vaga y Lewis sugiere que hay una diferencia gradual entre el mundo más cercano al actual y el actual. Es decir hay una infinidad de mundos posibles dentro de este rango. Por lo que cualquier diferencia entre el mundo actual y otro mundo posible es un corte arbitrario. La segunda razón es que hay que asumir la tesis del realimso modal, que afirma que los mundos posibles son concretos y esto infla la ontología.

El modelo de Woodward retoma parte de la postura causal de Lewis. Woodward también se apoya de los enunciados contrafácticos para decir que este proceso es en lo que consiste una relación causal. Sin embargo no es necesario que se comprometa con el realismo modal. Para salir de esta primera dificultad, Woodward sugiere que podemos explorar alguna semántica alternativa, entre algunas opciones están los diagramas de Markov \cite{pittphilsci18628}.

Con estas distinciones en mano, podemos comenzar a exponer la teoría causal de Woodward siguiendo su artículo \citeyear{Woodward2000} y su posterior libro \citeyear{Woodward2003}.  Woodward  retoma a la causalidad como un elemento importante en la explicación y desarrolla una teoría acerca de cómo podemos rastrear causas y por qué estas son explicativas. En sentido estricto, Woodward nos ofrece una respuesta a \textit{cómo} podemos rastrear relaciones causales. Como dijimos anteriormente, puede que esto sólo capture una parte, pero sin duda captura algo, de lo que es causalidad.

En la postura de Woodward, son importantes las nociones de ``invarianza'' e ``intervención''. Estas son importantes porque a partir de estas nociones se define una generalización que, a pesar de no ser una ley de la naturaleza, es explicativa. Si el modelo de Woodward logra hacer lo que se propone, entonces tendremos un modelo de explicación que no apele a leyes y, por tanto, más adecuado para ciencias como la biología. Además, retomando causalidad, evitamos la simetría de la que pecaba el modelo de Hempel y resolvemos los problemas de correlación del modelo de Salmon. Además, es claro que muchas de nuestras explicaciones exitosas (si bien no todas las explicaciones exitosas) son causales.

\textit{Grosso modo} Woodward nos dice que explicar tiene que ver con hacer explícitas las relaciones causales entre dos variables: sean ``$A$'' y ``$B$'' dos eventos cualquiera, decimos que el evento $A$ explica al evento $B$ cuando hay una relación causal que liga la ocurrencia de $A$ con la ocurrencia de $B$. Esto es sólo cuando al manipular $A$, el valor de $B$ cambia en consecuencia. El hecho de que haya una intervención no implica que tenga que haber un agente. Intervención está definido de manera que fenómenos naturales en los que sucede un evento $A$ y esto a su vez modifica el valor de otro evento $B$ cuenta como una intervención en el modelo de Woodward.

Para asegurar que esta relación es causal, este cambio en $B$ debe estar relacionado sólamente con los cambios en $A$ y no deberíamos poder explicar el cambio en $B$ por intervención directa. De manera más esquemática la noción de explicación de Woodward es definir una relación $R$ tal que $R<A, B>$ esté constreñida por las siguientes características: i) cambios en el valor de $B$ deben estar directamente relacionado con cambios en el valor de $A$ de manera que sin cambios en $A$, no habría habido cambios en $B$\footnote{Este criterio es evidentemente modal}; ii) mediante $R$ debemos ser capaces de hacer una ``generalización'' tal que dicha generalización nos describe el comportamiento del sistema en los casos donde la relación es invariante (que son casos en los que bajo ciertas restricciones si ocurre $A$, entonces ocurre $B$)\footnote{Supongamos por ejemplo que quiero saber bajo qué condiciones un vaso que se cae, se rompe. Podríamos variar la altura de la caída, así como el material sobre el que cae. Si, por ejemplo, tiráramos un vaso en un colchón, a 10 cm. de altura, no se romperá; si lo tiráramos de una altura de dos metros en un piso de concreto, seguro se romperá. Esto quiere decir que la relación entre tirar un vaso y que se rompa en consecuencia es invariante bajo algunos valores de altura y del material sobre el que cae.}.  iii) $A$ hace un cambio en $B$ y el cambio en $B$ no debe darse por ninguna otra ruta; iv) No hay causas diferentes a $A$ que cambien a $B$ (ya sea una causa común o alguna otra razón), por último todo debe estar acotado a i-iv \cite[p. 201]{Woodward2000}. Todo esto constituye la noción de intervención. Si diseñamos una manera en la que podamos intervenir en $A$, que cambié el valor de $B$ en consecuencia y dicha intervención cumple las características i-iv, entonces tenemos una explicación \textbf{causal} de la ocurrencia de $B$.

Por ejemplo, supongamos que una nueva píldora minimiza los dolores de cabeza. Esta píldora actúa disminuyendo la sensibilidad de dolor en todo el cuerpo, entonces si la tomara, disminuirá mi dolor. El cambio en el dolor de cabeza debe estar directamente relacionado con la toma de la píldora y no con que, por ejemplo, haya tenido un accidente que cercenó mi cabeza (algo que seguramente habría eliminado mi dolor). También tiene que ver con el posible evento en el que si no me hubiera tomado la píldora, entonces no hubiera disminuido mi dolor de cabeza.

Un ejemplo más elaborado de esto y que pone de manifiesto que no es necesario un agente que intervenga es el caso de cómo la luz y la temperatura afecta el proceso de floración en plantas del género \emph{Arabidopsis}. En \cite{AusinEnviro}, los autores argumentan a favor de la hipótesis que afirma que el proceso de florecimiento es un proceso altamente plástico. Los datos arrojan que en el caso de \emph{Arabidopsis}, hay al menos dos factores que modifican la velocidad con la que este género florece: temperatura y luz. En el caso de la temperatura se ha observado que si se somete a los especímenes a temperaturas bajas (aunque no al punto de congelamiento), el proceso de florecimiento se acelera. En el caso de la luz, los especímenes reaccionan a la luz roja, a la luz roja lejana (longitudes de onda entre 700 y 750 nm.) y a la luz azul. Cuando hay bajos niveles de estas tres, se promueve el florecimiento.

Lo que este ejemplo pretende mostrar es que no necesariamente debe haber un agente interviniendo directamente en los factores relevantes. Sin duda, se pueden recrear diferentes condiciones en el laboratorio. Por ejemplo, supongamos que un observador cayó en cuenta de que sus plantas florecían más rápido cuando estaban bajo una sombra que al sol directo. Esto podría ayudar a diseñar condiciones en las que el observador replique lo que accidentalmente observó. Puede entonces crear condiciones en las que mantenga la temperatura igual y que modifique la luz que llega a la planta. O bien puede mantener la luz fija y variar la temperatura. Si todo esto además cumple con las características que pide Woodward, podemos concluir que hay una relación causal.

Cabe aclarar que dichas intervenciones deben ser posibles. Por ejemplo puedo preguntarme qué pasaría en el caso en el que no me tomara la píldora. Si realmente es la píldora lo que hace que cese mi dolor de cabeza, entonces me seguirá doliendo en el caso en el que no me la tome. Puedo preguntarme también qué pasaría en caso de que el componente de la píldora fuese diferente al que de hecho es, etc. Puedo también preguntarme por qué un cuervo es negro, y puedo preguntarme qué debería cambiar para que el cuervo tuviera un color diferente. Pero sería absurdo preguntarme qué pasaría si en lugar de ser ``este'' cuervo fuera un cardenal. Puedo hacer que en el cuarto haya un cardenal y no un cuervo, pero no puedo hacer que ``este'' cuervo se convierta en un cardenal. Son casos como los anteriores los que acotan las intervenciones posibles.

Con respecto a la noción de invarianza, Woodward nos dice que cualquier generalización que describa una relación entre dos o más variables es invariante si se sostiene aún cuando se modifican otras condiciones. Esta noción de invarianza es lo que permite hacer generalizaciones de la relación entre dos variables. Porque si hay una relación causal entre $A$ y $B$ y dicha relación se sostiene aún cuando otras variables se modifican, entonces podemos decir que para cualquier $A$ y $B$ habrá la misma relación causal. Cuando esto se cumple, tenemos un indicio de que es posible manipular y controlar la variable independiente para ver qué cambios hay en la variable dependiente \cite{Woodward2000}.

Sin duda el modelo de Woodward tiene muchas virtudes. Primero tiene una aplicación clara para las ciencias especiales ya que no parte de la noción de ley, sino que construye generalizaciones como ``invarianza bajo intervenciones''. Woodward resuelve los problemas que tenía el modelo de Salmon al poner más restricciones en lo que deberíamos hacer cuando buscamos relaciones de dependencia causal. Otra virtud es que la noción de intervención encaja con el hecho de que en las investigaciones se llevan a cabo experimentos y que es a partir de ello que obtenemos información que indica si hay o no una relación entre variables.

\subsection{Cosas por resolver y algunos esbozos de soluciones}

\noindent Hasta aquí hemos dado un repaso muy breve de tres modelos de explicación. Empezamos por el modelo Nomológico-Deductivo de Hempel y discutimos los problemas más famosos que se le han detectado. Nos centramos en el problema de las leyes porque en ciencias especiales no es claro que haya enunciados que cumplan los criterios para ser una ley de la naturaleza (al menos no cumplen el conjunto de criterios de Nagel). Después expusimos el modelo de Salmon y mencionamos algunos de sus problemas. Algo importante que hay que señalar del primer modelo de Salmon es que el método mediante el que explicamos un fenómeno es, en parte, defectuoso: porque puede haber casos donde tomamos por cierta una explicación, pero al final descubrimos que hay una causa común (como el caso del barómetro), por tanto no tendríamos una explicación.

El modelo de Woodward resuelve el problema de ND porque no necesita que haya leyes para ofrecer una explicación. Además resuelve los problemas del modelo de Salmon ya que evita casos de causa común al ser tan astringente en las condiciones de lo que cuenta como una intervención. Por lo que hasta ahora parece ser una buena opción para hacer claro cómo las ciencias especiales explican.

Ojalá hayamos encontrado la panacea. Pero el modelo de Woodward aún tiene problemas que necesitamos resolver para que sea más adecuado. En primer lugar, esta noción de explicación depende fuertemente de contrafácticos. Woodward nos dice que una parte importante de la explicación es poder responder a preguntas contrafácticas, pensando en el caso en el que intervenimos en uno de los valores (si me tomo la píldora) para analizar que pasa con el otro valor (mi dolor de cabeza). Y los contrafácticos se han tomado con delicadeza porque no es claro cómo estamos justificados sobre nuestro conocimiento modal.

¿Qué son los enunciados contrafácticos? Un enunciado contrafáctico es un enunciado condicional en el que el antecedente se presenta contrario a los hechos. Por ejemplo ``si yo me tirara de un séptimo piso, entonces moriría''. Ahora bien, ¿qué hace verdadero a este condicional? Según la teoría de Lewis, este condicional es verdadero si en los mundos posibles suficientemente parecidos a este (mundos posibles cercanos), mi doble-lewisiano (contraparte) se tira de un séptimo piso y se muere.

Aquí cabe hacer una distinción dde dos cuestiones que involucran a los enunciados modales. Hay al menos dos cuestiones involucradas a los enunciados contrafácticos. Por un lado hay una cuestión semántica respecto a qué los hace verdaderos o falsos. Por otro lado, hay una cuestión epistémica acerca de cómo tenemos acceso a este tipo de enunciados. Dicho de otro modo, ¿cómo justificamos nuestro conocimiento modal?

El trato semántico de los contrafácticos depende de los mundos posibles. Siendo el de Lewis la explicación más famosa para este tipo de enunciados \citeyear{Lewis1973}, estos condicionales en la teoría de Lewis dependen de aceptar el realismo modal. El realismo modal es, a grandes rasgos, la tesis que afirma que existen los mundos posible a los que nos referimos. Esto es, existen de la misma manera en la que existe el mundo en el que estamos nosotros, pero están causalmente aislados de nuestro mundo.

Pero si están causalmente aislados, ¿cómo podemos jsutificar nuestro conocimiento modal? Si bien Lewis hace un análisis semántico de estos enunciados al darnos un criterio para decir cuándo son verdaderos o falsos, no tenemos claro cómo justificamos nuestro conocimiento de ellos. Uno de los objetivos es no asumir compromisos que no tienen un vínculo claro con la evidencia. Lo que hace que esta semántica de los contrafácticos no cumpla el requisito y, por tanto, habría que desecharla. En un capítulo, \cite[cap. 5]{Woodward2003} se dedica a los contrafácticos. Aquí Woodward nos dice que los contrafácticos relevantes para la explicación son aquellos que tienen una interpretación intervencionista. De esta manera podemos responder a preguntas del tipo ``¿qué si las cosas hubieran sido diferentes?'' como en el ejemplo de la píldora y mi dolor.

A pesar de ello, la exposición que hace Woodward no dice claramente qué semántica asume para las oraciones contrafácticas, por lo que asumimos que continúa usando la semántica de Lewis. Al ser estas una parte importante del análisis causal de Woodward y estar relacionado con la noción de explicación, es necesario ver qué alternativas podemos explorar. ¿Cómo sabemos que es verdad el condicional ``si me hubiera tirado de un séptimo piso, entonces hubiera muerto''? Una manera de hacer esto y no comprometer el significado de los contrafácticos, es hacer verdadero el condicional en nuestro mundo actual\footnote{Esta es una sugerencia que presentan Anjum y Mumford en \cite{Mumford2013}.}.

Esta forma de tratar a estos enunciados es una sugerencia para una teoría epistémica que haga claro cómo estamos justificados en nuestro conocimiento modal. Podríamos pensar, por ejemplo, que hacer verdadero el enunciado es analizar la información que hay disponible acerca de caídas a esa altura y contar el porcentaje de personas que mueren. En este sentido, cada uno de los eventos contaría como un mundo posible: en algunos se muere quien se tira, quizás en otros no. La relación de acceso se da en términos de ser el mismo tipo de situación. Roca-Royes en \citeyear{rocaroyes} presenta un marco teórico que nos permite hacer esta generalización. Esta teoría epistémica la llama empirismo modal por hacer que nuestro conocimiento modal dependa de cosas que pasan en el mundo actual. La sugerencia de Roca-Royes es justificar las modalidades \textit{de re} en términos de similitud. Ella sugiere que nuestro conocimiento de posibilidades es anterior a cualquier otro conocimiento modal. Sabemos, por ejemplo, que la mesa se \textbf{puede} romper porque en este mismo mundo hay mesas lo suficientemente similares que se han roto. Esta relación de similitud nos informa que es lo que pude pasar con otras mesas similares porque sabemos que actualidad implica posibilidad.

La noción de similitud involucrada en este análisis de los enunciados modales nos ofrece una solución ``empírica'' al problema de los enunciados modales. Roca-Royes menciona además que nuestro conocimiento modal puede estar informado no sólo por una mesa particular que se rompió, sino por un conjunto grande de mesas lo suficientemente similares que se han roto. Esto suma a la intuición inicial de que el revisar los casos en los que hay gente que se ha caído del septimo piso y muere, es información relevante para saber que es posible caer de un séptimo piso y morir. En algunos casos, analizar el contrafáctico consistirá en diseñar un experimento en el cual podamos intervenir para modificar los valores, como en el ejemplo de cómo la luz y la temperatura afecta el florecimiento de las plantas. En ambos casos es clara cuál es la relación de acceso y de cercanía.

Debido a la definición de intervención, el hecho de ser capaces de replicar el experimento, nos da un modelo distinto en el que fijamos las condiciones y sólo modificamos la variable que nos interesa explicar. La teoría de Lewis, a diferencia de la de Kripke, no requiere identidad estricta a través de mundo, los dobles lewisianos son objetos lo suficientemente parecidos sin ser idénticos a los del mundo actual. Entonces replicar un experimento en donde intervenimos la misma variable, y que dicho experimento no sea numéricamente idéntico podría ser una semántica alternativa para los contrafácticos. No dependemos de mundos posibles sino de situaciones posibles.

Por supuesto esta solución empírica de nuestro cnocimiento modal no está libre de problemas. Esta teoría padece de los problemas que generalmente asociamos a nuestro conocimiento inductivo, esto es, que o bien no está justificado, o bien hay que asumir algun postulado (la regularidad de la naturaleza) para poder justificarlo. Volviendo al antiguo problema de Hume. Por otro lado está el problema de definir exactamente qué queremos decir con ``similitud'' entre diferentes objetos ya que varios de ellos instancian diferentes propiedades y no es claro cuál de ellas hay que elegir: ¿es relevante que sean mesas de la misma tienda?, ¿que sean del mismo material?, ¿que sean el mismo modelo de mesa?, etc.

%Si bien el modelo de Woodward es útil para decir cuándo hay una relación causal y cuando no, aún deja de lado cómo es que comenzamos a indagar sobre cierto tipo de relaciones. Esto es, el modelo de Woodward toma como valores ciertas relaciones y nos ofrece una maquinaria para decir si dicha relación es causal o no. Pero, ¿cómo sabemos en principio que variables están relacionadas? Una sugerencia es que la búsqueda de relaciones es algo que se deja a los investigadores particulares, es decir se deja esta parte al contexto de descubrimiento. Lo que hace el modelo de Woodward es operar en el contexto de justificación diciendo si dichas relaciones son o no causales. Esto parece también ser algo que sugiere Hanson en su segundo capítulo en \citeyear{Hanson1958}.

Es importante aclarar la epistemología de los contrafácticos porque es central para la teoría de la explicación que presenta Woodward. La explicación consiste en decir qué causa que un evento ocurra. Pensar en una manera de intervenir en la variable que sospechamos es la culpable, y preguntarnos qué pasaría en este caso; y si en este caso en el que modificamos la variable que sospechamos el la culpable y ocurre el evento que estamos estudiando, entonces tenemos una explicación causal de dicho evento.

Sin embargo, queda pendiente generalizar el punto para las explicaciones causales de fenómenos singulares, porque fenómenos irrepetibles como la transición de la vida a la tierra, el big bang o la revolución mexicana, no son experimentalmente repetibles y la evidencia disponible para derivar conclusiones (por ejemplo, viendo qué paso en otras revoluciones del mundo), es vaga. En estos casos se padece el problema de exactamente qué propiedad tienen que instanciar para decir que son caso similares. No es lo mismo que hacer un diseño experimental que podamos repetir en condiciones parecidas. Perdemos expresividad, pero ganamos una epistemología más clara que además encaja con cómo se trabaja en ciencias.

Por último hay que decir algo sobre las leyes. En el modelo de Woodward, las leyes no son necesarias para explicar. En \cite{Woodward2000} Woodward menciona que la explicación tiene que estar estrechamente relacionada al cambio. Hay leyes que no están relacionadas al cambio, por tanto las leyes no son explicativas. Más aún, algunas generalizaciones explicativas(relacionadas al cambio) no son leyes. Por tanto, las leyes no son suficientes ni necesarias para la explicación. Sin embargo, en una publicación posterior \citeyear{Woodward2003} Woodward nos dice que hay sin duda algunas leyes que explican y que bien ND y su modelo intervencionista pueden convivir: una ley es sólo el caso límite de una generalización.

Hasta este momento hemos expuesto que el modelo de Woodward es una alternativa para la explicación en ciencias especiales como la biología, en particular porque no echa mano de leyes, además de que al ser tan estricto con los criterios de qué cuenta como una intervención, resuelve el problema de la causa común que tiene el modelo de Relevancia estadística. Pero aún tenemos un problema con respecto a dejar que las causas próximas sean parte de las explicaciones en biología evolutiva. En su artículo a Mayr \citeyear{Mayr1998} le preocupaba que hubiera una distinción entre las causas próximas y últimas porque esta disntinción hace que la biología evolutiva sea especial, en el sentido de que es diferente a otras ciencias como la física en dónde hay fenómenos determinados. Todo esto sin tener que echar mano de alguna entelequia. A Mayr le interesa hacer esta distinción debido a que parece sugerir que los conjuntos de causas próximas sí son deterministas, mientras que la biología evolutiva no lo es. Menciona, por ejemplo, que ``

\begin{quote}
  En las teorias de \textit{causalidad probabilista}, nuestro objetivo es dar una conceptualización de relaciones causales que son inherentemente probabilistas, estocásticas o indeterministas. Así que viene bien hablar de `causalidad indeterminista'. En las \textit{teorías probabilistas} de la causalidad nuestro objetivo es dar una caracterización probabilista u ofrecer modelos de relaciones causales que pueden ser o no ellas mismas probabilistas. En este segundo caso podemos o bien admitir que existen casos de causalidad determinista y casos de causalidad indeterminista, o podemos sostener que toda relación causal es determinista, pero que nuestros modelos de relaciones causales es probabilista porque o bien no tenemos conocimiento completo o bien debido a errores de medida\footnote{In theories of \emph{probabilistic causation} we aim to provide a conceptualization of causal relations that are inherently probabilistic, stochastic, chancy or indeterministic. So we may well talk about ‘indeterministic causation’. In \emph{probabilistic theories} of causation we aim to provide a probabilistic characterization or modelling of causal relations that may or may not be probabilistic in themselves. In this second case we may either admit that there exist cases of deterministic causation and indeterministic causation, or we could hold that causation is all deterministic, but our modelling of causal relations is probabilistic because we don’t have complete knowledge or because of measurement error.deterministic, but our modelling of causal relations is probabilistic because we don’t have complete knowledge or because of measurement error.}. \cite[p. 82-83]{Illari2014}
\end{quote}

En el siguiente capítulo nos proponemos a armar un caso en contra de que la causalidad implica determinismo. El determinismo se define como la tesis que afirma que podemos predecir exactamente que sucederá en un tiempo $t_{n}$ a partir de $t_{n-1}$ en conjunción con las leyes de la naturaleza. Nosotros nos proponemos a ir en contra de que las leyes indican conexiones necesarias en el mundo, debilitando así la tesis determinista.
