%El autor de este texto es Oscar Abraham Olivetti Alvarez

\chapter*{Reflexiones finales}

\noindent Llegado a este punto hemos expuesto razones para tomar al modelo de explicación de Woodward como una teoría de la explicación que puede ser aplicada a ciencias como la biología. En el primer capítulo hicimos una distinción entre los aspectos metafísicos y los aspectos epistemológicos de la causalidad y ofrecimos razones a favor de tomar a la teoría de Woodward como una buena alternativa para elucidar el concepto de explicación. Algunos de los valores del modelo de Woodward de hecho encajan con nuestras pretensiones explicativas en áreas que no son la física. Quizás el valor más relevante es el hecho de que no necesita leyes para explicar fenómenos y que puede dar cuenta de causas sin que haya determinismo (lo que es deseable dado el factor contingente en la evolución).

Sin embargo, aún hacía falta una defensa de por qué la causalidad no implica determinismo. Aún si la teoría de Woodward es una buena teoría de la explicación, ésta también pretende elucidar el concepto de lo que es una causa. Esto nos lleva directo a las teorías de la causalidad, en particular a los aspectos metafísicos de la causalidad. Por lo regular se divide a las teorías de la causalidad en humeanas y anti-humeanas. Siendo las anti-humeanas las que afirman la tesis de que hay conexiones necesarias en la naturaleza. En la segunda parte de este trabajo argumentamos en favor de la tesis de que la causalidad no implica conexiones necesarias, dando apoyo a las tesis humeanas de la causalidad.

Para defender esto, comenzamos notando que las teorías anti-humeanas de la causalidad toman a las leyes naturales como evidencia de que hay conexiones necesarias. Nos concentramos en presentar una alternativa argumentando que la necesidad en la naturaleza es aparente. Esta necesidad aparente surge del hecho de que las llamadas leyes naturales son modelos particulares de un sistema formal, y por supuesto que una buena derivación formal hace necesaria su conclusión. Pero esta necesidad es una propiedad de la teoría formal, no de la naturaleza misma.

En este último capítulo quisimos integrar lo dicho hasta ahora. En primer lugar el modelo de Woodward es una teoría de la causalidad que no implica determinismo, por lo que nos parece que es útil para trabajar en biología dadas sus características particulares. Además refleja algo de la práctica experimental de los biólogos y da espacio a la contingencia. En segundo y último lugar las leyes no son necesarias ni para explicar, ni necesarias en términos metafísicos. Esto deja libre el camino para una teoría de la explicación causal en biología .

%El autor de este texto es Oscar Abraham Olivetti Alvarez
