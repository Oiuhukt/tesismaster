%El autor de este texto es Oscar Abraham Olivetti Alvarez

\chapter*{Reflexiones finales}

\noindent Llegados a este punto hemos propuesto que el modelo de Woodward nos puede ayudar a resolver el problema de individuar los ``problemas de diseño'' que tiene la noción ecológica de adecuación. En el primer capítulo de este trabajo expusimos cuál es el problema clásico de la adecuación. Ofrecimos una caracterización de adecuación que no es tautológica: la adecuación ecológica. Además ofrecimos un esquema de solución al problema de individuar los problemas de diseño utilizando el modelo de Woodward.

En este capítulo integramos al modelo de Woodwad con la práctica experimental para aclarar el concepto de adecuación. Defendimos que, para resolver el problema de que el concepto de adecuación es tautológico, hay que leer causalmente el enunciado ``los más aptos dejan más descendencia'' afirmando que los más aptos son aquellos que mejor resuelven problemas de diseño impuestos por el medio ambiente.

Para medir la adecuación, afirmamos que hay que modelar el entorno natural de una población en un ambiente de laboratorio. Si lo observado en el laboratorio difiere de nuestros valores estimados, entonces decimos que ha habido deriva génica. Si lo observado es consistente con lo estimado, entonces apelamos a una explicación por selección natural.

En el segundo capítulo de este trabajo defendimos por qué el modelo que presenta Woodward tiene ventajas frente a otros modelos alternativos de explicación. En primer lugar el modelo de Woodward es una teoría de la explicación causal que no requiere de leyes en su formulación, por lo que nos parece que es útil para trabajar en biología dadas sus características particulares. Además, refleja algo de la práctica experimental de los biólogos. Asumir esta solución pretendida tiene además la ventaja de que nos permite distinguir entre un proceso de selección natural y un proceso por deriva génica.

El segundo capítulo de este trabajo se centró en mostrar cuáles son las ventajas del modelo de explicación causal de Woodward frente a otros modelos de explicación. En particular, nos centramos en mostrar las ventajas frente al modelo Nomológico-Deductivo y el modelo de Relevancia estadística de Salmon. Buscamos posicionar al modelo de Woodward como una alternativa a estos dos modelos y tratamos de ofrecer una solución a los problemas que tiene el mismo modelo de Woodward, en particular el problema de cómo tenemos acceso epistémico a los enunciados contrafácticos.

A pesar de haber hecho claro nuestro punto, aún quedan problemas que hay que resolver para poder afirmar con convicción mucho de lo dicho en este trabajo. Por ejemplo, hemos afirmado que podemos utilizar el modelo de Woodward para justificar relaciones causales en biología evolutiva. Lo hicimos apelando a que en un ambiente de laboratorio es posible intervenir variables. Estas intervenciones harán que ciertos individuos se reproduzcan con más éxito que otros y es analizando esta población en el laboratorio que aventuramos una conclusión causal.

Aún queda justificar cómo tenemos acceso epistémico a enunciados contrafácticos. No defendimos explícitamente cuál sería la semántica de estos enunciados, con respecto a la epistemología hicimos una sugerencia a partir del trabajo de  Roca-Royes. La sugerencia hecha en este trabajo fue que podíamos tener acceso a través del diseño experimental. Esto quiere decir, que hacemos verdaderos los contrafácticos en el mundo actual (cabe aquí mencionar si en este sentido aún deberíamos llamarlos contrafácticos). Esto sugiere que tenemos acceso a los contrafácticos a través de la replicación experimental. Sin embargo, esta sugerencia tiene problemas acerca de qué se considera lo suficientemente análogo para poder entonces obtener conocimiento de los contrafácticos.

Quedan entonces pendientes al menos dos cuestiones importantes en lo que respecta a este trabajo. Queda pendiente el desarrollo de una semántica de contrafácticos apropiada. Una cuyos compromisos ontológicos sean menores que los del realismo modal. Por otro lado queda la evaluación de la teoría epistemológica propuesta en este trabajo, en particular tratar de resolver algunos de los problemas que presenta.

Sin embargo, aún hacía falta una defensa de por qué la causalidad no implica determinismo. Aún si la teoría de Woodward es una buena teoría de la explicación, ésta también pretende elucidar el concepto de lo que es una causa. Esto nos lleva directo a las teorías de la causalidad, en particular a los aspectos metafísicos de la causalidad. Por lo regular se divide a las teorías de la causalidad en humeanas y anti-humeanas\footnote{Aquí seguimos la distinción que hace Schrenk en \citeyear{Schrenk2017} en donde divide a las tesis de la causalidad en Humeanas y no-humeanas. Siendo las no-humeanas las que afirman la tesis de que no hay conexiones necesarias en el mundo.}. Siendo las anti-humeanas las que afirman la tesis de que hay conexiones necesarias en la naturaleza. En la segunda parte de este trabajo argumentamos, en contra de Graves y compañía, a favor de la tesis de que la causalidad no implica conexiones necesarias, dando apoyo a las tesis humeanas de la causalidad.

Para defender esto, comenzamos notando que las teorías anti-humeanas de la causalidad toman a las leyes naturales como evidencia de que hay conexiones necesarias en el mundo. Nos concentramos en presentar una alternativa argumentando que la necesidad en la naturaleza es aparente. Esta necesidad aparente surge del hecho de que las llamadas leyes naturales son modelos particulares de un sistema formal, y por supuesto que una buena derivación formal hace necesaria su conclusión. Pero esta necesidad es una propiedad de la teoría formal, no de la naturaleza misma.

Creemos que en este trabajo hemos posicionado al modelo de Woodward como una alternativa para describir explicaciones causales en biología evolutiva. Creemos además que hemos expuesto lo suficiente como para que hablar de causalidad en biología no sea tan problemático como resulta al principio. Creemos que esta preocupación de Mayr se debía a que la causalidad se entendía en términos mecanicistas y que estaba involucrado cierto riesgo de determinismo que no encontramos en biología evolutiva. A pesar de esto, nosotros creemos que la solución que ofrecemos en base en el modelo de explicación de Woodward tiene sus limitaciones. En particular, falta un desarrollo de la epistemología de contrafácticos que nos ayude a resolver el problema de cómo sabemos cuáles enunciados contrafácticos son verdaderos. Aquí hicimos sólo un esbozo de solución, pero ahondar en este problrema nos ayduará a desarrollar mejor el modelo para que se adecúe a nuestros compromisos epistémicos. Llenar este hueco es algo que pretendemos seguir investigando a futuro. 

%El autor de este texto es Oscar Abraham Olivetti Alvarez
