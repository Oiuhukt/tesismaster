%El autor de este texto es Oscar Abraham Olivetti Alvarez

\chapter*{Reflexiones finales}

\noindent Llegado a este punto hemos expuesto razones para tomar al modelo de explicación de Woodward como una teoría de la explicación que puede ser aplicada a ciencias como la biología. En el primer capítulo hicimos una distinción entre los aspectos metafísicos y los aspectos epistemológicos de la causalidad y ofrecimos razones a favor de tomar a la teoría de Woodward como una buena alternativa para elucidar el concepto de explicación en el nivel epistémico. Algunos de los valores del modelo de Woodward de hecho encajan con nuestras pretensiones explicativas en áreas que no son la física. Quizás el valor más relevante es el hecho de que no necesita leyes para explicar fenómenos y que puede dar cuenta de causas sin que haya determinismo (lo que es deseable dado el factor contingente en la evolución).

Sin embargo, aún hacía falta una defensa de por qué la causalidad no implica determinismo. Aún si la teoría de Woodward es una buena teoría de la explicación, ésta también pretende elucidar el concepto de lo que es una causa. Esto nos lleva directo a las teorías de la causalidad, en particular a los aspectos metafísicos de la causalidad. Por lo regular se divide a las teorías de la causalidad en humeanas y anti-humeanas\footnote{Aquí seguimos la distinción que hace Schrenk en \citeyear{Schrenk2017} en donde divide a las tesis de la causalidad en Humeanas y no-humeanas. Siendo las no-humeanas las que afirman la tesis de que no hay conexiones necesarias en el mundo.}. Siendo las anti-humeanas las que afirman la tesis de que hay conexiones necesarias en la naturaleza. En la segunda parte de este trabajo argumentamos en favor de la tesis de que la causalidad no implica conexiones necesarias, dando apoyo a las tesis humeanas de la causalidad.

Para defender esto, comenzamos notando que las teorías anti-humeanas de la causalidad toman a las leyes naturales como evidencia de que hay conexiones necesarias. Nos concentramos en presentar una alternativa argumentando que la necesidad en la naturaleza es aparente. Esta necesidad aparente surge del hecho de que las llamadas leyes naturales son modelos particulares de un sistema formal, y por supuesto que una buena derivación formal hace necesaria su conclusión. Pero esta necesidad es una propiedad de la teoría formal, no de la naturaleza misma.

En este último capítulo quisimos integrar lo dicho hasta ahora. En primer lugar el modelo de Woodward es una teoría de la causalidad que no implica determinismo, por lo que nos parece que es útil para trabajar en biología dadas sus características particulares. Además refleja algo de la práctica experimental de los biólogos y da espacio a la contingencia. En segundo y último lugar las leyes no son necesarias para explicar. Esto deja libre el camino para una teoría de la explicación causal en biología.

En este último capítulo integramos al modelo de Woodwad con la práctica experimental para aclarar el concepto de \emph{fitness}. Defendimos que, para resolver el problema de que el concepto de \emph{fitness} es tautológico, hay que leer causalmente el enunciado ``los más aptos dejan más descendencia'' afirmando que los más aptos son aquellos que mejor resuelven problemas de diseño impuestos por el medio ambiente.

Para medir el \emph{fitness}, afirmamos que hay que modelar estadísticamente a la población. Si lo observado en el laboratorio difiere de nuestros valores estimados, entonces decimos que ha habido deriva génica. Si lo observado es consistente con lo estimado, entonces apelamos a una explicación por selección natural. Defendimos, por tanto, que la teoría de la selección natural es de naturaleza dinámica. Apoyamos, además, que la deriva génica es una explicación puramente estadística.

A pesar de haber hecho claro nuestro punto, aún quedan problemas que hay que resolver para poder afirmar con convicción mucho de lo dicho en este trabajo. Por ejemplo, hemos afirmado que podemos utilizar el modelo de Woodward para justificar relaciones causales en biología evolutiva. Lo hicimos apelando a que en un ambiente de laboratorio es posible intervenir variables. Estas intervenciones harán que ciertos individuos se reproduzcan con más éxito que otros y es analizando esta población en el laboratorio que aventuramos una conclusión causal.

Sin embargo, aún queda justificar cómo tenemos acceso epistémico a enunciados contrafácticos. No defendimos explícitamente cuál sería la semántica de estos enunciados, con respecto a la epistemología hicimos una seugerencia a partir del trabajo de  Roca-Royes. La sugerencia hecha en este trabajo fue que podíamos tener acceso a través del diseño experimental. Esto quiere decir, que hacemos verdaderos los contrafácticos en el mundo actual (cabe aquí mencionar si en este sentido aún deberíamos llamarlos contrafácticos). Esto sugiere que tenemos acceso a los contrafácticos a través de la replicación experimental. Sin embargo, esta sugerencia tiene problemas acerca de qué se considera lo suficientemente análogo para poder entonces obtener conocimiento de los contrafácticos. 

Quedan entonces pendientes al menos dos cuestiones importantes en lo que respecta a este trabajo. Queda pendiente el desarrollo de una semántica de contrafácticos apropiada. Una cuyos compromisos ontológicos sean menores que los del realismo modal. Por otro lado queda la evaluación de la teoría epistemologica propuesta en este trabajo, en particular tratar de resolver algunos de los problemas que presenta.



%El autor de este texto es Oscar Abraham Olivetti Alvarez
