%El autor de este texto es Oscar Abraham Olivetti Alvarez

\chapter{Una definición causal de \emph{fitness}}


\noindent El concepto de \emph{fitness} es central a la teoría de la evolución por selección natural. Decimos que hay selección natural cuando los organismos más aptos son los que sobreviven. En contraste, la deriva génica es el proceso en el cual no se involucra a los más aptos, sino que es un proceso puramente estocástico.

Hay una definición de \emph{fitness} en términos de genética de poblaciones. Esta definición \textbf{define} y \textbf{mide} el fitness como reproducción diferencial. Esto quiere decir el cambio que hay en el número de organismos a través de diferentes generaciones. Esta definición diferencial trata de hacer claro el concepto de \emph{fitness} dando una definición cuantitativa. Cuando observamos en la población un cambio en la frecuencia alélica, entonces podemos hablar de que ha operado la selección natural.

Sin embargo, esta definición tiene algunas complicaciones. Una complicación importante es que vuelve el concepto de fitness una tautología. Si definimos a los organismos con más \emph{fitness} como aquellos que tienen más descendencia y fitness está definido en términos de dejar más descendencia, entonces nuestra definición es tautológica. Es por ello que queremos leer causalmente la oración anterior: que un organismo tenga más \emph{fitness} causa que tenga más descendencia. Si esto es así, entonces hay que ofrecer una nueva definición de \emph{fitness} que nos permita hacer esto.

En esta discusión sobre el \emph{fitness} hay dos cuestiones de por medio. La primera es ¿qué es el \emph{fitness}? y la segunda es ¿de qué objetos podemos predicarlo? Aquí nos concentraremos en la primera de estas cuestiones.

Hay opciones disponibles para definir el concepto de \emph{fitness} que pueden salvar de la carga tautológica que parecen tener las explicaciones por selección natural. Una de estas opciones es tomar al fitness como un primitivo de la teoría y de esta manera salvarnos de la carga. Sin embargo, queremos que el concepto nos sea de utilidad en explicaciones sobre selección natural. Un primitivo sin definir no ayuda para usar el concepto de forma que expliquemos por qué los más aptos son los que sobreviven. Por tanto, necesitamos una mejor estrategia que nos ayude a definir la aptitud que tienen unos organismos y no otros.

Dos opciones famosas restan para atacar este problema. Podemos interpretar al \emph{fitness} como una propensión, o bien interpretar al fitness como un concepto ecológico. El concepto de \emph{fitness} como propensión es volver a este una propiedad disposicional. Las propiedades disposicionales son aquellas que sólo se expresan cuando llevamos a cabo una acción. Por ejemplo, el azúcar es soluble en agua, sin embargo, no observamos la solubilidad del agua a menos que la pongamos en agua. En analogía, los organismos tendrán la propensión de dejar $n$ número de descendientes sin que de hecho se reproduzcan. La definición de fitness según los propensionistas: $x$ es más apto que $y$ en un ambiente $E =_{df}$ $x$ tiene una probabilidad más alta de dejar más descendencia que $y$.

Esta definición nos libra de la carga tautológica que tiene la definición original de \emph{fitness}. Sin embargo, en esta definición queda por aclarar exactamente cómo medir la probabilidad de que $x$ deje mayor descendencia que $y$. Una opción es que se mida la frecuencia relativa en un número muy grande de generaciones. La ley de los números grandes nos dirá cuál será la probabilidad a ``la larga'' de que un organismo deje más descendencia que otro. Sin embargo, esta lectura no es apta para ayudarnos a explicar. Nos importa que podamos medir el \emph{fitness} en generaciones finitas y no que la propensión dependa de lo que suceda ``a la larga''. Si la propensión depende de lo que pasará a ``la larga'' y esto es potencialmente infinito, entonces no tendremos acceso a dicha propensión.

Para poder salvar al \emph{fitness} de la carga tautológica y aclarar cómo medir esta propensión de los organismos, quisiéramos hacer una propuesta: que podemos definir causalmente al \emph{fitness} utilizando la maquinaria que nos ofrece Woodward. Esto lo haremos apoyándonos en el argumento que exponen Bouchard y Rosenberg \cite{Bouchard2004}. Bouchard y Rosenberg defienden una noción de \emph{fitness} ecológico. Si bien se presenta al \emph{fitness} ecológico y al \emph{fitness} como propensión como contradictorias \cite{sep-fitness}, creemos que podemos tomar una parte de ambas nociones pueden a través de utilizar a la teoría causal de Woodward como modelo causal.

La definición de \emph{fitnes} se inserta en un debate sobre cómo debemos interpretar a la teoría de la selección natural: como una teoría sobre fuerzas o una teoría estadística. Este debate lo tomamos de lo expuesto en tres artículos. Por un lado se defiende que la selección natural es de naturaleza causal y a nivel de individuos \cite{Bouchard2004}, en otro se defiende que es simplemente una consecuencia estadística de las poblaciones de organismos \cite{Walsh2002}, por último, hay una defensa de que la selección natural es de naturaleza causal y es a nivel de poblaciones \cite{Millstein2006}. En este trabajo queremos defender que una vez definida la noción central de \emph{fitness} en términos causales, es más natural decir que la selección natural es de naturaleza dinámica, es decir, sobre fuerzas.

Antes de continuar con el debate acerca del \emph{fitness} y pasar a la interpretación de la teoría de la selección natural, quisiéramos motivar la tesis de la naturaleza dinámica exponiendo algunas explicaciones causales en biología. Nos parece que hacer esto es necesario si es que queremos definir causalmente un término que está al centro de la teoría de la evolución por selección natural. A esto dedicamos en el siguiente apartado.

\section{Explicaciones causales en biología}

\noindent La síntesis moderna en biología evolutiva fue un cambio enorme para la teoría darwinista. En pocas palabras, lo que se hizo durante este periodo fue acercar la genética mendeleiana y la teoría de la selección natural de Darwin. En su investigación, Darwin había descubierto que hay variación entre organismos de una misma población. Además, se podía observar que algunas de estas variaciones se mantenían entre los padres y su progenie. Más aún, algunas de estas variaciones son ventajosas para los organismos. Estas variaciones ventajosas hacen que ciertos individuos tiendan a tener más descendencia como resultado.

De manera esquemática la teoría de la selección natural sostiene que\footnote{1. There is variation in morphological, physiological, and behavioural traits among members of a species (the principle of variation), 2. The variation is in part heritable, so that individuals resemble their relations more than they resemble other individuals and, in particular, offsprings resemble their parents (the principle of heredity), 3. Different variants leave different numbers of offspring either in inmediate or remote generations (the principle of differential \emph{fitness})}:

\begin{enumerate}
  \item Hay variación en las características morfológicas, fisiológicas y de comportamiento entre los miembros de una especie.
  \item La variación es en parte heredable, de manera que lo individuos se parecen más a sus padres que a otros miembros de la especie.
  \item Diferencias en las variaciones tiene como consecuencia diferencias en el número de descendientes, bien en generaciones inmediatas, o bien en generaciones más distantes\cite{Godfrey-Smith2013}.
\end{enumerate}

A pesar de que podemos observar que los organismos tienden a parecerse más a sus progenitores que a otros organismos dentro de la misma población, Darwin no dijo cuál era el mecanismo de herencia mediante el cuál se transmitían estas ventajas de los padres a su progenie. La síntesis moderna afirma que estas características se heredan principalmente mediante los genes, cerrando la brecha entre herencia y selección natural y afirmando que hay un mecanismo físico que es el que explica estos parentescos. Esto de entrada ya nos da una pauta para una explicación causal en biología. Los genes que codifican para ciertas proteínas son causalmente responsables y podemos dar una explicación causal al nivel genético.

Pero esto no es suficiente para incorporar causalidad a las explicaciones evolutivas. No lo es porque es claro que las explicaciones genéticas sí pueden incorporar causas próximas, tal como afirmaba Mayr. Pero lo que aquí nos concierne son las explicaciones evolutivas, que es justo donde se encuentra el concepto de \emph{fitness} y al que Mayr nos dice, quedan excluidas las causas próximas.

Ahora bien, la síntesis moderna fue bastante reticente en aceptar cualquier otro mecanismo de herencia que no fuera la herencia genética. Por lo que se excluyeron otros tipos de explicación y mecanismos no genéticos, o bien que no tuvieran una reducción a la genética. Ya desde hace varios años se ha comenzado a explorar si es necesario incorporar nuevos mecanismos de herencia. De ser esto correcto, estos nuevos mecanismos de herencia y su incorporación a la teoría tienen consecuencias importantes en qué evidencia debemos tomar para justificar hipótesis en selección natural y por supuesto ponen en tela de juicio la distinción que Mayr propone. Todo esto está estrechamente relacionado con lo que se ha denominado la Síntesis Extendida.

En este trabajo no vamos a discutir las consecuencias filosóficas y biológicas que tiene el integrar nuevos mecanismos de herencia. Lo que vamos a hacer es ofrecer ejemplos que apoyen la hipótesis de que los mecanismos de herencia no son todos genéticos. Haremos esto porque para este trabajo es útil el integrar estos nuevos modos de herencia para poder incorporar causas próximas en las explicaciones evolutivas.

Una advertencia antes de continuar: el que debamos integrar una dimensión ambiental es aún un debate abierto. Por lo que en un futuro la investigación puede encontrar una manera de reducir los factores ambientales a factores genéticos, lo que haría que debamos excluir la dimensión ambiental de las explicaciones evolutivas. Esta estrategia es la que siguen algunos de los biólogos que aún defienden algún tipo de neo-darwinismo. Esto sin duda es posible, pero la investigación actual en biología apunta en la otra dirección  \cite{Bateson2014}. Hay investigaciones que de hecho incorporan una dimensión ambiental y que afirman que hay otros medios de herencia no necesariamente genéticos. Esto es indicio de que podemos hablar de causas próximas en biología evolutiva. Los ejemplos que hemos expuesto a lo largo del trabajo están en consonancia con esta tesis.

A pesar de la reticencia de algunos biólogos, se han investigado sistemas de herencia distintos que no son herencia genética. Por ejemplo en el libro de Jablonka y Lamb \citeyear{Jablonka2020}, las autoras argumentan que hay diferentes mecanismos de herencia que no necesariamente están al nivel genético. Entre ellos se encuentran los mecanismos epigenéticos y la imitación del comportamiento. La construcción de nicho también es un caso en el que puede haber herencia no genética. Aún con la evidencia experimental, las autoras comentan que se ha relegado este tipo de mecanismos con el argumento de que son triviales porque no hacen una diferencia en términos evolutivos. Sin embargo, las autoras argumentan que hay evidencia de lo contrario.

Algunos de los ejemplos expuestos hasta este momento tienen este espíritu. Estos ejemplos no sólo pretenden ilustrar que la teoría de la causalidad de Woodward es compatible con el quehacer del biólogo evolutivo, sino además ofrecer evidencia para lo que argumentan Jablonka y Lamb, a saber, que hay una importante injerencia del medio en el que se desarrollan los organismos.

Regresemos a alguno de nuestros ejemplos anteriormente señalados. La luz y la temperatura afectan el proceso de florecimiento en plantas del género \emph{Arabidopsis}. En el artículo \cite{AusinEnviro}, los autores desarrollan un modelo experimental en el que observaron que hay una relación entre la temperatura y la luz afectan cómo florecen las plantas del género \emph{Arabidopsis}. Para el experimento se sometió a los especímenes a temperaturas bajas (aunque no al punto de congelamiento) y observaron cómo el proceso de florecimiento se acelera en consecuencia. Esto según la metodología de Woodward nos permite concluir que hay una relación causal entre la temperatura y el florecimiento. En el caso de la luz, se observó que los especímenes reaccionan a la luz roja, a la luz roja lejana (longitudes de onda entre 700 y 750 nm.) y a la luz azul. Cuando hay bajos niveles de estas tres, se promueve el florecimiento. Esto indica que hay una relación causal entre los bajos niveles de este tipo de luces y el florecimiento. Para esto, se interviene en las condiciones de temperatura y de luz para poder concluir que hay una relación causal. Pasa lo mismo en el caso de \emph{A. sagrei} y \emph{L. Carinatus}, se interviene la variable depredador y se observa que hay un cambio en el tamaño de las extremidades de \emph{A. sagrei}.

Esto se manifiesta en que una parte de la investigación biológica actual pretende integrar factores del ambiente en sus explicaciones de fenómenos biológicos. Esto indica que las causas próxima sí están presentes en las explicaciones evolutivas por selección natural. Que haya un nuevo depredador en el medio de \emph{A. sagrei} causa que se seleccionen los organismos que tienden a tener extremidades más largas. Por lo que podemos fijarnos en como el fenotipo interactúa con el ambiente y como hay características que pueden ser seleccionadas. Esto de nuevo lleva a pensar que la distinción hecha por Mayr deja de ser útil para explicar fenómenos evolutivos. Los límites de la distinción de Mayr han sido explorados recientemente en \cite{Uller2020, Dayan2020, Laland2011}. Pensemos en un ejemplo más. El experimento realizado por Amarillo Suárez y Fox \citeyear{Amarillo-Suarez2006}.

Hay insectos que se desarrollan dentro de un hospedero. Se tiene evidencia que el hospedero en el que se desarrollan las crías tiene influencia en el tamaño de los insectos. En el artículo de Amarillo-Suárez y Fox, se explora cómo el hospedero del \emph{Stator limbatus} que puede hospedarse en dos tipos de árbol: \emph{Acacia greggi y Pseudosamanea guachapele}, tiene consecuencias en su desarrollo. Las particularidades de estos árboles es que el \emph{Acacia greggi} tiene unas semillas más grandes que el \emph{Pseudosamanea Guachapele}. Se analizó cómo varía el tamaño de los insectos cuando el hospedero es un árbol u otro. El resultado experimental mostró que cuando este insecto se hospeda en el árbol con las semillas más grandes, los organismos son de mayor tamaño. Este mayor tamaño es independiente al tamaño de los progenitores. Según las autoras del artículo esto indica plasticidad fenotípica. El diseño experimental se encarga de mantener a la población fija, es decir, sortea de manera aleatoria en qué hospedero pondrán a qué organismos. Al hacer esto, controlamos por el tamaño de los organismos. Si después se observa que a pesar de ello las nuevas generaciones de organismos son más grandes en el árbol con las semillas más grandes, entonces podemos afirmar que hay una relación causal en términos de la metodología de Woodward.

Este ejemplo es evidencia de que se ha estado trabajando en la tesis que afirma que el medio ambiente es un factor que hace la diferencia en los fenotipos. No sólo hay evidencia a favor de esto, sino que además hay evidencia que estas relaciones entre medio ambiente e individuos son un factor relevante para la evolución por selección natural \cite{Jablonka2020, Dayan2020, MacColl2011}. Esto no es lo único importante, si queremos ofrecer explicaciones causales de otros fenómenos biológicos como Eco-Evo-Devo \cite{PfenningEco-Evo-Devo}, Plasticidad fenotípica \cite{WESTEBERHARD20082701}, CGV \cite{CVG}, entonces hay que hablar de causas próximas en biología evolutiva.

Estos problemas con la distinción que hizo Mayr dan entrada a que podamos hablar de relaciones causales en las explicaciones evolutivas. Por lo que nos atrevemos a afirmar que las causas próxima sí están presentes en las explicaciones evolutivas por selección natural. Esta injerencia causal ha hecho que haya un interés por parte de los biólogos para entrar al debate acerca de la causalidad. Este interés es más marcado cuando se desarrollan explicaciones que intentan incorporar una variable ambiental en el desarrollo de los organismos.

Esto además casa bien con la metodología ofrecida por Woodward que expusimos en el capítulo 1. En términos de lo que afirma esta teoría, hay una relación causal entre el medio ambiente y los organismos que lo habitan. Más aún, hay intentos de exponer que un enfoque manipulabilista puede ser de utilidad en las explicaciones por selección natural \cite{MacColl2011}. Dado que podemos hablar de causas próximas, nos dedicamos en la siguiente sección a hablar en particular de cómo lo dicho hasta ahora está relacionado con el debate acerca del \emph{fitness}

\section{\emph{Fitness}, un concepto causal}

\noindent Como dijimos anteriormente el \emph{fitness} es un concepto central de la teoría de la evolución por selección natural. En la esquematización que dimos anteriormente, la tercera línea captura el hecho de que los organismos más aptos son aquellos que dejan más descendencia. En particular, decimos que el \emph{fitness} es medido en términos de descendencia y esto quiere decir que la selección natural opera en este organismo cuando, en ausencia de otros factores (por ejemplo, deriva génica) un organismo deja más descendencia que otro.

Habiendo mencionado que podemos incorporar causas próximas en nuestras explicaciones por selección natural, queda decir cómo el modelo de Woodward está relacionado con el debate sobre la interpretación de \emph{fitness}. Mostramos en la sección anterior cómo el modelo nos ayuda para extraer información causal a través de los diseños experimentales y la manera en que podemos intervenir en las variables y observar cómo se modifican otras variables. Al poder hacer esto, podemos definir al \emph{fitness} a través de estos modelos experimentales. Esto nos permite decir exactamente cuál es el organismo más apto dependiendo del medio ambiente en el que se desarrolla. Todo esto en consonancia con el modelo causal de Woodward. Esto tiene la ventaja de que tenemos una medida de fitness que nos permite explicar por qué unos organismos tienen ciertas características y si estas características son mejores para resolver problemas de ``diseño'' que otras características en la población. Esto nos deja con una definición de fitness no tautológica y explicativa. El valor explicativo se desprende de estar inserto en el modelo de Woodward. El modelo además nos permite que la definición sea causal. Además como defendimos en el capítulo 2 de este trabajo, no es necesario que causalidad implique determinismo. Por lo que parece que esta definición de \emph{fitness} es mejor que la propensionista, al mismo tiempo que permite involucrar un patrón probabilístico; además de ser más apta que la versión ecológica, al mismo tiempo que podemos involucrar un componente ambiental.

Aceptar esta definición de \emph{fitness} tiene las ventajas de dar un concepto más claro, independiente al número de descendencia que de hecho deja un organismo. Esta definición además nos permite explicar cuál es el factor relevante para la sobrevivencia de los organismos. Sin embargo, tiene la aparente desventaja de que no podemos predecir cómo serán en el futuro los organismos, es decir, qué tendencia evolutiva hay. Creemos que este problema es sólo aparente dada la asimetría que hay entre predecir y explicar en biología.

Tiene además la consecuencia de que la única manera en que podemos medir qué organismos son más aptos es cuando ya hay una cierta adaptación que podemos medir en un diseño experimental. Pensemos, por ejemplo, en el caso de \emph{A. sagrei} y \emph{L. carinatus}. En este diseño experimental se observa una tendencia al crecimiento de extremidades que les permitan escalar. Los organismos que mejor resuelven este problema y que, por ello, dejan más descendencia son aquellos con extremidades más largas. Aquí tenemos una explicación apelando a la selección natural. Pero es sólo en este diseño experimental y una vez observada esta tendencia que podemos usarla para explicar.

\section{\emph{fitness} y la interpretación dinámica de la Selección natural}

Como también mencionamos, hay un debate entre cuál es la mejor manera de interpretar a la teoría de la evolución por selección natural. Por un lado Bouchard y Rosenberg argumentan en favor de una interpretación dinámica (es decir que involucra fuerzas) \citeyear{Bouchard2004}; por otro lado Walsh, Lewens y Ariew \citeyear{Walsh2002} argumentan que la teoría no es sobre fuerzas, sino sobre consecuencias puramente estadísticas.

Walsh y compañía argumentan a favor de una interpretación estadística de la teoría de la selección natural. Pensemos, por ejemplo, en un grave al que dejamos caer de cierta altura. En este caso, podemos describir las fuerzas que hacen que caiga y podemos predecir el lugar en el que el grave de hecho va a caer. En el caso de las monedas, el hecho de que una $x$ cantidad de monedas caiga en cara y una $y$ cantidad caiga en cruz, no depende de las fuerzas actuando en cada moneda particular. De lo que depende el caso de las monedas es consecuencia de la estructura de la población.

En ambos casos hay dos tipos diferentes de error. El error en el caso de un grave que cae dependerá de que no tomamos en cuenta todas las fuerzas actuando para ser capaces de predecir el lugar de caída. En el caso de las monedas, el error es intrínseco a la probabilidad de las monedas. Debido a esta diferencie entre teorías dinámicas y teorías estocásticas, cabe la pregunta de cómo interpretamos a la teoría de la evolución.

El argumento de Walsh y compañía descansa en lo siguiente: si asumimos una interpretación dinámica de la selección natural, entonces seríamos capaces de distinguir entre selección debido a deriva génica y selección a través de selección natural. Pero no podemos distinguir entre ambas fuerzas. Por lo tanto, negamos la interpretación dinámica de la selección natural. Para la premisa de que no podemos hacer una distinción, los autores afirman que los eventos generalmente encapsulados bajo la rúbrica de "deriva génica" son indistinguibles de los eventos bajo la rúbrica "selección natural".

Hay algunas razones para rechazar la segunda premisa: que deriva génica y selección natural son indistinguibles. En primer lugar, la selección natural no es indiscriminatoria. Esto significa que cuando actúa selección natural en una población, esperamos que algunos de los organismos sean más aptos que otros en un medio ambiente. Cuando decimos que deriva génica actúa en una población es porque la población de alguna manera se ha desviado de lo que habíamos esperado fueran los organismos más aptos. Por decirlo de otra manera, cuando observamos una tendencia en nuestro diseño experimental y lo que observamos se desvía de lo que de hecho ocurre naturalmente, entonces apelamos a una explicación por deriva génica. Tal como argumenta Lange en \citeyear{Lange2013}, cuando pedimos una explicación en términos de selección natural buscamos una explicación causal. En cambio, cuando esta explicación causal falla debido a algún evento y la expectativa que teníamos del crecimiento de la población se ve afectada, entonces apelamos a una explicación por deriva génica. Esta segunda explicación es puramente estadística. Debido a que podemos hacer esta distinción, aceptamos la interpretación dinámica de la selección natural.

En la sección anterior dijimos por qué es ventajoso aceptar una interpretación causal de \emph{fitness}. Entre las razones estuvo que es un concepto de fitness que es explicativo, pero además nos permite distinguir entre deriva génica y selección natural. Bouchard y Rosenberg \cite{Bouchard2004} nos dicen que la definición tradicional de \emph{fitness} tiene tres problemas. El primer problema está relacionado con las frecuencias si fitness tuviera una interpretación puramente estadística, entonces decimos que ``a la larga'' el organismo más apto dejará más descendencia. Sin embargo, nos interesa que el tiempo sea finito, porque de otra manera no tendríamos acceso a dichas frecuencias. El segundo problema que tiene una definición de este tipo es que es una tautología. Se supone que los organismos más aptos, esto es, con más \emph{fitness} son aquellos que dejan más descendencia. Pero si definimos el grado de \emph{fitnes} de un organismo como aquél que deja más descendencia, entonces los organismos que dejan más descendencia son aquellos organismos que dejan más descendencia. Para resolver este problema, nosotros apelamos a una definición causales de \emph{fitness}. La solución de Bouchard y Rosenberg está en línea con la propuesta por nosotros. Para resolver el problema de la tautologicidad, Bouchard y Rosenberg nos dice que hay que leer causalmente la definición anterior: el hecho de que un organismo sea más apto causa que tenga más descendencia. Pero aquí aptitud o \emph{fitness} tiene que tener una definición diferente a ``dejar más descendencia''. El último problema es de naturaleza puramente biológica: no siemnpre el organismo que deja más descendencia es el organismo más apto.

Nos concentraremos en el segundo problema que es el que nos atañe. Para resolverlo, los autores apelan a una nueva definición de \emph{fitness}. La apuesta es que el concepto de \emph{fitness} es una comparación dos a dos entre individuos y su relación con el medio ambiente. En esta nueva estrategia para describir al \emph{fitness} encontramos al menos un problema. Los autores afirman que para medir qué tan apto es un organismo es, necesitamos ver qué problemas de \textbf{diseño} resuelve un organismo que otro organismo no resuelva.

Hay al menos dos conceptos poco claros en la definición anterior. La comparación dos a dos en una población puede volverse complicada tomando en cuenta el tamaño de la población. Para poblaciones grandes este proceso podría tardar un tiempo indefinido. Creemos que la definición propuesta por nosotros salva este problema al permitir explicar cuál es el organismo más apto recreando las presiones ambientales en nuestro diseño experimental.

El segundo concepto poco claro es el papel que juega la palabra diseño. Esto puede sugerir que la selección natural está guiada por un diseñador. Sin embargo, podemos apoyarnos en lo que dice Ayala. Ayala argumenta que uno de los grandes aportes de Darwin es haber descrito un mecanismo en el cual podemos hablar de diseño sin que haya un diseñador \cite{Ayala2004}. Si podemos hablar en estos términos, entonces hablar de diseño en los organismos nos permitirá, en la propuesta de Bouchard y Rosenberg, hacer una evaluación dos a dos de los organismos y como estos resuelven problemas impuestos por el ambiente. Esto nos da una pauta para hablar de que el medio ambiente es un factor causal en las explicaciones evolutivas, que como vimos anteriormente es algo que de hecho se está investigando. Esto en conjunción con lo que proponemos es un buen método para obtener información causal para sustentar hipótesis, nos permite hacer una distinción entre deriva génica y selección natural. Además, el método de Woodward para rastrear relaciones causales nos permite evitar la comparación dos a dos del concepto  ecológico que presentan Bouchard y Rosenberg. Este proceso queda cubierto por el diseño experimental en el que recreamos las condiciones ambientales que sospechamos hacen una diferencia en la supervivencia de los organismos. Es cuando los organismos resuelven estos problemas de diseño que apelamos a una explicación por selección natural. Cuando hacemos el conteo poblacional y no observamos lo que se esperaba, entonces apelamos a una explicación por deriva génica.

\section{Poblaciones o individuos}

\noindent Hasta este momento hemos apoyado la tesis de que el \emph{fitness} es causal en su naturaleza. En esto concuerdan los autores de ambos artículos: Bouchard y Rosenberg; Ariew, Lewens y Walsh. Sin embargo, se difiere en torno a si la selección natural es dinámica o sólo de naturaleza estadística. Para argumentar a favor de la naturaleza estadística, Ariew, Lewens y Walsh argumentan que la selección natural es indistinguible de la deriva génica. Dicen además que el comportamiento de la selección natural se parece a una tirada aleatoria de monedas.

En contra de estos autores, se argumentó que sí podemos distinguir a la deriva génica de la selección natural cuando asumimos una noción de \emph{fitness} centrada en cómo los organismos resuelven ciertos problemas de diseño. Dijimos además que la selección natural no es indiscriminada, sino que de hecho hay un cierto tipo de selección. Esta noción de \emph{fitness} nos permite separar al proceso evolutivo por selección natural del proceso evolutivo por deriva génica. Esto por supuesto apoya a la tesis de que la selección natural es de naturaleza dinámica. La deriva génica es un proceso más aleatorio.

Ahora bien, queda pendiente si esta interpretación dinámica ocurre a nivel de poblaciones o a nivel de los organismos. Millstein \citeyear{Millstein2006} argumenta que la selección natural es un proceso causal a nivel de poblaciones. El argumento que presenta Millstein a favor de que la selección natural es un proceso causal es que si sólo fuera un proceso estadístico, entonces no podríamos elegir entre hipótesis. La pura distribución estadística no nos permite distinguir entre si el proceso se debe a deriva génica o bien a selección natural. Para poder controlar las variables es necesario inducir un ambiente artificial en el laboratorio que nos permita decir cuál es la causa de que los organismos tengan cierto genotipo.

Todo lo anterior está en el tenor de lo que hemos argumentado hasta aquí. Que causalidad no es necesariamente determinista y que el modelo de Woodward es útil para elucidar el concepto de causa y que además casa con la metodología utilizada por los biólogos evolutivos. Si asumimos, como hemos hecho hasta aquí que el modelo de Woodward es útil para todo esto, entonces es compatible con lo que la misma Millstein acerca de la interpretación dinámica de la causalidad.

Sin embargo, una vez que argumentamos en favor de una interpretación dinámica de la selección natural, queda decir en qué nivel se da el proceso: si es a nivel de individuos o es a nivel de poblaciones. El argumento de Millstein comienza con definición de evolución por selección natural como el cambio en la frecuencia genética de una generación a otra.

Millstein procede entonces a separar dos tipos de defensas del individualismo: individualismo ingenuo e individualismo sofisticado. EL individualismo ingenuo nos dice que hay que rastrear toda la cadena causal de un organismo particular para saber si hay selección natural. Millstein nos dice que esto no nos ayudaría a distinguir hipótesis, porque no podemos discriminar entre si un individuo sobrevive por deriva génica o bien por selección natural. Pensemos en el siguiente ejemplo: un árbol sobrevive a un incendio forestal. Ahora pensemos en dos opciones: 1) con respecto a los demás árboles, el árbol que sobrevivió lo hizo por una característica heredable; 2) el árbol que sobrevivió lo hizo por cuestión de suerte. Si sólo siguiéramos la historia causal de ése árbol en particular no podríamos discriminar entre 1) y 2). Para poder hacer dicha discriminación es necesario hacer un sampleo de la población y una comparación de los genotipos para determinar entre 1) y 2).

Contra el individualismo sofisticado, su argumento es que la tesis termina reduciéndose a la tesis poblacional. En primer lugar, la tesis del individualismo sofisticado es la tesis de Bouchard y Rosenberg expuesta anteriormente: se hace una comparación dos a dos de individuos y se estima cómo algunos de los individuos resuelven problemas de diseño que otros no resuelven. Millstein argumenta que esta comparación dos a dos de individuos se tiene que hacer para absolutamente todos los de la población debido a la existencia de poblaciones no transitivas. Millstein dice que una vez que hacemos la jerarquía de cuáles individuos resuelven mejor ciertos problemas de diseño impuestos por el medio ambiente, entonces ya no estamos hablado de individuos, sino de poblaciones.

En el argumento de Millstein veo al menos dos problemas. El primero es que su noción de selección natural es demasiado estrecha dado lo que hemos expuesto en este trabajo: hay más formas de herencia que las estrictamente genéticas. Es por eso que la caracterización de \emph{fitness} hecha por Bouchard y Rosenberg es más adecuada al tener en cuenta presiones medioambientales y cómo los organismos pueden sobrellevar las fuerzas impuestas por el medio ambiente. Esto da pie a que podamos hablar de selección natural en términos más amplios.

Esto no es problemático para lo que dice Millstein, sino que sirve como un complemento a la definición que ofrece. Sin embargo, hay un problema en cómo Millstein presenta el argumento de Bouchard y Rosenberg. Millstein presenta la comparación dos a dos como si esta constituyera a selección natural, pero en el artículo de Bouchard y Rosenberg esta es una caracterización de \emph{fitness}. Si bien Bouchard y Rosenberg dicen que esto es suficiente para viciar una interpretación de selección natural hablando \textbf{sólo} sobre poblaciones, no dan más razones para dicha intención.

Pero entiendo que este \textbf{sólo}, no excluye la posibilidad de que podamos hablar de que selección natural actúa a nivel de poblaciones. Tal como dice la misma Millstein, una vez que hacemos la jerarquía de individuos y la resolución de problemas, ya no estamos hablando de individuos, sino de un conjunto de individuos. En esto último detecto sólo un problema. Como se argumentó en el primer capítulo, hay que hacer una diferencia entre los aspectos epistémicos y metafísicos de la causalidad. Millstein quiere extraer una conclusión metafísica: que selección natural es de naturaleza dinámica y poblacional, a partir de una premisa epistémica: que la única manera en la que podemos obtener información sobre si actúa selección natural es cuando hablamos de un sampleo. Sin embargo, esto cae en el error de reducir los métodos para obtener información causal, con la naturaleza misma de la causalidad.

Sobre esto queremos decir sólo dos cosas. La primera es que lo que dicen Millstein, Bouchard y Rosenberg no es incompatible. Esta comparación dos a dos y la parte que el ambiente juega en esta relación es importante para distinguir entre selección natural y deriva génica. Sin embargo, lo que dice Millstein es insuficiente para concluir que la selección natural es a nivel de poblaciones, lo más sensato dada la distinción entre aspectos epistémicos y ontológicos es que no sabemos en qué nivel actúa. Lo que sabemos es que causalidad nos ayuda a hacer una distinción entre selección natural y deriva génica, que el modelo de Woodward puede capturar estas relaciones cuando hacemos un sampleo y recreamos en el laboratorio las fuerzas en cuestión y que no hay problema en hablar de causalidad en biología evolutiva porque no estamos comprometidos con el determinismo causal.




\section{Reflexiones finales}

\noindent Llegado a este punto hemos expuesto razones para tomar al modelo de explicación de Woodward como una teoría de la explicación que puede ser aplicada a ciencias como la biología. En el primer capítulo hicimos una distinción entre los aspectos metafísicos y los aspectos epistemológicos de la causalidad y ofrecimos razones a favor de tomar a la teoría de Woodward como una buena alternativa para elucidar el concepto de explicación. Algunos de los valores del modelo de Woodward de hecho encajan con nuestras pretensiones explicativas en áreas que no son la física. Quizás el valor más relevante es el hecho de que no necesita leyes para explicar fenómenos y que puede dar cuenta de causas sin que haya determinismo (lo que es deseable dado el factor contingente en la evolución).

Sin embargo, aún hacía falta una defensa de por qué la causalidad no implica determinismo. Aún si la teoría de Woodward es una buena teoría de la explicación, ésta también pretende elucidar el concepto de lo que es una causa. Esto nos lleva directo a las teorías de la causalidad, en particular a los aspectos metafísicos de la causalidad. Por lo regular se divide a las teorías de la causalidad en humeanas y anti-humeanas. Siendo las anti-humeanas las que afirman la tesis de que hay conexiones necesarias en la naturaleza. En la segunda parte de este trabajo argumentamos en favor de la tesis de que la causalidad no implica conexiones necesarias, dando apoyo a las tesis humeanas de la causalidad.

Para defender esto, comenzamos notando que las teorías anti-humeanas de la causalidad toman a las leyes naturales como evidencia de que hay conexiones necesarias. Nos concentramos en presentar una alternativa argumentando que la necesidad en la naturaleza es aparente. Esta necesidad aparente surge del hecho de que las llamadas leyes naturales son modelos particulares de un sistema formal, y por supuesto que una buena derivación formal hace necesaria su conclusión. Pero esta necesidad es una propiedad de la teoría formal, no de la naturaleza misma.

En este último capítulo quisimos integrar lo dicho hasta ahora. En primer lugar el modelo de Woodward es una teoría de la causalidad que no implica determinismo, por lo que nos parece que es útil para trabajar en biología dadas sus características particulares. Además refleja algo de la práctica experimental de los biólogos y da espacio a la contingencia. En segundo y último lugar las leyes no son necesarias ni para explicar, ni necesarias en términos metafísicos. Esto deja libre el camino para una teoría de la explicación en biología \cite{Brandon1997}.



%El autor de este texto es Oscar Abraham Olivetti Alvarez
