\chapter{Hacia una definición causal de \emph{fitness}}

\section{Introducción}

Me parece que llegado a este punto he expuesto las razones suficientes para tomar al modelo de  Woodward como una teoría de la explicación que puede ser aplicada a ciencias commo la biología. En el primer capítulo hice una disntinción entre los aspectos metafísicos y los aspectos epistemológicos de la causalidad.  Ofrecí razones a  favor de tomarla como una buena alternativa para elucidar el concepto de explicación. Algunos de los valores del modelo de Woodward de hecho encajan con nuestras pretensiones explicativas en áreas que no son la física. Quizás el valor más relevante es el hecho de que no necesita leyes para explicar fenómenos y que puede dar cuenta de causas sin que haya determinación (lo que es deseable dado el factor contingente en la evolución).

Aún si la teoría de Woodward es una buena teoría de la explicación, esta también pretende elucidar el concepto de lo que es una causa, lo que nos lleva directo a las teorías de la causalidad. Por lo regular se divide a las teorías de la causalidad en humeanas y no-humeanas. Siendo las no-humeanas las que afirman la tesis de que hay conexiones necesarias en la naturaleza. En la segudna parte de este trabajo argumenté en favor de por qué la teoría de Woodward es una buena teoría de la causalidad. Las teorías no-humeanas de la causalidad toman a las leyes naturales como evidencia de que hay conexiones necesarias. Me concentré en presentar una alternativa argumentando que la necesidad en la naturaleza es aparente. Esta necesidad aparente surge del hecho de que las llamadas leyes naturales son modelos particulares de un sistema formal, y por supuesto que una buena derivación formal hace necesaria su conclusión. Pero esta necesidad es una propiedad de la teoría formal, no de la naturaleza misma.

En este último capítulo quiero integrar ambas conclusiones. En primer lugar el modelo de Woodward es una teoría de la causalidad que no implica determinismo, por lo que me parece que es útil para trabajar en biología dadas sus características particulares. Además refleja algo de la práctica experimental de los biólogos y da espacio a la contingencia. En segundo y último lugar: el hecho de que las leyes no sean necesarias ni para explicar, ni en términos metafísicos, deja libre el camino para una teoría de la explicación en biología. Argumenté que la teoría de WOodward es útil para este propósito.

Ahora quisiera hacer una propuesta: que podemos definir causalmente al fitness. Hay un debate que se expone en tres artículos. Por un lado se defiende que el fitness es de naturaleza causal y a nivel de individuos \cite{Bouchard2004}, en otro se defiende que el fitness es simplemente una consecuencia estadística de las poblaciones de organismos \cite{Walsh2002}, por último, hay una defensa de que el fitness tiene naturaleza causaly es a nivel de poblaciones \cite{Millstein2006}. Yo quiero defender que hay algo empantanado en el debate y creo que lo dicho hasta ahora ofrecerá claridad en este problema.

Sin embargo, antes de continuar con el debate acerca del \emph{fitness}, quisiera motivar la tesis que afirma que hay explicaciones causales en biologia. A esto me dedico en el siguiente apartado.


\section{Explicaciones causales en biología}

\noindent La síntesis moderna en biología evolutiva fue un cambio enorme para la teoría darwinista. En pocas palabras, lo que se hizo durante este periodo fue acercar la genética mendeleiana y la teoría de la selección natural de Darwin. En su investigación, Darwin había descubierto que hay variación entre organismos de una misma población. Además, se podía observar que algunas de estas variaciones se mantenían entre los padres y la progenie y algunas de estas variaciones son ventajosas para los organismos. Estas variaciones ventajosas hacen que ciertos individuos tiendan a tener más descendencia como resultado.

El siguiente esquema está tomado de , la teoría de la selección natural sostiene que:

\begin{enumerate}
  \item Hay variación en las características morfológicas, fisiológicas y de comportamientoentre los miembros de una especie.
  \item La variación es en parte heredable, de manera que lo sindividuos se parecen más a sus padres que a otros miembros de la especie.
  \item Difrencias en la variación tiene como consecuencia diferencias en el número de descendientes, bien en generaciones inmediatas, o bien en generaciones más distantes\footnote{
  \begin{enumerate}
    \item There is variation in morphological, physiological, and behavioural traits among members of a species (the principle of variation)
    \item The variation is in part heritable, so that individuals resemble their relations more than they resemble other individuals and, in particular, offsprings resemble their parents (the principle of heredity)
    \item Different variants leave different numbers of offspring either in inmediate or remote generations (the principle of differential fitness)
  \end{enumerate}
  }\cite{Godfrey-Smith2013}.
\end{enumerate}

A pesar de que podemos observar que los organismos tienden a parecerse más a sus progenitores que a otros organismos dentro de la misma población, Darwin no dijo cuál era el mecanismos mediante el cuál se transmitían estas pequeñas ventajas. La síntesis moderna afirma que estas características se heredan principalmente mediante los genes, cerrando la brecha entre herencia y selección natural y afirmando que hay un mecanismo físico que es el que explica estos parentescos. Esto de entrada ya nos da una pauta para una explicación causal en biología. Los genes que codifican para ciertas proteínas son causalmente responsables. Podemos dar una explicación causal al nivel genético.

Pero esto no es suficiente para explicaciones evolutivas. No lo es porque es claro que las explicaciones genéticas sí pueden incorporar causas próximas, tal como afirmaba Mayr. Pero lo que aquí nos concierne son las explicaciones evolutivas, que es justo donde se encuentra el concepto de \emph{fitness}.

Ahora bien, la síntesis moderna fue bastante astringente en aceptar otro tipo de herencia que no fuera la herencia genética. Por lo que se excluyeron otros tipos de explicación que no fueran genéticos, o bien que no tuvieran una reducción a la genética.

A pesar de la astringencia de algunos investigadores, se han investigado sistemas de herencia distintos que no son herencia genética. Por ejemplo en \cite{Jablonka2020} las autoras argumentan que hay diferentes mecanismos de herencia que no necesariamente están al nivel genético. Entre ellos se encuentran los mecanismos epigenéticos y la imitación del comportamiento. Aún así, las autoras comentan que se ha relegado este tipo de mecanismos debido a que se argumenta que son triviales y no hacen una diferencia en términos evolutivos. Sin embargo, las autoras argumentan que hay evidencia de lo contrario.

Algunos de los ejemplos expuestos hasta este momento tienen este espíritu. Estos ejemplos no sólo pretenden ilustrar que la teoría de la causalidad de Woodward es comaptible con el quehacer del biólogo evolutivo, sino además ofrecer evidencia para lo que argumentan Jablonka y Lamb. Algunos de los ejemplos y las investigaciones en biologia sugieren que el gen no es la unidad principal sobre la que actúa la selección natural, sino que hay casos en los que el fenotipo es la unidad sobre la que actúa la selección natural y el código genṕetico es quien persigue.

Esto queda de relieve en que una parte de la investigación biológica actual pretende integrar factores del ambiente en sus explicaciones de fenómenos biológicos. Si el ambiente interactúa causalmente con los individuos y hay al menos un caso en el el que el gen no es la unidad pŕincip, entonces las causas próxima sí están presentes en las explicaciones evolutivas. Lo que lleva a pensar que la distincón hecha por Mayr deja de ser útil para explicar fenómenos evolutivos (esto ha sido explorado recientemente por \cite{Uller2020, Dayan2020, Laland2011}).

Los ejemplos sugieren que es verdad el antecendente en el condicional anterior: el ambiente interactúa causalmente con los individuos. Nos falta uno de los conyuntos: que haya al menos un caso en dónde el gen no es la unidad de selección. Jablonka y Lamb Argumentan a favor de ersta tesis y muestran ejemplos de herencia no genética. Pero pensemos en un  ejemplo más. El experimento realizado por Amarillo Suárez y Fox \citeyear{Amarillo-Suarez2006}.

Hay insectos que se desarrollan dentro de un hospedero. Se tiene evidencia que el hospedero en el que se desarrollan las crías tiene influencia en el tamaño de los insectos. En el artículo de Amarillo-Suárez y Fox, se explora cómo el hospedero del \emph{Stator limbatus} que puede hospedarse en dos tipos de árbol: \emph{Acacia greggi y Pseudosamanea guachapele}, tiene consecuencias en su desarrollo. Las particularidades de estos árboles es que el \emph{Acacia greggi} tiene unas semillas más grandes que el \emph{Pseudosamanea Guachapele}. Se analizó cómo varía el tamaño de los insectos cuando el hospedero es un árbol u otro. El resultado experimental mostró que cuando este insecto se hospeda en el árbol con las semillas más grtandes, los organismos son de mayor tamaño. Este mayor tamaño es independiente al tamaño de los progenitores. Según las autoras del artículo esto indica plasticidad fenotípica.

Estos ejemplos, tanto el anterior, como los referidos a lo largo del trabajo, indican que se ha estado trabajando en la tesis que afirma que el medio ambiente es un factor que hace la diferencia en los fenotipos. Esto sugiere que hay casos en donde los genes son los seguidores y los fenotipos son los líderes. No sólo hay evidencia a favor de esto, sino que además hay evidencia que estas relaciones entre medio ambiente e individuos son un factor reelevante para la evolución por selección natural \cite{Jablonka2020, Dayan2020, MacColl2011}. Esto no es lo único importante, si queremos ofrecer explicaciones causales de otros fenómenos biológicos como Eco-Evo-Devo \cite{PfenningEco-Evo-Devo}, Plasticidad fenotípica \cite{WESTEBERHARD20082701}, CGV \cite{CVG}, hay que hablar de causas próximas en biología evolutiva.

Estos problemas con la distinción que hizo Mayr dan entrada a que podamos hablar de relaciones causales en las explicaciones evolutivas. Esto ha hecho que haya un interés por parte de los biólogos para entrar al debate acerca de la causalidad. Este interés es más marcado cuandop se desarrollan explicaciones que intentan incorporar una variable ambiental en el desarrollo de los organismos.

Esto además casa bien con la metodología ofrecida por Woodward que expuse en el capítulo 1. En términos de lo que afirma esta teoría, hay una relación causal entre el medio ambiente y los organismos que lo habitan. Más aún, hay intentos de exponer que un enfoque manipulabilista puede ser de utilidad en las explicaciones por selección natural \cite{MacColl2011}. Estas afirmaciones: los problemas con la distinción de Mayr y la integración de aspectos ambientales que de hecho son factores relevantes para la evolución por selección natural, nos será de utilidad cuando discuta el artículo de Bouchard y Rosenberg.


\section{\emph{Fitness}}

El \emph{fitness} se define en términos de la descendencia que pueden dejar los organismos particulares. Esta propiedad depende de diferentes factores, pero por lo general decimos que son más aptos aquellos organismos con más probabilidad de sobrevivir, que son los que además pueden dejar mayor número sw descendencia. Esto de tener más probabilidad de sobrevivir depende de varios factores que no son obviamente calculables, pero que sabemos que es parte del proceso de selección natural el que los organismos más aptos \emph{i. e.} con más \emph{fitness} son los que sobreviven. Como además se entiende generalmente que la selección natural es la supervivencia del más apto, pone al \emph{fitness} como un elemento central de la teoría.

Sin embargo, esto no siempre resulta de la manera esperada. En particular, decimos que el fitness es medido en términos de descendencia y esto quiere decir que la selección natural opera en este organismo cuando, en ausencia de otros factores (por ejemplo, deriva génica) un organismo deja más descendencia que otro. Pero de entrada tenemos un problema si la tesis de selección natural afirma que el más apto es el que deja más descendencia, : fitness es circular. Esto no es particularmente algo que vaya a tratar. No es difícil asumir que claro que la tesis de la selección natural tiene estatus cognitivo como lo tienen otras teorías. Lo tiene porque nos permite explicar un montón de fenómenos. Sobre el cargo de vacuidad, es importante tratarlo.

Fitness es una propiedad relacional. Los relata son los individuos y el medio ambiente. No podemos medir el fitness de manera directa debido al número exagerado de variables que se relacionan entre el medio ambiente y el organismo. Está relación es de superviniencia, tal como la describe Kim. Sin embargo, podemos medirlo a través de sus efectos, es decir, el éxito reproductivo. Sin embargo, hay que etsbailizar para las diferentes variables. Supongamos que dos organismos idénticos comparten el mismo ambiente y ambos tienen la misma medida de _fitness_, supongamos ahora que uno de los dos es destruido. A partir de esto concluiremos que el organismo que no fue destruído tenía una medida de _fitnees_ mayor. Por eso hay que corregir por variables como Deriva Génica. Es por eso que _fitness_ está definido en términos d elo que sucederá a la larga, y medimos la diferencial de esto, para estimar el fitness de los organismos.

Pero podemos proceder de otra manera. Podemos ayudarnos de teorías externas a la propia selección natural para definir _fitness_ podemos, por ejemplo, apelar al diseño satisfactorio de un organismo para sobrevivir en el medio ambiente.

Si definimos el fitness como capacidad reproductiva, entonces este cocnepto no nos ayuda a explicar la proporción de reproducción de una población, y por tanto, no explica evolución. Rosenberg dice que podemos apelar a Fitness es un primitivo dentro de la teoría de la selección natural y q  ue puede ser definido por teorías externas. . "Biologists can appeal to optimal design for correcting comparative judgements in particular cases" (Alex Rosenberg Fitness). Rosenberg señala que el fitness no necesariamente está definido dentro de la teoría de la selección natural, sino que, podemos apelar a otros valores teóricos y a otras teorías para definir el fitness, y luego aplciarlo en la teoría de la selección natural. Esto, en principio, nos libera del problema de la vacuidad.

Los experimentos para controlar por variables de selección natural se diseñan en base a la historia evolutiva del organismo en ceustión.


If we take Goodman (and Mill) seriously we will have to add to our
characterization of laws. Laws are true universal generalizations that:
(1) have nomic or natural necessity (and so support counterfac-
tuals);
(2) are used essentially in scientific explanation; and
(3) receive confirmation from (a small number of) their positive
instances



No depende de que podamos eliminar la hipótesis de la deriva génica. Pero la comapración uno a uno nos ayuda a ditinguir el fitness incluso en casos en los que puede haber deriva génica.
