\chapter{Definir causalmente al fitness}

Me parece que llegado a este punto he expuesto las razones suficientes para tomar al modelo de  Woodward como una teoría de la explicación que puede ser aplicada a ciencias commo la biología. En el primer capítulo hice una disntinción entre los aspectos metafísicos y los aspectos epistemológicos de la causalidad.  Ofrecí razones a  favor de tomarla como una buena alternativa para elucidar el concepto de explicación. Algunos de los valores del modelo de Woodward de hecho encajan con nuestras pretensiones explicativas en áreas que no son la física. Quizás el valor más relevante es el hecho de que no necesita leyes para explicar fenómenos y que puede dar cuenta de causas sin que haya determinación (lo que es deseable dado el factor contingente en la evolución).

Aún si la teoría de Woodward es una buena teoría de la explicación, esta también pretende elucidar el concepto de lo que es una causa, lo que nos lleva directo a las teorías de la causalidad. Por lo regular se divide a las teorías de la causalidad en humeanas y no-humeanas. Siendo las no-humeanas las que afirman la tesis de que hay conexiones necesarias en la naturaleza. En la segudna parte de este trabajo argumenté en favor de por qué la teoría de Woodward es una buena teoría de la causalidad. Las teorías no-humeanas de la causalidad toman a las leyes naturales como evidencia de que hay conexiones necesarias. Me concentré en presentar una alternativa argumentando que la necesidad en la naturaleza es aparente. Esta necesidad aparente surge del hecho de que las llamadas leyes naturales son modelos particulares de un sistema formal, y por supuesto que una buena derivación formal hace necesaria su conclusión. Pero esta necesidad es una propiedad de la teoría formal, no de la naturaleza misma.

En este último capítulo quiero integrar ambas conclusiones. En primer lugar el modelo de Woodward es una teoría de la causalidad que no implica determinismo, por lo que me parece que es útil para trabajar en biología dadas sus características particulares. Además refleja algo de la práctica experimental de los biólogos y da espacio a la contingencia. En segundo y último lugar,  el hecho de que las leyes no sean necesarias ni para explicar, ni en términos metafísicos, nos da una pauta para las explicaciones en biología.

\section{Explicaciones causales en biología}

Algunos de los ejemplos expuestos hasta este momento pretenden ilustrar que la teoría de la causalidad de Woodward es comaptible con el quehacer del biólogo evolutivo. Mucha de la investrigación biológica actual pretende integrar factores del ambiente en sus explicaciones de fenómenos biológicos. El ambiente interactúa causalmente con los individuos, lo que es una causa próxima en términos de la distinción hecha por Mayr. Lo que lleva a pensar que la distincón hecha por Mayr deja de ser útil para explicar fenómenos evolutivos \cite{Uller2020, Dayan2020, Laland2011}.

Estos problemas con la distinción uqe hizo Mayr dan entrada a que podamos hablar de relaciones causales en las explicaciones evolutivas. Esto ha hecho que haya un interés por parte de los biólogos para entrar al debate acerca de la causalidad. Este interés es más marcado cuandop se desarrollan explicaciones que intentan incorporar una variable ambiental en el desarrollo de los organismos.

El ambiente interactúa causalmente con los organismos

El _fitness_ se define en términos de la descendencia que pueden dejar los organismos particulares. Esta propiedad depende de diferentes factores, pero por lo general aquellos organismos con más probabilidad de sobrevivir, son los que además pueden dejar descendencia. Esto de tener más probabilidad de sobrevivir depende de varios factores que no son obviamente calculables, pero que sabemos que es parte del proceso de selección natural el que los organismos más aptos _i. e._ con más _fitness_ son los que sobreviven. __Selección natural es la supervivencia del más apto__

Esto no siempre resulta de la manera esperada. En particular, decimos que el fitness es medido en términos de descendencia y esto quiere decir que la selección natural opera en este organismo cuando, en ausencia de otros factores (por ejemplo, deriva génica) un organismo deja más descendencia que otro.

Vacuidad  de la tesis de selección natural: fitness es circular. Esto no es particularmente algo que vaya a tratar. No es difícil asumir que claro que la tesis de la selección natural tiene estatus cognitivo como lo tienen otras teorías. Lo tiene porque nos permite explicar un montón de fenómenos. Sobre el cargo de vacuidad, es importante tratarlo.

Fitness es una propiedad relacional. Los relata son los individuos y el medio ambiente. No podemos medir el fitness de manera directa debido al número exagerado de variables que se relacionan entre el medio ambiente y el organismo. Está relación es de superviniencia, tal como la describe Kim. Sin embargo, podemos medirlo a través de sus efectos, es decir, el éxito reproductivo. Sin embargo, hay que etsbailizar para las diferentes variables. Supongamos que dos organismos idénticos comparten el mismo ambiente y ambos tienen la misma medida de _fitness_, supongamos ahora que uno de los dos es destruido. A partir de esto concluiremos que el organismo que no fue destruído tenía una medida de _fitnees_ mayor. Por eso hay que corregir por variables como Deriva Génica. Es por eso que _fitness_ está definido en términos d elo que sucederá a la larga, y medimos la diferencial de esto, para estimar el fitness de los organismos.

Pero podemos proceder de otra manera. Podemos ayudarnos de teorías externas a la propia selección natural para definir _fitness_ podemos, por ejemplo, apelar al diseño satisfactorio de un organismo para sobrevivir en el medio ambiente.

Si definimos el fitness como capacidad reproductiva, entonces este cocnepto no nos ayuda a explicar la proporción de reproducción de una población, y por tanto, no explica evolución. Rosenberg dice que podemos apelar a Fitness es un primitivo dentro de la teoría de la selección natural y q  ue puede ser definido por teorías externas. . "Biologists can appeal to optimal design for correcting comparative judgements in particular cases" (Alex Rosenberg Fitness). Rosenberg señala que el fitness no necesariamente está definido dentro de la teoría de la selección natural, sino que, podemos apelar a otros valores teóricos y a otras teorías para definir el fitness, y luego aplciarlo en la teoría de la selección natural. Esto, en principio, nos libera del problema de la vacuidad.

Los experimentos para controlar por variables de selección natural se diseñan en base a la historia evolutiva del organismo en ceustión.


If we take Goodman (and Mill) seriously we will have to add to our
characterization of laws. Laws are true universal generalizations that:
(1) have nomic or natural necessity (and so support counterfac-
tuals);
(2) are used essentially in scientific explanation; and
(3) receive confirmation from (a small number of) their positive
instances



No depende de que podamos eliminar la hipótesis de la deriva génica. Pero la comapración uno a uno nos ayuda a ditinguir el fitness incluso en casos en los que puede haber deriva génica.
