%El autor de este texto es Oscar Abraham Olivetti Alvarez

\chapter{Hacia una definición causal de \emph{fitness}}

\section{Qué ha pasado hasta ahora}

\noindent Llegado a este punto hemos expuesto razones para tomar al modelo de explicación de Woodward como una teoría de la explicación que puede ser aplicada a ciencias como la biología. En el primer capítulo hicimos una distinción entre los aspectos metafísicos y los aspectos epistemológicos de la causalidad y ofrecimos razones a favor de tomar a la teoría de Woodward como una buena alternativa para elucidar el concepto de explicación. Algunos de los valores del modelo de Woodward de hecho encajan con nuestras pretensiones explicativas en áreas que no son la física. Quizás el valor más relevante es el hecho de que no necesita leyes para explicar fenómenos y que puede dar cuenta de causas sin que haya determinismo (lo que es deseable dado el factor contingente en la evolución).

Sin embargo, aún hacía falta una defensa de por qué la causalidad no implica deternminismo. Aún si la teoría de Woodward es una buena teoría de la explicación, ésta también pretende elucidar el concepto de lo que es una causa. Esto nos lleva directo a las teorías de la causalidad, en particular a los aspectos metafísicos de la causalidad. Por lo regular se divide a las teorías de la causalidad en humeanas y anti-humeanas. Siendo las anti-humeanas las que afirman la tesis de que hay conexiones necesarias en la naturaleza. En la segunda parte de este trabajo argumentamos en favor de la tesis de que la causalidad no implica conexiones necesarias, dando apoyo a las tesis humeanas de la causalidad.

Para defender esto, comenzamos notando que las teorías anti-humeanas de la causalidad toman a las leyes naturales como evidencia de que hay conexiones necesarias. Nos concentramos en presentar una alternativa argumentando que la necesidad en la naturaleza es aparente. Esta necesidad aparente surge del hecho de que las llamadas leyes naturales son modelos particulares de un sistema formal, y por supuesto que una buena derivación formal hace necesaria su conclusión. Pero esta necesidad es una propiedad de la teoría formal, no de la naturaleza misma.

En este último capítulo queremos integrar lo dicho hasta ahora. En primer lugar el modelo de Woodward es una teoría de la causalidad que no implica determinismo, por lo que nos parece que es útil para trabajar en biología dadas sus características particulares. Además refleja algo de la práctica experimental de los biólogos y da espacio a la contingencia. En segundo y último lugar las leyes no son necesarias ni para explicar, ni necesarias en términos metafísicos. Esto deja libre el camino para una teoría de la explicación en biología \cite{Brandon1997}.

Ahora quisieramos hacer una propuesta: que podemos definir causalmente al \emph{fitness}. Hay un debate sobre cómo debemos interpretar a la teoría de la selección natural: como una teoría sobre fuerzas o una teoría estadística. Después de asumir alguna de las dos opciones, aún queda decir si es a nivel de individuos o a nivel de grupos. Este debate queda claramente expuesto en tres artículos. Por un lado se defiende que la selección natural es de naturaleza causal y a nivel de individuos \cite{Bouchard2004}, en otro se defiende que es simplemente una consecuencia estadística de las poblaciones de organismos \cite{Walsh2002}, por último, hay una defensa de que la selección natural es de naturaleza causal y es a nivel de poblaciones \cite{Millstein2006}. En este trabajo queremos defender que una vez definida la noción central de \emph{fitness} en términos causales, es más natural decir que la selección natural es de naturaleza dinámica, es decir, sobre fuerzas.

Sin embargo, antes de continuar con el debate acerca del \emph{fitness}, quisieramos motivar la tesis de la naturaleza dinámica exponiendfo tipo de explicaciones causales en biología. Me parece que hacer esto es necesario si es que queremos definir causalmente un término que está al centro de la teoría de la evolución por selección natural. Además, queremos presentar algunas consecuencias filosóficas que tiene involucrar relaciones causales en nuestras explicaciones evolutivas. Las vamos a presentar pero no las discutiremos a fondo, no lo hacemos porque estas consecuencias necesitan una extensión quizás incluso mayor que la de este trabajo. A esto dedicamos en el siguiente apartado.

\section{Explicaciones causales en biología}

\noindent La síntesis moderna en biología evolutiva fue un cambio enorme para la teoría darwinista. En pocas palabras, lo que se hizo durante este periodo fue acercar la genética mendeleiana y la teoría de la selección natural de Darwin. En su investigación, Darwin había descubierto que hay variación entre organismos de una misma población. Además, se podía observar que algunas de estas variaciones se mantenían entre los padres y su progenie. Más aún, algunas de estas variaciones son ventajosas para los organismos. Estas variaciones ventajosas hacen que ciertos individuos tiendan a tener más descendencia como resultado.

De manera esquemática la teoría de la selección natural sostiene que\footnote{1. There is variation in morphological, physiological, and behavioural traits among members of a species (the principle of variation), 2. The variation is in part heritable, so that individuals resemble their relations more than they resemble other individuals and, in particular, offsprings resemble their parents (the principle of heredity), 3. Different variants leave different numbers of offspring either in inmediate or remote generations (the principle of differential \emph{fitness})}:

\begin{enumerate}
  \item Hay variación en las características morfológicas, fisiológicas y de comportamiento entre los miembros de una especie.
  \item La variación es en parte heredable, de manera que lo individuos se parecen más a sus padres que a otros miembros de la especie.
  \item Diferencias en las variaciones tiene como consecuencia diferencias en el número de descendientes, bien en generaciones inmediatas, o bien en generaciones más distantes\cite{Godfrey-Smith2013}.
\end{enumerate}

A pesar de que podemos observar que los organismos tienden a parecerse más a sus progenitores que a otros organismos dentro de la misma población, Darwin no dijo cuál era el mecanismo de herencia mediante el cuál se transmitían estas ventajas de los padres a su progenie. La síntesis moderna afirma que estas características se heredan principalmente mediante los genes, cerrando la brecha entre herencia y selección natural y afirmando que hay un mecanismo físico que es el que explica estos parentescos. Esto de entrada ya nos da una pauta para una explicación causal en biología. Los genes que codifican para ciertas proteínas son causalmente responsables y podemos dar una explicación causal al nivel genético.

Pero esto no es suficiente para incorporar causalidad a las explicaciones evolutivas. No lo es porque es claro que las explicaciones genéticas sí pueden incorporar causas próximas, tal como afirmaba Mayr. Pero lo que aquí nos concierne son las explicaciones evolutivas, que es justo donde se encuentra el concepto de \emph{fitness}.

Ahora bien, la síntesis moderna fue bastante astringente en aceptar cualquier otro mecanismo de herencia que no fuera la herencia genética. Por lo que se excluyeron otros tipos de explicación y mecanismos no genéticos, o bien que no tuvieran una reducción a la genética. Ya desde hace varios años se ha comenzado a explorar si es necesario incorporar nuevos mecanismos de herencia. De ser esto correcto, estos nuevos mecanismos de herencia y su incorporación a la teoría tienen consecuencias importantes en qué evidencia debemos tomar para justificar hipótesis en selección natural. Además tiene consecuencias filosóficas importantes, entre ellas poner en tela de juicio la distinción de Mayr. Todo esto está estrechamente relacionado con lo que se ha denominado la Síntesis Extendida.

En este trabajo no vamos a discutir las consecuencias filosóficas y biológicas que tiene el integrar nuevos mecanismos de herencia. Este es un tema que por sí mismo requeriría un trato tan largo como este trabajo. Lo que vamos a hacer es ofrecer ejemplos que apoyen la hipótesis de que los mecanismos de herencia no son todos genéticos. Haremos esto porque para este trabajo es útil el integrar estos nuevos modos de herencia. Esto se debe a que gran parte de lo dicho en el debate en torno a la naturaleza del \emph{fitness} asume una dimensión ambiental en las explicaciones evolutivas via selección natural.

Una advertencia antes de continuar: el que debamos integrar una dimensión ambiental es aún un debate abierto. Por lo que en un futuro la investigación puede encontrar una manera de reducir los factores ambientales a factores genéticos, lo que haría que debamos excluir la dimensión ambiental de las explicaciones evolutivas. Esto sin duda es posible, pero la investigación actual en biología apunta en la otra dirección. Hay investigaciones que de hecho incorporan una dimensión ambiental y que afirman que hay otros medios de herencia no necesariamente genéticos. Los ejemplos que hemos expuesto a lo largo del trabajo están en consonancia con esta tesis. Menciono que es un debate aún abierto porque vamos a hacer una suposición bastante fuerte: que si queremos una definición de \emph{fitness} no circular, entonces hay que incorporar un factor ecológico/ambiental a la teoría. No creo que esto sea un salto demasiado grande debido a la cantidad de evidencia disponible.

A pesar de la astringencia de algunos biólogos, se han investigado sistemas de herencia distintos que no son herencia genética. Por ejemplo en \cite{Jablonka2020} las autoras argumentan que hay diferentes mecanismos de herencia que no necesariamente están al nivel genético. Entre ellos se encuentran los mecanismos epigenéticos y la imitación del comportamiento. La construcción de nicho también es un caso en el que puede haber herencia no genética. Aún con la evidencia experimental, las autoras comentan que se ha relegado este tipo de mecanismos con el argumento de que son triviales porque no hacen una diferencia en términos evolutivos. Sin embargo, las autoras argumentan que hay evidencia de lo contrario.

Algunos de los ejemplos expuestos hasta este momento tienen este espíritu. Estos ejemplos no sólo pretenden ilustrar que la teoría de la causalidad de Woodward es compatible con el quehacer del biólogo evolutivo, sino además ofrecer evidencia para lo que argumentan Jablonka y Lamb, a saber, que hay una importante ingerencia del medio en el que se desarrollan los organismos. Algunos de los ejemplos y las investigaciones en biología sugieren que el gen no es la unidad principal sobre la que actúa la selección natural, sino que hay casos en los que el fenotipo es la unidad sobre la que actúa la selección natural y el código genético es quien persigue estos cambios.

Esto se manifiesta en que una parte de la investigación biológica actual pretende integrar factores del ambiente en sus explicaciones de fenómenos biológicos. Si el ambiente interactúa causalmente con los individuos y hay al menos un caso en el el que el gen no es la unidad de herencia, entonces las causas próxima sí están presentes en las explicaciones evolutivas por selección natural. Por lo que podemos fijarnos en como el fenotipo interactúa con el ambiente y si hay características que pueden ser seleccionadas. Esto de nuevo lleva a pensar que la distinción hecha por Mayr deja de ser útil para explicar fenómenos evolutivos. Los límites de la distinción de Mayr han sido explorados recientemente en \cite{Uller2020, Dayan2020, Laland2011}.

Los ejemplos sugieren que es verdad el antecedente en el condicional anterior: el ambiente interactúa causalmente con los individuos. Nos falta uno de los conyuntos: que haya al menos un caso en dónde el gen no es la unidad de selección. Jablonka y Lamb Argumentan a favor de esta tesis y muestran ejemplos de herencia no genética. Pensemos en un ejemplo más. El experimento realizado por Amarillo Suárez y Fox \citeyear{Amarillo-Suarez2006}.

Hay insectos que se desarrollan dentro de un hospedero. Se tiene evidencia que el hospedero en el que se desarrollan las crías tiene influencia en el tamaño de los insectos. En el artículo de Amarillo-Suárez y Fox, se explora cómo el hospedero del \emph{Stator limbatus} que puede hospedarse en dos tipos de árbol: \emph{Acacia greggi y Pseudosamanea guachapele}, tiene consecuencias en su desarrollo. Las particularidades de estos árboles es que el \emph{Acacia greggi} tiene unas semillas más grandes que el \emph{Pseudosamanea Guachapele}. Se analizó cómo varía el tamaño de los insectos cuando el hospedero es un árbol u otro. El resultado experimental mostró que cuando este insecto se hospeda en el árbol con las semillas más grandes, los organismos son de mayor tamaño. Este mayor tamaño es independiente al tamaño de los progenitores. Según las autoras del artículo esto indica plasticidad fenotípica.

Este ejemplo es evidencia de que se ha estado trabajando en la tesis que afirma que el medio ambiente es un factor que hace la diferencia en los fenotipos. Esto sugiere que hay casos en donde los genes son los seguidores y los fenotipos son los líderes. No sólo hay evidencia a favor de esto, sino que además hay evidencia que estas relaciones entre medio ambiente e individuos son un factor relevante para la evolución por selección natural \cite{Jablonka2020, Dayan2020, MacColl2011}. Esto no es lo único importante, si queremos ofrecer explicaciones causales de otros fenómenos biológicos como Eco-Evo-Devo \cite{PfenningEco-Evo-Devo}, Plasticidad fenotípica \cite{WESTEBERHARD20082701}, CGV \cite{CVG}, entonces hay que hablar de causas próximas en biología evolutiva.

Estos problemas con la distinción que hizo Mayr dan entrada a que podamos hablar de relaciones causales en las explicaciones evolutivas. Por lo que nos atrevemos a afirmar que las causas próxima sí están presentes en las explicaciones evolutivas por selección natural. Esta ingerencias causal ha hecho que haya un interés por parte de los biólogos para entrar al debate acerca de la causalidad. Este interés es más marcado cuando se desarrollan explicaciones que intentan incorporar una variable ambiental en el desarrollo de los organismos.

Esto además casa bien con la metodología ofrecida por Woodward que expusimos en el capítulo 1. En términos de lo que afirma esta teoría, hay una relación causal entre el medio ambiente y los organismos que lo habitan. Más aún, hay intentos de exponer que un enfoque manipulabilista puede ser de utilidad en las explicaciones por selección natural \cite{MacColl2011}.

Lo dicho hasta ahora ofrece evidencia en favor de que los factores ambientales sí tienen ingerencia en la historia evolutiva de los organismos. Esto nos deja una brecha para poder hablar de relaciones causales no necesariamente genéticas y poder definir causalmente otro tipo de conceptos como el de \emph{fitness}. En la siguiente sección nos dedicamos a esto.

\section{\emph{Fitness} y la interpretación dinámica de la Selección natural}

\noindent El \emph{fitness} es un concepto central de la teoría de la evolución por selección natural. En la esquematización que dimos anteriormente, la tercera línea captura el hecho de que los organismos más aptos son aquellos que dejan más descendencia. En particular, decimos que el \emph{fitness} es medido en términos de descendencia y esto quiere decir que la selección natural opera en este organismo cuando, en ausencia de otros factores (por ejemplo, deriva génica) un organismo deja más descendencia que otro.

Sin embargo, hay un debate entre cuál es la mejor manera de interpretar a la teoría de la evolución por selección natural. Por un lado Bouchard y Rosenberg argumentan en favor de una interpretación dinámica (es decir que involucra fuerzas) \citeyear{Bouchard2004}; por otro lado Walsh, Lewens y Ariew \citeyear{Walsh2002} argumentan que la teoría no es sobre fuerzas, sino sobre consecuencias puramente estadísticas.

Walsh y compañía arghumentan a favor de una interpretación estadística de la teoría de la selección natural. Pensemos, por ejemplo, en un grave al que dejamos caer de cierta altura. En este caso, podemos describir las fuerzas que hacen que caiga y podemos predecir el lugar en el que el grave de hecho va a caer. En el caso de las monedas, el hecho de que una $x$ cantidad de monedas caiga en cara y una $y$ cantidad caiga en cruz, no depende de las fuerzas actuando en cada moneda particular. De lo que depende el caso de las monedas es consecuencia de la estructura de la población.

En ambos casos hay dos tipos dierentes de error. El error en el caso de un grave que cae dependerá de que no tomamos en cuenta todas las fuerzas actuando para ser capaces de predecir el lugar de caída. En el caso de las monedas, el error es intrínseco a la probabilidad de las monedas. Debido a esta diferencie entre teorías dinámicas y teorías estocásticas, cabe la pregunta de cómo interpreamos a la teoría de la evolución.

El argumento de Walsh y compañia descansa en lo siguiente: si asumimos una interpretación dinámica de la selección natural, entonces seríamos capaces de distinguir entre selección debido a deriva génica y selección a través de selección natural. Pero no podemos distinguir entre ambas fuerzas. Por lo tanto, negamos la interpretación dinámica de la selección natural. Para la premisa de que no podemos hacer unma distinción, los autores afirman que los eventos generalmente encapsulados bajo la rúbrica de "deriva génica" son indisitnguibles de los eventos bajo la rúbrica "selección natural".

Hay algunas razones para rechazar la segunda premisa: que deriva génica y selección natural son indistinguibles. En primer lugar, la selección natural no es indiscriminatoria. Esto significa que cuando actúa selección natural en una población, esperamos que algunos de los organismos sean más aptos que otros en un medio ambiente. Cuando decimos que deriva génica actúa en una población es porque la población de alguna manera se ha desviado de las predicciones. Tal como argumenta Lange en \citeyear{Lange2013}, cuando pedimos una explicación en términos de selección natural buscamos una explicación causal. En cambio, cuando esta explicación causal falla debido a algún evento y la expectativa que teníamos del crecimiento de la población se ve afectada, entonces apelamos a una explicación por deriva génica. Esta segunda explicación es púramente estadística. Debido a que podemos hacer esta distinción, aceptamos la interpretación dinámica de la selección natural.

Sin embargo, aún hay que decir algo sobre el concepto central a la teoría. Bouchard y Rosenberg \cite{Bouchard2004} nos dicen que la definición tradicional de \emph{fitness} tiene tres problemas. El primer problema está relacionado con las frecuencias: decimos que ``a la larga'' el organismo más apto dejará más descendencia. Sin embargo, nos interesa que el tiempo sea finito, porque de otra manera no tendríamos acceso a dichas frecuencias. El segundo problema que tiene una definición de este tipo es que es una tautología. Se supone que los organismos más aptos, esto es, con más \emph{fitness} son aquellos que dejan más descendencia. Pero si definimos el grado de \emph{fitnes} de un organismo como aquél que deja más descendencia, entonces los organismos que dejan más descendencia son aquellos organismos que dejan más descendencia. Para resolver el problema de la circularidad, Rosenberg nos dice que hay que leer causalmente la definición anterior: el hecho de que un organismo sea más apto causa que tenga más descendencia. Pero aquí aptitud o \emph{fitness} tiene que tener una definición diferente a ``dejar más descencia''. El último problema es de naturaleza puramente biológica: no siemnpre el organismo que deja más descendencia es el organismo más apto.

Nos concentraremos en el segundo problema que es el que nos atañe. PAra resolverlo, los autores apelan a una nueva definición de \emph{fitness}. La apuesta es que el concepto de \emph{fitness} es una comparación dos a dos entre individuos y su relación con el medio ambiente. En esta nueva estrategia para describir al \emph{fitness} encontramos al menos un problema. Los autores afirman que para medir qué tan apto es un organismo es, necesitamos ver qué problemas de \textbf{diseño} resuelve un organismo que otro organismo no resuelva.

La palabra diseño puede sugerir que la selección natural está guiada por un diseñador. Sin embargo, podemos apoyarnos en lo que dice Ayala. Ayala argumenta que uno de los grandes aportes de Darwin es haber descrito un mecanismo en el cual podemos hablar de diseño sin que haya un diseñador \cite{Ayala2004}. Si podemos hablar en estos términos, entonces hablar de diseño en los organismos nos permitirá hacer una evaluación dos a dos de los organismos y como estos resuelven problemas impuestos por el ambiente. Esto nos da una pauta para hablar de que el medio ambiente es un factor causal en las explicaciones evolutivas, que como vimos anteriormente es algo que de hecho se está investrigando. Es cuando los organismos resuelven estos problemas de diseño que apelamos a una explicación por selección natural. Cuando a pesar de la expectativa de qué organismo es más apto, cuando hacemos el conteo poblacional y no osbervamos lo que se esperaba, entonces apelamos a una explicación por deriva génica.

\section{Poblaciones o individuos}

Hasta este momento hemos apoyado la tesis de que el \emph{fitness} es causal en su naturaleza. En esto concuerdan los autores de ambos artículos: Bouchard y Rosenberg; Ariew, Lewens y Walsh. Sin embargo, se difiere en torno a si la selección natural es dinámica o sólo de naturaleza estadística. Para argumentar a favor de la naturaleza estadística, Ariew, Lewens y Walsh argumentan que la selección natural es indisitnguible de la deriva génica. Dicen además que el comportamiento de la selección natural se parece a una tirada aleatoria de monedas.

En contra de estos autoires, se argumentó que sí pódemnos distinguir a la deriva génica de la selección natural cuando asumimos una noción de \emph{fitness} centrada en cómo los organismos resuelven ciertos problemas de diseño. Dijimos además que la selección natural no es indiscriminada, sino que de hecho hay un cierto tipo de selección. Esta noción de \emph{fitness} nos permite separar al proceso evolutivo por selección natural del proceso evolutivo por deriva génica. Esto por supuesto apoya a la tesis de que la selección natural es de naturaleza dinámica. La deriva génica es un proceso más aleatorio.

Ahora bien, queda pendiente si esta interpretación dinámica ocurre a nivel de poblaciones o a nivel de los organismos. Millstein \citeyear{Millstein2006} argumenta que la selección natural es un proceso causal a nivel de poblaciones. El argumento que presenta Millstein a favor de que la selección natural es un proceso causal es que si sólo fuera un proceso estadístico, entonces no podríamos eleghir entre hipótesis. La pura distribución estadística no nos permite disntinguir entre si el proceso se debe a deriva génica o bien a selección natural. Para poder controlar las variables es necesario inducir un ambiente artificial en el laboratorio que nos pérmita decir cuál es la causa de que los organismos tengan cierto genotipo.

Todo lo anterior está en el tenor de lo que hemos argumentado hasta aquí. Que causalidad no es necesariamente determinista y que el modelo de Woodward es útil para elucidar el concepto de causa y que además casa con la metodología utilizada por los biólogos evolutivos. Si asumimos, como hemos hecho hasta aquí que el modelo de Woodward es útil para todo esto, entonces es compatible con lo que la misma Millstein acerca de ula intepretación dinámica de la causalidad.

Sin embargo, una vez que argumentamos en favor de una intepretación dinámica de la selección natural, queda decir en qué nivel se da el proceso: si es a nuivel de individuos o es a nivel de poblaciones. El argumento de Millstein comienza con definición de evolución por selección natural como el cambio en la frecuencia genética de una generación a otra.

Millstein procede entonces a seprar dos tipós de defensas del individualismo: individualismo inegnuo e individualismo sofisticado. EL individualismo ingenuo nos dice que hay que rastrear toda la cadena causal de un organismo particular para saber si hay selección natural. Millstein nos dice que esto no nos ayudaría a disntinguir hipótesis, porque no podemos discriminar entre si un indioviduo sobrevive por deriva génica o bien por selección natural. Pensemo en el siguiente ejemplo: un árbol sobrevive a un incendio forestal. Ahora pensemos en dos opciones: 1) con respecto a los demás árboles, el árbol que sobrevivió lo hizo por una característica heredable; 2) el árbol que sobrevivió lo hizo por cuestión de suerte. Si sólo siguiéramos la historia causal de ése árbol en particular no podríamo discriminar entre 1) y 2). Para pdoer hacer dicha discriminación es encesario hacer un sampleo de la población y una comparación de los genotipos para determinar entre 1) y 2).

Contra el individualismo sofisticado, su argumento es que la tesis termina reduciéndose a la tesis poblacional. En primer lugar, la tesis del individualismo sofisticado es la tesis de Bouchard y Rosenberg expuesta anteriormente: se hace una comparación dos a dos de individuos y se estima cómo algunos de los individuos resuelven problemas de diseño que otros no resuelven. Millstein argumenta que esta comparación dos a dos de individuos se tiene que hacer para absolutamente todos los de la población debido a la existencia de poblaciones no transitivas. Millstein dice que una vez que hacemos la jerarquía de cuáles individuos resuelven mejor cietos problemas de diseño impuestos por el medio ambiente, entonces





%El autor de este texto es Oscar Abraham Olivetti Alvarez
