%El autor de este texto es Oscar Abraham Olivetti Alvarez

\chapter{Hacia una definición causal de \emph{fitness}}

\section{Qué ha pasado hasta ahora}

\noindent Llegado a este punto he expuesto razones para tomar al modelo de explicación de Woodward como una teoría de la explicación que puede ser aplicada a ciencias como la biología. En el primer capítulo hice una distinción entre los aspectos metafísicos y los aspectos epistemológicos de la causalidad y ofrecí razones a  favor de tomar a la teoría de Woodward como una buena alternativa para elucidar el concepto de explicación. Algunos de los valores del modelo de Woodward de hecho encajan con nuestras pretensiones explicativas en áreas que no son la física. Quizás el valor más relevante es el hecho de que no necesita leyes para explicar fenómenos y que puede dar cuenta de causas sin que haya determinación (lo que es deseable dado el factor contingente en la evolución).

Aún si la teoría de Woodward es una buena teoría de la explicación, esta también pretende elucidar el concepto de lo que es una causa, lo que nos lleva directo a las teorías de la causalidad, en particular a los aspcectos metafísicos de la causalidad. Por lo regular se divide a las teorías de la causalidad en humeanas y no-humeanas. Siendo las no-humeanas las que afirman la tesis de que hay conexiones necesarias en la naturaleza. Siendo la de Woodward una teoría humeanista de la causadidad. En la segunda parte de este trabajo argumenté en favor de por qué la teoría de Woodward es una buena teoría de la causalidad. Las teorías no-humeanas de la causalidad toman a las leyes naturales como evidencia de que hay conexiones necesarias. Me concentré en presentar una alternativa argumentando que la necesidad en la naturaleza es aparente. Esta necesidad aparente surge del hecho de que las llamadas leyes naturales son modelos particulares de un sistema formal, y por supuesto que una buena derivación formal hace necesaria su conclusión. Pero esta necesidad es una propiedad de la teoría formal, no de la naturaleza misma.

En este último capítulo quiero integrar ambas conclusiones. En primer lugar el modelo de Woodward es una teoría de la causalidad que no implica determinismo, por lo que me parece que es útil para trabajar en biología dadas sus características particulares. Además refleja algo de la práctica experimental de los biólogos y da espacio a la contingencia. En segundo y último lugar las leyes no son necesarias ni para explicar, ni en términos metafísicos. Esto deja libre el camino para una teoría de la explicación en biología \cite{Brandon1997}.

Ahora quisiera hacer una propuesta: que podemos definir causalmente al \emph{fitness}. Hay un debate que se expone en tres artículos. Por un lado se defiende que el \emph{fitness} es de naturaleza causal y a nivel de individuos \cite{Bouchard2004}, en otro se defiende que el \emph{fitness} es simplemente una consecuencia estadística de las poblaciones de organismos \cite{Walsh2002}, por último, hay una defensa de que el \emph{fitness} tiene naturaleza causal y es a nivel de poblaciones \cite{Millstein2006}. Yo quiero defender que hay algo empantanado en el debate y creo que lo dicho hasta ahora ofrecerá claridad en este problema.

Sin embargo, antes de continuar con el debate acerca del \emph{fitness}, quisiera motivar la tesis que afirma que hay otro tipo de explicaciones causales en biología. Me parece que hacer esto es necesario si es que queremos definir causalmente un término que está al centro de la teoría de la evolución por selección natural. Además, quiero presentar algunas consecuencias filosóficas que tiene involucrar relaciones causales en nuestras explicaciones evolutivas. Las voy a presentar pero no las discutiré, no lo hago porque estas consecuencias necesitan una extensión quizás incluso mayor que la de este trasbajo. A esto me dedico en el siguiente apartado.

\section{Explicaciones causales en biología}

\noindent La síntesis moderna en biología evolutiva fue un cambio enorme para la teoría darwinista. En pocas palabras, lo que se hizo durante este periodo fue acercar la genética mendeleiana y la teoría de la selección natural de Darwin. En su investigación, Darwin había descubierto que hay variación entre organismos de una misma población. Además, se podía observar que algunas de estas variaciones se mantenían entre los padres y la progenie y algunas de estas variaciones son ventajosas para los organismos. Estas variaciones ventajosas hacen que ciertos individuos tiendan a tener más descendencia como resultado.

De manera esquemática la teoría de la selección natural sostiene que\footnote{1. There is variation in morphological, physiological, and behavioural traits among members of a species (the principle of variation) \\
2. The variation is in part heritable, so that individuals resemble their relations more than they resemble other individuals and, in particular, offsprings resemble their parents (the principle of heredity) \\
3. Different variants leave different numbers of offspring either in inmediate or remote generations (the principle of differential \emph{fitness})}:

\begin{enumerate}
  \item Hay variación en las características morfológicas, fisiológicas y de comportamiento entre los miembros de una especie.
  \item La variación es en parte heredable, de manera que lo individuos se parecen más a sus padres que a otros miembros de la especie.
  \item Diferencias en la variación tiene como consecuencia diferencias en el número de descendientes, bien en generaciones inmediatas, o bien en generaciones más distantes\cite{Godfrey-Smith2013}.
\end{enumerate}

A pesar de que podemos observar que los organismos tienden a parecerse más a sus progenitores que a otros organismos dentro de la misma población, Darwin no dijo cuál era el mecanismo de herencia mediante el cuál se transmitían estas ventajas de los padres a su progenie. La síntesis moderna afirma que estas características se heredan principalmente mediante los genes, cerrando la brecha entre herencia y selección natural y afirmando que hay un mecanismo físico que es el que explica estos parentescos. Esto de entrada ya nos da una pauta para una explicación causal en biología. Los genes que codifican para ciertas proteínas son causalmente responsables y podemos dar una explicación causal al nivel genético.

Pero esto no es suficiente para incorporar causalidad a las explicaciones evolutivas. No lo es porque es claro que las explicaciones genéticas sí pueden incorporar causas próximas, tal como afirmaba Mayr. Pero lo que aquí nos concierne son las explicaciones evolutivas, que es justo donde se encuentra el concepto de \emph{fitness}.

Ahora bien, la síntesis moderna fue bastante astringente en aceptar cualquier otro mecanismo de herencia que no fuera la herencia genética. Por lo que se excluyeron otros tipos de explicación y mecanismos no genéticos, o bien que no tuvieran una reducción a la genética. Ya desde hace varios años se ha comenzado a explorar si es necesario incorporar nuevos mecanismos de herencia. De ser esto correcto, estos nuevos mecanismos de herencia y su incorporación a la teoría tienen consecuencias importantes en qué evidencia debemos tomar para justificar hipótesis en selección natural. Además tiene consecuencias filosóficas importantes, entre ellas poner en tela de juicio la distinción de Mayr. Todo esto está estrechamente relacionado con lo que se ha denominado la Síntesis Extendida.

En este trabajo no voy a discutir las consecuencias filosóficas y biológicas que tiene el integrar nuevos mecanismos de herencia. Este es un tema que por sí mismo requeriría un trato tan largo como este trabajo. Pero lo voy a asumir, porque para este trabajo es útil el integrar estos nuevos modos de herencia. Esto se debe a que gran parte de lo dicho en el debate en torno a la naturaleza del \emph{fitness} asume una dimensión ambiental en las explicaciones evolutivas via selección natural.

Una advertencia antes de continuar: el que debamos integrar una dimensión ambiental es aún un debate abierto. Por lo que en un futuro puede que haya una manera de reducir los factores ambientales a factores genéticos, lo que haría que debamos excluir la dimensión ambiental de las explicaciones evolutivas. Esto sin duda es posible, pero la investigación en biología parece apuntar en la otra dirección. Hay investigaciones que de hecho incorporan una dimensión ambiental y que afirman que hay otros medios de herencia no necesariamente genéticos. Los ejemplos que he tomado a lo largo del trabajo están en consonancia con esta tesis. Menciono que es un debate aún abierto porque voy a hacer una suposición bastante fuerte: que si queremos una definición de \emph{fitness} no circular, entonces hay que incorporar un factor ecológico/ambiental a la teoría. No creo que esto sea un salto demasiado grande debido a la cantidad de evidencia.

A pesar de la astringencia de algunos investigadores, se han investigado sistemas de herencia distintos que no son herencia genética. Por ejemplo en \cite{Jablonka2020} las autoras argumentan que hay diferentes mecanismos de herencia que no necesariamente están al nivel genético. Entre ellos se encuentran los mecanismos epigenéticos y la imitación del comportamiento. La construcción de nicho también es un caso en el que puede haber herencia no genética. Aún con la evidencia experimental, las autoras comentan que se ha relegado este tipo de mecanismos con el argumento de que son triviales. Son triciales porque no hacen una diferencia en términos evolutivos. Sin embargo, las autoras argumentan que hay evidencia de lo contrario.

Algunos de los ejemplos expuestos hasta este momento tienen este espíritu. Estos ejemplos no sólo pretenden ilustrar que la teoría de la causalidad de Woodward es compatible con el quehacer del biólogo evolutivo, sino además ofrecer evidencia para lo que argumentan Jablonka y Lamb, a saber, que hay una importante ingerencia del medio en el que se desarrollan los organismos. Algunos de los ejemplos y las investigaciones en biología sugieren que el gen no es la unidad principal sobre la que actúa la selección natural, sino que hay casos en los que el fenotipo es la unidad sobre la que actúa la selección natural y el código genético es quien persigue.

Esto queda de relieve en que una parte de la investigación biológica actual pretende integrar factores del ambiente en sus explicaciones de fenómenos biológicos. Si el ambiente interactúa causalmente con los individuos y hay al menos un caso en el el que el gen no es la unidad de herencia, entonces las causas próxima sí están presentes en las explicaciones evolutivas. Por lo que podemos fijarnos en como el fenotipo interactúa con el ambiente y si hay características que pueden ser seleccionadas. Esto de nuevo lleva a pensar que la distinción hecha por Mayr deja de ser útil para explicar fenómenos evolutivos. Los límites de la distinción de Mayr han sido explorados recientemente en \cite{Uller2020, Dayan2020, Laland2011}.

Los ejemplos sugieren que es verdad el antecedente en el condicional anterior: el ambiente interactúa causalmente con los individuos. Nos falta uno de los conyuntos: que haya al menos un caso en dónde el gen no es la unidad de selección. Jablonka y Lamb Argumentan a favor de esta tesis y muestran ejemplos de herencia no genética. Pensemos en un ejemplo más. El experimento realizado por Amarillo Suárez y Fox \citeyear{Amarillo-Suarez2006}.

Hay insectos que se desarrollan dentro de un hospedero. Se tiene evidencia que el hospedero en el que se desarrollan las crías tiene influencia en el tamaño de los insectos. En el artículo de Amarillo-Suárez y Fox, se explora cómo el hospedero del \emph{Stator limbatus} que puede hospedarse en dos tipos de árbol: \emph{Acacia greggi y Pseudosamanea guachapele}, tiene consecuencias en su desarrollo. Las particularidades de estos árboles es que el \emph{Acacia greggi} tiene unas semillas más grandes que el \emph{Pseudosamanea Guachapele}. Se analizó cómo varía el tamaño de los insectos cuando el hospedero es un árbol u otro. El resultado experimental mostró que cuando este insecto se hospeda en el árbol con las semillas más grandes, los organismos son de mayor tamaño. Este mayor tamaño es independiente al tamaño de los progenitores. Según las autoras del artículo esto indica plasticidad fenotípica.

Estos ejemplos, tanto el anterior como los referidos a lo largo del trabajo, indican que se ha estado trabajando en la tesis que afirma que el medio ambiente es un factor que hace la diferencia en los fenotipos. Esto sugiere que hay casos en donde los genes son los seguidores y los fenotipos son los líderes. No sólo hay evidencia a favor de esto, sino que además hay evidencia que estas relaciones entre medio ambiente e individuos son un factor relevante para la evolución por selección natural \cite{Jablonka2020, Dayan2020, MacColl2011}. Esto no es lo único importante, si queremos ofrecer explicaciones causales de otros fenómenos biológicos como Eco-Evo-Devo \cite{PfenningEco-Evo-Devo}, Plasticidad fenotípica \cite{WESTEBERHARD20082701}, CGV \cite{CVG}, hay que hablar de causas próximas en biología evolutiva.

Estos problemas con la distinción que hizo Mayr dan entrada a que podamos hablar de relaciones causales en las explicaciones evolutivas, lo que ha hecho que haya un interés por parte de los biólogos para entrar al debate acerca de la causalidad. Este interés es más marcado cuando se desarrollan explicaciones que intentan incorporar una variable ambiental en el desarrollo de los organismos.

Esto además casa bien con la metodología ofrecida por Woodward que expuse en el capítulo 1. En términos de lo que afirma esta teoría, hay una relación causal entre el medio ambiente y los organismos que lo habitan. Más aún, hay intentos de exponer que un enfoque manipulabilista puede ser de utilidad en las explicaciones por selección natural \cite{MacColl2011}. Estas afirmaciones: los problemas con la distinción de Mayr y la integración de aspectos ambientales que de hecho son factores relevantes para la evolución por selección natural, nos será de utilidad cuando discuta el artículo de Bouchard y Rosenberg. Esto lo menciono porque según Rosenberg, el \emph{fitness} es una comparación dos a dos entre organismos y una relación entre el ambiente en el que viven dichos organismos. Rosenberg no es el único que defiende esta tesis, una defensa más la hace \cite{Glymour2011}.

\section{\emph{Fitness}}

\noindent pasemos al caso del \emph{fitness}. El \emph{fitness} por lo regular se define en términos de la descendencia que pueden dejar los organismos particulares. Esta propiedad depende de diferentes factores, pero por lo general decimos que son más aptos aquellos organismos con más probabilidad de sobrevivir, que son los que además pueden dejar mayor número de descendencia. Esto de tener más probabilidad de sobrevivir depende de varios factores que no son obviamente calculables, pero que sabemos que es parte del proceso de selección natural el que los organismos más aptos \emph{i. e.} con más \emph{fitness} son los que dejan más descendencia (un organismo muerto no puede dejar descendencia). Como además se entiende generalmente que la selección natural es la supervivencia del más apto, pone al \emph{fitness} como un elemento central de la teoría. En particular, decimos que el \emph{fitness} es medido en términos de descendencia y esto quiere decir que la selección natural opera en este organismo cuando, en ausencia de otros factores (por ejemplo, deriva génica) un organismo deja más descendencia que otro.

Bouchard y Rosenberg \cite{Bouchard2004} nos dicen que esta definición tiene tres problemas. El primer problema está relacionado con las frecuencias: decimos que "a la larga" el organismo más apto dejará más descendencia. Sin embargo, nos interesa que el tiempo sea finito, porque de otra manera no tendríamos acceso a dichas frecuencias. El segundo problema que tiene una definición de este tipo es que es una tautología. Entendemos que \emph{fitness} es dejar un mayor número de descendencia, entonces cualquier explicación sobre por qué tiene más descendencia un organismo que otro, apelará a \emph{fitness}, voilviendo circular al concepto. El último problema es de naturaleza puramente biológica: no siemnpre el organismo que deja más descendencia es el organismo más apto.

Para resolver estos problemas, los autores apelan a una definición de \emph{fitness} entendida como una propiedad relacional. Esto quiere decir que debemos hacer una comparación dos a dos entre los individuos y ver cómo se relacionan coin su medio ambiente. El primer conyunto no es problemático: asumimos que para que opere selección natural debe haber variaciones entre la población. Como se argumentó que podemos fijarnos en los fenotipos, entonces podemos comparar y ver qué variaciones fenotípicas hay en una población.

El segundo conyunto es más problemático. En eso encuentro al menos dos problemas. El primer problema es que no todos aceptarán que hay que fijarnos en las variaciones fenotípicas. El segundo problema es que los autores afirman que para medir qué tan apto es un organismo es ver qué problemas de \textbf{diseño} resuelve un organismo que otro organismo no resuelva.

Para resolver el primer problema, voy a apelar a la evidencia que ya mencioné y a la advertencia que sugerí más atrás. Esto es, que la investigación en biología de hecho está tratando de incorporar mecanismos de herencia no genéticos y esto nos permite centrar nuestra atención en los fenotipos y no en los genotipos. El segundo problema es que la palabra diseño puede sugerir que la selección natural está guiada por un diseñador y que no es contingente que los organismos tengas las características que de hecho tienen.

Este problema es menos interesante porque es un problema puramente terminológico. Me escudo en Dennett

\emph{fitness} es una propiedad relacional. Los relata son los individuos y el medio ambiente. No podemos medir el \emph{fitness} de manera directa debido al número exagerado de variables que se relacionan entre el medio ambiente y el organismo. Está relación es de superviniencia, tal como la describe Kim. Sin embargo, podemos medirlo a través de sus efectos, es decir, el éxito reproductivo. Sin embargo, hay que estabilizar para las diferentes variables. Supongamos que dos organismos idénticos comparten el mismo ambiente y ambos tienen la misma medida de \emph{fitness}, supongamos ahora que uno de los dos es destruido. A partir de esto concluiremos que el organismo que no fue destruido tenía una medida de \emph{fitness} mayor. Por eso hay que corregir por variables como Deriva Génica. Es por eso que \emph{fitness} está definido en términos d elo que sucederá a la larga, y medimos la diferencial de esto, para estimar el \emph{fitness} de los organismos.

Pero podemos proceder de otra manera. Podemos ayudarnos de teorías externas a la propia selección natural para definir \emph{fitness} podemos, por ejemplo, apelar al diseño satisfactorio de un organismo para sobrevivir en el medio ambiente.

Si definimos el \emph{fitness} como capacidad reproductiva, entonces este concepto no nos ayuda a explicar la proporción de reproducción de una población, y por tanto, no explica evolución. Rosenberg dice que podemos apelar a \emph{fitness} es un primitivo dentro de la teoría de la selección natural y q  que puede ser definido por teorías externas. . ``Biologists can appeal to optimal design for correcting comparative judgements in particular cases'' (Alex Rosenberg \emph{fitness}). Rosenberg señala que el \emph{fitness} no necesariamente está definido dentro de la teoría de la selección natural, sino que, podemos apelar a otros valores teóricos y a otras teorías para definir el \emph{fitness}, y luego aplciarlo en la teoría de la selección natural. Esto, en principio, nos libera del problema de la vacuidad.

Los experimentos para controlar por variables de selección natural se diseñan en base a la historia evolutiva del organismo en ceustión.


If we take Goodman (and Mill) seriously we will have to add to our ncharacterization of laws. Laws are true universal generalizations that

\begin{enumerate}
  \item have nomic or natural necessity (and so support counterfactuals
  \item are used essentially in scientific explanation
  \item receive confirmation from (a small number of) their positive instances
  \end{enumerate}

No depende de que podamos eliminar la hipótesis de la deriva génica. Pero la comparación uno a uno nos ayuda a distinguir el \emph{fitness} incluso en casos en los que puede haber deriva génica.



Hay una diferencia entre recabar evidencia sobre causalidad y qué es la causalidad en sí.

Para obtener evidencia, apelamos a datos estadísticos.

La teoría de la causalidad manipulabilista casa bien con el quehacer del biólogo.

El \emph{fitness} es

%El autor de este texto es Oscar Abraham Olivetti Alvarez
