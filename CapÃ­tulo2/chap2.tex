\chapter{?`Qué es una ley de la naturaleza?}

\noindent Las leyes parecen tener carga modal. De acuerdo con las leyes, las cosas \emph{deben} pasar de cierta manera y es en virtud de este \emph{deben} que tienen una carga modal. En su entrada, \cite{French2021} argumenta que las dos posiciones más famosas son la teoría humneana: que la modalidad de las leye ssuperviene del conjunto de eventos localmente conectados, y el disposicionalismo: que la modalidad está en los objetos y sus interacciones.

French menciona que niunguna de las dos posturas puede incorporar completamente a la práctica de los físicos. Después hace un caso a favor de una teoría estructuralista. En este capítulo quiero argumentar que es de las estructuras que se deriva la necesidad de las leyes. Me parece que esta tesis no es incompatible con la teoría de Woodward. Tomando a los contrafácticos como base, podemos esclarecer los conceptos de explicación y necesidad. De heco podemos extender el trabajo de Woodward para que trate casos donde no hay causalidad \cite{Woodward2018}. De acuerdo al argumento de este capítulo, podemos incorporar una explicación en términos de la teoría de Woodward preguntándonos que pasaría si tuviéramos otro conjunto de axiomas. Además, como mencioné anteriormente, no es claro que la biología tenga leyes, pero sí puede integrar una noción de explicación en términos de contrafácticos. Tomo lo anterior como una motivación para esta sección de mi trabajo

\section{Necesidad nómica}

\noindent Regularmente se toma como evidencia a favor de las tesis anti-humeanas sobre las leyes el hecho de que las leyes son enunciados que describen conexiones necesarias. En esta postura las leyes no son sólo descirpciones, sino que gobiernan cómo el mundo se comporta\cite{Bhogal2020}. La otra alternativa es el famoso mosaico humeano del que habla Lewis\footnote{El mosaico humeano se refiere a la configuracion espacio-temporal de eventos concretos: un evento sucede, y después otro evento sucede.}. Los humeanistas asumen que hay una relación de superviniencia entre el mosaico humeano y las leyes de la naturaleza. Esta relación de superviniencia, sin embargo, tiene problemas. Uno de ellos es que hay contraejemplos: casos donde las lyes cambian a pesar de que el mosaico humeano no. Bhogal sugirere que una manera en que el humeanista puede responder a estos contraejemplos es relajando la relación eentre el llamado mosaico humeano y las leyes de la naturaleza.

La teoría de Wodward es compatible con el humeanismo. Bhogal asume que las leyes de la naturaleza son una partte imnportante de la explicación científica. Como expuse en el primer capítulo de este trabajo, esto no necesariamente es el caso. Por lo que podemo llevar a cabo muchas de nuestras tareas epistémicas sin tener que recurrir a leyes. En este capítulo quiero argumentar que las leyes no supervienen del mosacio humeano, porque no hay leyes que gobiernen cómo se comporta el mundo. Al menos el anti-humeano acerca de las leyes asume que las leyes reflejan conexiones necesarias. Contra esta tesis me dedico en lo que sigue.

Me centraré pues en los criterios que da Nagel, para que algo cuente como una ley de la naturaleza. Supongamos que tenemos una ley $X$ y que los criterios de Nagel presentados en el capítulo anterior son correctos. Esto quiere decir que tenemos un universal irrestricto que es verdadero y cuyo vocabulario es puramente cualitativo. Buenos candidatos para $X$ podrían ser las leyes de Galileo. Si aceptamos que las leyes son universales irrestrictos, excluimos de inmediato el hecho de que la causalidad juegue un papel. No podríamos distinguir entre que siempre que se cumplen ciertas condiciones, otras se cumplen al mismo tiempo aunque sin relación causal; y entre leyes genuinamente causales. El no-humeanista acerca de las leyes tendría algo que decir al respecto.

En este capítulo quiero presentar un argumento modesto acerca de la naturaleza de las leyes. El argumento concluye en un tipo de antirrealismo acerca de las leyes de la naturaleza. Las premisas son que hay una distinción tajante entre lo que llamamos ``leyes'' y otro tipo de generalizaciones. Otra premisa es que lo que llamamos ``leyes'' son aplicaciones de un formalismo particular y es de este formalismo particular que hay una necesidad aparente. Por último, diré que lo que llamamos ``leyes'' son aplicaciones de este formalismo.

\section{El nacimiento del término}

\noindent El concepto de ``ley de la naturaleza'' ha estado presente en los debates filosóficos sobre la ciencia. Este debate está relacionado con discusiones acerca de la metafísica de la ciencia, la epistemología de la ciencia y los objetivos de la ciencia. Nos interesa, por un lado, una caracterización de qué \textbf{son} las leyes; por otro lado cómo nos sirven para explicar y cómo obtenemos evidencia de ellas (tanto explicación como los métodos de obtención de evidencia son marcadamente epistémicos); aunado a esto, también se suele afirmar que el objetivo de la ciencia es buscar las leyes de la naturaleza.

Con respecto al primer punto, Nagel nos ofrece una caracterización al respecto. El segundo punto está estrechamente relacionado con el modelo de explicación de Hempel, que padece de los problemas que revisamos en el primer capítulo. Con respecto al último factor, Giere \citeyear[p. 69]{Giere2006} menciona que las leyes de la naturaleza no ``[son] parte de literal del contenido de ninguna ciencia. [Las leyes de la naturaleza] son una noción que pertenece a un meta-nivel de interpretación acerca de lo que hacen los científicos''\footnote{[...] a law of nature is not part of the literal content of any science. It is a notion that belongs to a meta-level interpretation of what scientists do [...].}. Estas distinciones son importantes para dar claridad a la discusión acerca de las leyes. El argumento que expondré es marcadamente sobre la ontología de las leyes, es decir, de lo que las leyes \textit{son}.

Comenzando con el uso del término, el término ley de la naturaleza comenzó a ser utilizado a principios del siglo XVII. En su origen el término estuvo ligado a la teología y a la jurisprudencia \cite{Giere2006, Giere1999}. Evidencia de esta relación con la jurisprudencia la encontramos en la entrada de leyes de la naturaleza en la enciclopedia de d'Alambert y Diderot \cite{lawna}. Más aún, en la entrada se menciona que primero se llegó a pensar que las leyes naturales son aquellas impuestas por Dios para la buena conducta de los seres humanos. Si se actúa de acuerdo a los deseos de Dios, entonces las acciones son buenas; las acciones son moralmente deleznables en caso contrario. Además, Dios dota con conocimiento a los seres humanos. Los seres humanos al estar dotados de conocimiento, somos capaces de conocer esas leyes al examinar la naturaleza. Esto nos da evidencia del origen del término y su uso: las leyes de  la naturaleza son las leyes de Dios, y en esto recae la afirmación de que sean necesarias.

El ejemplo clásico de leyes de la naturaleza es el que Newton acuña, Pero este término está cargado de teología (algo que no es extraño porque Newton es religioso). Si el término de ley de la naturaleza está ligado a la teología, los defensores del anti-humeanismo tendrían algo que decir al respecto de esto: dudo que algún anti-humeanista acerca de las leyes quiera aceptar que las leyes de la naturaleza son las leyes de Dios. Hay al menos dos soluciones que se me ocurren. La primera es aceptar que ley de la naturaleza es de hecho un término teológico, o bien decir que, si bien el origen de las leyes es iluminador, no captura el significado de ``ley'' que nos interesa. Después de todo, los términos cambian y no se está utilizando el término ``leyes de la naturaleza'' en el mismo sentido en el lo entienden Hempel y Nagel. Siendo el de Hempel y Nagel el sentido importante de ley de la naturaleza.

En particular, yo estaría de acuerdo con esta línea de argumentación. El hecho de que el origen del término esté cargado de teología (y, por tanto, desearíamos excluirlo de la investigación científica actual), no implica que sea así necesariamente. Es decir, se puede secularizar el término. Si esto es correcto, entonces lo que resta es ver si existen enunciados que funcionen a la manera descrita por Hempel y Nagel. En la siguiente sección expondré argumentos en contra de que haya este tipo de enunciados.

\section{Leyes de la naturaleza: epistemología, metafísica y los objetivos de la ciencia}

\noindent Como revisamos en el capítulo anterior, ND indica que las leyes son un componente necesario para explicar un fenómeno. Sin embargo, somos capaces de ofrecer explicaciones que no necesitan ligarse una ley para contar como tales, por ejemplo, explicar que mi teclado se descompuso porque dejé caer agua sobre él. Por tanto, las leyes no son necesarias para la explicación.

Un ejemplo extra tomado de la biología. En \cite{Losos2004} se quiere investigar qué sucede con el proceso evolutivo de la especie \textit{Anolis Sagrei} cuando se introduce un depredador, en específico el \textit{Leiocephalus Carinatus}. Sucede que al introducir a este depredador, que habita principalmente en el suelo, el \textit{A. Sagrei} tiende a habitar lugares más altos y comienza a desarrollar sus extremidades para que le permita escalar. Esto responde a la pregunta ¿por qué \textit{A. Sagrei}, que suele ser una lagartija que habita en los suelos, comenzó a desarrollar extremidades que le permitan escalar? Cuya respuesta es ``porque se introdujo un depredador que principalmente habita en el suelo''. Esto claramente es una explicación. A pesar de ello, no hay leyes en su formulación y no parece plausible generar una ley a partir de este caso particular. Sin embargo, aún queda el hecho de que podrían ser suficientes para la explicación y, por tanto, favorecer a la postura anti-humeana. El argumento de esta sección descansa en que las leyes no capturan conexiones necesarias como lo requiere el anti-humeanista.

Para hacer una distinción entre leyes y generalizaciones accidentales, supongamos por ejemplo que tengo una leve obsesión con las monedas de \$1 y que en mi bolsillo siempre llevo al menos 5 monedas de \$1. En este caso, el enunciado ``todas las monedas del bolsillo de Abraham son de \$1'' sería verdadero, pero no lo consideramos una ley. Un caso que parece tener las características de ser una ley es el siguiente: cuando afirmamos que todos los trozos de cobre se dilatan al calentarse.  Con esta afirmación, queremos decir algo más que ``no hay un pedazo de cobre que no se dilate al calentarse''. Se quiere capturar cierto tipo de conexión entre el que algo sea de cobre y que se dilate al calentarse. Esta conexión, al menos en la literatura clásica, es necesitación: el hecho de que el material sea cobre, hace necesario que se dilate al calentarse. Este enunciado es distinto a mi ejemplo de las monedas. Los anti-humeanistas dirán que el enunciado sobre el cobre es una ley de la naturaleza, mientras que el enunciado sobre las monedas no lo es.

La postura de Nagel sobre las leyes, por ejemplo, asume que hay una conexión necesaria. De haber conexiones necesarias, podríamos justificar nuestro conocimiento de las causas a partir de los efectos. Es por ello que Nagel dice que lo que hace falta es una demostración de la necesidad de dicha conexión. Pero, ¿cómo podríamos hacer una prueba de dicha conexión? En primer lugar, según Nagel, las leyes son universales irrestrictos. Pero además tenemos la intuición de que las disciplinas científicas trabajan empíricamente. Estas dos afirmaciones están en tensión, parece ser que la sugerencia es que veamos y analicemos a todos y cada uno de los pedazos de cobre que se calientan. Pero esto es absurdo. Entonces al menos uno de los dos enunciados es falso. O bien no es verdad que las leyes son unviersales irrestrictos, o bien no es verdad que la ciencia trabaja empíricamente.

Nagel mismo menciona que no sería deseable que la ciencia proceda en términos de prueba necesaria a la manera en como lo hace la geometría \cite[cfr., p. 53]{Nagel2006} porque se perdería la intuición de que la ciencia es fundamentalmente empírica. Sin embargo, negar dicha conexión necesaria implicaría que la ciencia se ocupa sólo de verdades contingentes  y que las llamadas ``leyes'' que asumimos como verdaderas serían sólo leyes en apariencia. Esto nos lleva directo a otro dilema: o bien las leyes son necesarias y perdemos el componente empírico de la ciencia, o bien las leyes no son más que contingentes y las llamadas ``leyes de la naturaleza'' lo son sólo en apariencia.

Me parece que no es grave el hecho de que se trabaje con verdades contingentes y no necesarias. En lo expuesto en el capítulo anterior argumentamos en favor de que un buen modelo de explicación es el que nos presenta Woodward. Este modelo no necesita leyes a la manera tradicional, es decir, enunciados universales verdaderos y que además tengan contenido empíricos. Sin embargo, aún resta atacar la fuerte intuición de que las leyes son necesarias. Creo que esta aparente necesidad no es algo propio de las leyes, sino del método con el cual las construimos. Creo que Nagel tiene razón al afirmar que perderíamos el componente empírico de las leyes si fueran necesarias a la manera en que son las demostraciones geométricas. Pero además creo que la aparente necesidad es justo por la construcción geométrica de algunas de las leyes: en especial las de Newton y Galileo. Quiero argumentar que la tenbsión que menciona Nagel es aparente: que podemos tener leyes necesarias y que, sin embargo, podemoss tener un componente empírico. Quiero afirmar además que esta necesidad en las leyes no es las que necesita el anti-humeanista para sostener su postura. Esto hará que se mantenga la fuerte intuición de que als leyes sean necesarias, que haya evidencia en conmtra del anti-humeanista y que no perdamos el componente empírico de las leyes de la naturaleza.

\section{La necesidad en las leyes}

\noindent Empecemos por explorar la afirmación de Nagel de que el proceder científico perdería su parte empírica si procediera como lo hace la geometría. Cuando dicen que algo procede a la manera geométrica, por lo general se refieren al método que utiliza Euclides en los elementos \cite{Euclid2008}.

Cuando decimos que algo procede a la manera de Euclides, estamos sosteniendo la tesis de que cualquier consecuencia del sistema es derivable de las definiciones y las nociones comunes a las que se refiere Euclides. Todos los libros que conforman los elementos tienen la misma estructura. En particular, me interesa centrarme en el libro 5 que es donde Euclides expone la teoría de proporciones.

Detengámonos por un momento y pensemos a la teoría de proporciones como un modelo formal. Es decir que cualquier objeto con el que ``rellenemos'' la teoría\footnote{Utilizo la palabra rellenar para referirme específicamente al proceso en el que sustituimos las variables de las que habla Eucldies por constantes. En el caso de Galileo las constantes son distanci y tiempo, en el caso de Arquímedes las constantes son pesos y distancias}, es una consecuencia semántica de la teoría. Euclides comienza el libro 5 Dándonos un conjunto de definiciones y es a partir de estas definiciones que extrae una serie de teoremas.

La teoría de proporciones fue utilizada por Arquímedes y tiempo después por Galileo para derivar teoremas para su teoría ``rellenando'' los objetos de los cuales habla la teoría de proporciones con otro tipo de objetos. Si se puede derivar un teorema, entonces podemos afirmar que los objetos utilizados son magnitudes. Mi apuesta es que la teoría de proporciones es un modelo formal al que se puede rellenar con diferentes objetos y relaciones.

Tarski definió en términos formales qué es un modelo y cómo esto nos ayuda a aclarar el concepto de consecuencia lógica \cite{Tarski1956}. Tarski se preocupa por al menos tratar de recuperar dos nociones importantes en el concepto de consecuencia lógica: necesidad y forma. \cite{Torrente2000} Queremos que todos los esquemas con la misma forma nos lleven a las mismas consecuencias. Pero además, hay un sentido importante es que esto es así necesariamente: no es posible que las premisas sean verdaderas, tenga una forma válida y que la conclusión sea falsa.

Menciono esto de Tarski porque hay una analogía con la teoría de proporciones. Sabemos que de las definiciones que Euclides postula, se siguen los teoremas que deriva en el capítulo. Los modelos de esta teoría son cualquier teoría que pueda ser expresada en términos de la teoría de proporciones. Si esto es verdad, entonces no es un misterio de dónde proviene la intuición de que las leyes son necesarias, son necesarias porque dependen de un aparato formal. La parte empírica depende sólo de con qué constantes cambiemos las variables. Es decir que la naturaleza necesaria proviene de ser derivaciones, mientras que la parte empírica proviene de hacer las sustituciones adecuadas con objetos.

A continuación discuto dos ejemplos que muestran que la necesidad proviene directamente de la teoría de proporciones. Galileo en la tercera jornada prueba 6 teoremas acerca del movimiento uniforme y que demuestra que si un móvil con movimiento uniforme recorre dos espacios, esos espacios serán entre sí como las velocidades \cite[p. 215]{galtre}. Todo esto expresado en términos de la teoría de proporciones.

Galileo comienza la exposición de la tercera jornada dando una definición y 4 axiomas del movimiento uniforme. Galileo define al \textit{movimiento uniforme} como aquel movimiento que en los mismos periodos de tiempo recorre el mismo espacio. Es decir que si tenemos una línea dividida en segmentos iguales, durante cada uno de estos segmentos por los que pasa un móvil, el tiempo transcurrido es igual para todos los segmentos.

A parir de esta definición, Galileo nos presenta una serie de \textit{axiomas} que son consecuencias de la anterior definición. En estos axiomas, se dan una relación de desigualdad entre las magnitudes involucradas en el movimiento uniforme. El primer axioma señala que si dos objetos, digamos A y B tienen el mismo movimiento uniforme, entonces si A se desplaza durante más tiempo, la distancia recorrida también será mayor a la de B. Nos dice también que para el mismo movimiento uniforme, si el tiempo transcurrido es mayor, también lo será la distancia. Después de estos axiomas, Galileo se dispone a demostrar una serie de teoremas.

En el primer teorema Galileo demuestra que si un móvil con movimiento uniforme recorre dos espacios, esos espacios serán entre sí como las velocidades \cite[p. 215]{galtre}. Galileo nos pide que consideremos que un móvil recorre con velocidad constante dos espacios. Se nos pide que consideremos una distancia $A$ y un segmento $B$, de manera tal que $A = nB$. Además consideremos una distancia $C$ ($\neq A$) y un segmento $D$ ($\neq B$), de tal manera que $C = mD$. Ahora consideremos los tiempos correspondientes de $A$ y $C$. Digamos que $T[A]$ es el tiempo que tarda un móvil en recorrer la distancia $A$ y $T[A] = nB$. Además consideremos $T[C]$ que es el tiempo que el móvil tarda en recorrer la distancia $C$ y que $T[C] = mD$. Debido a que  $A = nB$ y $C = mD$. Por tanto, $A$ es equimúltiplo de $T[A]$, como $C$ es equimúltiplo de $T[C]$. Galileo menciona que debido a que como premisa el movimiento es uniforme, entonces durante cada intervalo $B$ y $D$, el tiempo es el mismo, entonces si $A > C$, $T[A] > T[D]$; si $A = D$, $T[A] = T[D]$; si $A < D$, $T[A] < T[D]$. Por tanto, según la definición 5 del libro V de Euclides \cite{Euclid2008} $A:D$ tiene la misma razón que $T[D]:T[A]$ y son proporcionales dada la definición 6 del mismo libro, que es lo que se quería probar. Pero no sólo eso, esto es un caso particular del primer teorema que demuestra Euclides en el libro 5 de los elementos.

El segundo ejemplo es lo que hace arquímnedes en el equilibrio de los cuerpos planos \cite{Archimedes1897}. Arquímedes utiliza la teoría de proporciones para probar una serie de teoremas que relacionan a los pesos y a las distancias. El libro comienza con una serie de postulados y las proposiciones 6 y 7 son instancias particulares del teorema 1 del libro 5 de Euclides. En los teoremas 6 y 7, Arquímedes demuestra que siendo $A:B$ = $DC:CE$, entonces su punto de equilibrio será recíprocamente proporcional a $A:B$.

No cabe duda que los resultados son necesarios al ser derivaciones que dependen de la teoría que Euclides desarrolla en el libro V de los elementos, en conjunción a lo que significa "seguirse de" como lo menciona Torrente. Al ser análogo a esquemas argumentales, tendríamos que decir o bien que alguna de las premisas es falsa, o bien que la derivación es incorrecta. Pero la prueba de Galileo no tiene pasos incorrectos. La otra opción nos lleva directo a un problema en filosofía de las matemáticas acerca de lo que constituyen un buen axioma, que es un problema que requiere un trato más extenso del que pudo ofrecer aquí. Lo único que quiero poner de relieve es que la necesidad de las leyes (al menos los ejemplos mostrados aquí) proviene de que son derivaciones de una teoría formal, a la que le hicimos un modelo, por tanto, no es algo intrínseco al mundo natural el que haya necesidad.

En el siguiente apartado me dedico a explorar el segundo cuerno del dilema que nos presentó Nagel e intentaré justificar la conclusión de que no es necesaria una teoría general que capture todas las características de lo que es una ``ley de la naturaleza'' para rescatar los criterios epistémicos que distinguen a la ciencia.

Nagel afirmaba que al hacer que las ciencias procedieran a la manera de pruebas geométricas, implicaba que se perdiera el componente empírico de la ciencia (que es una fuerte intuición acerca del proceder de la ciencia). Argumenté que los ejemplos de Galileo y de Arquímedes, aunados a la distinción trazada entre explicaciones matemáticas y otro tipo de explicaciones,  muestran ju esta necesidad es no es propia del mundo natural, sino de una teoría formal. Siendo los casos de Galileo y Arquímedes instancias que toman la teoría formal y la ``llenan'' para hacer un modelo. Es en la parte de rellenar la teoría que mantenemos su carácter empírico y es en la parte de la derivación que son necesarias. Esto quiere decir que no son necesarias porque haya algo en la naturaleza que implique conexiones necesarias. Por lo que el anti-humeanista carece de la evidencia principal para afirmar que causalidad implica conexión necesaria

Esto nos lleva directamente a afirmar el otro disyunto, esto es, el humeanismo acerca de la causalidad. Este punto de acción no es tan grave porque, como mencioné en el primer capítulo, no necesitamos tener generalizaciones universales para poder explicar. Según la teoría de Woodward es suficiente con tener generalizaciones que sean invariantes.

\section{El estatus de las leyes de la naturaleza}

\noindent Afirmar que las ``leyes de la naturaleza'' no son más necesarias no es una afirmación tan controversial como lo parece a primera vista. Esto no es terreno completamente inexplorado. Por ejemplo, Cartwright \citeyear{Cartwright1983} argumenta que las leyes de la física son literalmente falsas y que sólo se cumplen en casos muy concretos en los que se dan las condiciones adecuadas para que la ley ocurra tal como se describe. Fuera de estos casos, difícilmente veremos a la naturaleza comportarse como describen las leyes de la naturaleza. Cartwright asegura que es esto lo que le da la fuerza explicativa a la teoría. Cartwright asegura que la ley de la gravitación universal es falsa ya que siempre hay otras fuerzas actuando y que es sólo cuando se agrega la cláusula de ``siempre y cuando no haya otras fuerzas interactuando'' que dicha ley es verdadera. Pero al hacer que la ley sea verdadera perdemos poder explicativo.

Elgin presenta un punto semejante al de Cartwright. Elgin \citeyear{Elgin2004} señala que estamos en un dilema cuando hablamos de verdad en ciencia: o bien hacemos más laxos nuestros compromisos con la verdad, o bien aceptamos que la ciencia es deshonesta y epistémicamente deficiente. Elgin nos presenta varios casos en los que los científicos aceptan ``falsedades'' porque tienen virtudes cognitivas, en contraste con tener una descripción literalmente verdadera. En algunas ocasiones nos interesa interpretar datos o dar cuenta de fenómenos aún cuando no sea una descripción literalmente verdadera. Una manera de clarificar esto es el caso que nos presenta la autora sobre la ley de Snell. La ley de Snell nos dice que el ángulo de refracción de la luz, cuando un rayo de luz pasa de un medio a otro, es el ángulo de incidencia multiplicado por el índice de refracción del primer medio y esto es igual al ángulo de refracción multiplicado por el índice de refracción del segundo medio. Elgin señala que esta ley es útil, pero es falsa porque no se cumple para todos los casos, sino sólo para aquellos casos en que los medios de propagación son isotrópicos.

En este caso Elgin nos dice que si bien la ley es falsa, la utilizamos porque tiene virtudes cognitivas que perderíamos si buscáramos que la ley fuera verdadera (por ejemplo, acotando el cuantificador sólo a los casos isotrópicos), estas virtudes dependen de su falsedad y de qué tanto se desvía de una descripción literalmente verdadera de la realidad. Al darnos cuenta de dicha desviación, aprendemos algo sobre los medios por los que atraviesa la luz.

Menciono ambas posturas porque me interesa hacer un contraste con lo que nos dice Woodward. Ambas autoras aseguran que el hecho de que las leyes no sean necesarias es una ventaja. Sin embargo, me parece que pasar de la contingencia de las leyes a que la falsedad sea un valor que es útil en ciencia es un paso apresurado.

Detengámonos un momento en las afirmaciones de Cartwright y Elgin. Ambas autoras hacen de la verdad un objetivo secundario de la ciencia y que parecen hacer del ``poder explicativo'' el valor epistémico fundamental\footnote{Haría falta revisar exactamente a qué se refieren con poder explicativo. Si se refieren a que estos enunciados ayudan a explicar cómo suceden los fenómenos, entonces tendrían que describir correctamente el comportamiento de los fenómenos. Pero esto sólo lo podemos asegurar si la descripción es verdadera. Volviendo de nuevo al problema del que trataban de deshacerse.}. Pero dudo que cualquier persona que haya dedicado su vida a la investigación se refiera a su trabajo como ``aproximadamente verdadero'' o ``literalmente falso'', por lo que aún hay algo que explicar acerca de cómo la verdad juega un papel en la investigación científica.

\section{El papel de la verdad en la investigación}

\noindent En un artículo reciente \cite{Pritchard2019}, Pritchard argumenta que el paso de tener a la verdad como un valor epistémico secundario no resuelve los problemas que pretende. Pritchard explora la tesis de que la verdad es el valor epistémico por excelencia y desarrolla un argumento que trata de resolver los problemas que parece tener la afirmación ``la verdad es el único valor epistémico fundamental''.

Una razón para decir que la verdad no es fundamental, o bien no es el único valor importante, parece ser el hecho de que cualquiera que valore la verdad, valorará todas las verdades por igual. Por poner un ejemplo burdo, seguro hay una respuesta correcta a la pregunta ¿cuántos ladrillos hay en las paredes de mi casa?, pero sería ocioso perseguir esta respuesta. Debido a que hay una respuesta verdadera a la pregunta y debido a que no parece que sea valioso perseguirla, por tanto no tiene valor. Pero si no tiene valor en absoluto, entonces que algo sea verdad no es suficiente para la investigación. Por lo que es claro que la verdad sólo es una parte en la búsqueda del conocimiento y hay que tomar en cuenta otros valores\footnote{El problema de tratar a todas las verdades como igual es un problema al con el que Elgin intenta motivar su teoría, véase el primer capítulo de \cite{Elgin2017}}.

Sin embargo, este problema aparece cuando sólo tomamos en cuenta que es una proposición la que está en juego. Digamos que tengo dos respuestas correctas a dos preguntas diferentes, llamémoslas P y Q. Una de las preguntas es trivial y la otra es ``de más peso''. Si lo único que estuviera en juego fuesen P y Q, entonces tendríamos el problema mencionado antes. Pero no es claro por qué cualquiera que acepte que la verdad es un valor fundamental debería aceptar esto. Podría argumentarse que no importa sólo que una proposición sea verdadera, sino que las proposiciones vinculadas a ella también lo sean. Si el resultado permite obtener más verdades, entonces será un resultado más valioso. Pritchard apela a la teoría de virtudes epistémicas\footnote{Las virtudes epistémicas son capacidades de los agentes que se desarrollan con el hábito. Entre algunas podemos nombrar: que el agente sea observador, que preste atención a la evidencia, que tenga buena memoria, etc. Esto hace que las otras virtudes que probablemente juegan un papel como valores epistémicos, estén dependan del agente y no de valores como los que describen las tesis de Cartwright y Elgin.} y hace que este último punto dependa del agente y no de los valores que perseguimos en la búsqueda de conocimiento, lo que hace que podamos asumir que el valor epistémico fundamental es la verdad.

Este desarrollo que hace Pritchard podría ayudar a explicar cómo es que la verdad juega un papel en la investigación científica, independiente al hecho de que haya o no leyes de la naturaleza, que, como se dijo en la sección 1.3, probablemente no sea algo más que una etiqueta.

En este capítulo argumenté que la aparente necesidad de las leyes deriva del hecho de que son modelos de una teoría formal: la teoría de proporciones. Si todo esto es correcto, entonces el regularista nómico o alguna teoría no-humeanista de las leyes (aquella que afirma que hay necesidad en algunas conexiones naturales), está en serios problemas para dar cuenta de las conexiones causales. Aunado a esto, me parece que el modelo de Woodward es una buena alternativa para una teoría general de la explicación y una teoría general de la causalidad.

Lo que resta en este trabajo es una aplicación concreta de lo dicho hasta ahora. En el siguiente capítulo me proponga analizar el debate de la naturaleza del \emph{fitness}. Creo que lo dicho hasta ahora ofrecer claridad en torno al debate del \emph{fitness}.
