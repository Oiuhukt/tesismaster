\chapter{¿Qué es una ley de la naturaleza?}

\section{Introducción}

\noindentm Regularmente se toma como evidencia a favor del regularismo nómico: la alternativa no-humeana a la causalidad, el hecho de que las leyes son enunciados que describen conexiones necesarias. Estas conexiones causales son descritas por las leyes de la naturaleza. Lo que resta es investigar a las leyes de la naturaleza para darse cuenta de que hay causalidad y esta implica necesidad. Podemos enmarcar el problema de las leyes de la naturaleza de la sigiente manera: trazar una distinción entre leyes genuinas y generalizaciones accidentales.

Antes de proceder con lo que sigue, quiero hacer dos observaciones. La primera de ellas es que el modelo de Woodward serviría para describir estas ``leyes''. Lo haría porque es necesario que una ley se cumpla en algún caso, para que se cumpla en todos los caos. Las leyes de la naturaleza serían un caso límite de las generalizaciones relacionadas al cambio de Woodward.

La segunda observación tiene que ver con la caracterización de las leyes que Nagel propone. Supongamos que tenemos una ley $X$ y que los criterios de Nagel presentados en el capítulo anterior son correctos. Esto quiere decir que tenemos un universal irrestricto que es verdadero y cuyo vocabulario es puramente cualitativo. Buenos candidatos para $X$ podrían ser las leyes de Galileo. Si aceptamos que las leyes son universales irrestrictos, excluimos de inmediato el hecho de que la causalidad juegue un papel. No podríamos distinguir entre que siempre que se cumplen ciertas condiciones, otras se cumplen al mismo tiempo aunque sin relación causal; y entre leyes genuinamente causales. El no-humeanista acerca de las leyes tendría algo que decir al respecto.

En este capítulo quiero presentar un argumento modesto acerca de la naturaleza de las leyes. El argumento concluye en un tipo de antirrealismo acerca de las leyes de la naturaleza. Las premisas son que hay una distinción tajante entre lo que llamamos ``leyes'' y otro tipo de generalizaciones. Otra premisa es que lo que llamamos ``leyes'' son aplicaciones dde un formalismo particular y es de este formalismo particular que hay una necesidad aparente. Por último, diré que lo que llamamos ``leyes'' son aplicaciones de este formalismo.

\section{El nacimiento del término}

El concepto de ``ley de la naturaleza'' ha estado presente en los debates filosóficos sobre la ciencia. Este debate está relacionado con discusiones acerca de la metafísica de la ciencia, la epistemología de la ciencia y los objetivos de la ciencia. Nos interesa, por un lado, una caracterización de qué \textbf{son} las leyes; por otro lado cómo nos sirven para explicar y cómo obtenemos evidencia de ellas (tanto explicación como los métodos de obtención de evidencia son marcadamente epistémicos); pero también se suele afirmar que el objetivo de la ciencia es buscar las leyes de la naturaleza.

Con respecto al primer punto, Nagel nos ofrece una caracterización al respecto. El segundo punto está estrechamente relacionado con el modelo de explicación de Hempel, que padece de los problemas que revisamos en el primer capítulo. Con respecto al último factor, Giere \citeyear[p. 69]{Giere2006} menciona que las leyes de la naturaleza no ``[son] parte de literal del contenido de ninguna ciencia. [Las leyes de la naturaleza] son una noción que pertenece a un meta-nivel de interpretación acerca de lo que hacen los científicos''\footnote{[...] a law of nature is not part of the literal content of any science. It is a notion that belongs to a meta-level interpretation of what scientists do [...].}. Estas distinciones son importantes para dar claridad a la discusión acerca de las leyes. El argumento que expondré es marcadamente sobre la ontología de las leyes, es decir, de lo que las leyes \textit{son}.

Comenzando con el uso del término, el término ley de la naturaleza comenzó a ser utilizado a principios del siglo XVII. En su origen el término estuvo ligado a la teología y a la jurisprudencia \cite{Giere2006, Giere1999}. Evidencia de esta relación con la jurisprudencia la encontramos en la entrada de leyes de la naturaleza en la enciclopedia de d'Alambert y Diderot \cite{lawna}. Más aún, en la entrada se menciona que primero se llegó a pensar que las leyes naturales son aquellas impuestas por Dios para la buena conducta de los seres humanos. Si se actúa de acuerdo a los deseos de Dios, entonces las acciones son buenas; las acciones son moralmente deleznables en caso contrario. Pero además, al dotar de conocimiento a los seres humanos, podemos conocer esas leyes al examinar la naturaleza. Esto nos da evidencia del origen del término y su uso: las leyes de  la naturaleza son las leyes de Dios, y en esto recae la afirmación de que sean necesarias.

Si el ejemplo clásico de leyes de la naturaleza es el que Newton acuña, los defensores del anti-humeanismo tendrían algo que decir al respecto de esto. Una línea de argumentación es: si bien el origen de las leyes es iluminador, no captura el significado de ``ley'' que nos interesa. Después de todo, los términos cambian y no se está utilizando el término ``leyes de la naturaleza'' en el mismo sentido en el lo entienden Hempel y Nagel, y ese es el sentido importante de ley de la naturaleza.

En particular, yo estaría de acuerdo con esta línea de argumentación. El hecho de que el origen del término esté cargado de teología (y, por tanto, desearíamos excluírlo de la investigación científica actual), no implica que sea así necesariamente, es decir, se puede secularizar el término.

\section{Leyes de la naturaleza: epistemología, metafísica y los objetivos de la ciencia}

Como revisamos en el capítulo anterior, ND indica que las leyes son una característica necesaria para explicar un fenómeno. Sin embargo, somos capaces de ofrecer explicaciones que no necesitan ligarse una ley para contar como tales, por ejemplo, explicar que mi teclado se descompuso porque dejé caer agua sobre él. Por tanto, este criterio no es necesario.

Puede sugerirse que este criterio se restringe sólo a explicaciones científicas, pero aún así no es claro que todas las disciplinas científicas utilicen leyes en sus explicaciones, \textit{e.g.}, la biología. Creemos que sin duda la biología ofrece explicaciones de fenómenos naturales. Por ejemplo en \cite{Losos2004} se quiere investigar qué sucede con el proceso evolutivo de la especie \textit{Anolis Sagrei} cuando se introduce un depredador, en específico el \textit{Leiocephalus Carinatus}. Sucede que al introducir a este depredador, que habita principalmente en el suelo, el \textit{A. Sagrei} tiende a habitar lugares más altos y comienza a desarrollar sus extremidades para que le permita escalar. Esto responde a la pregunta ¿por qué \textit{A. Sagrei}, que suele ser una lagartija que habita en los suelos, comenzó a desarrollar extremidades que le permitan escalar? Cuya respuesta es ``porque se introdujo un depredador que principalmente habita en el suelo''. Esto claramente es una explicación. A pesar de ello, no hay leyes en su formulación y no parece plausible generar una ley a partir de este caso particular.

Sin embargo, hay una fuerte intuición de que la ciencia en general trabaja con leyes. Nagel, por ejemplo, dedica una parte de \cite{Nagel2006} a hacer explícito qué es una ley de la naturaleza. Nagel discute por qué estos criterios son adecuados, al motivar la distinción entre enunciados universales que son leyes, de aquellos enunciados universales que no lo son. Supongamos por ejemplo que tengo una leve obsesión con las monedas de \$1 y que en mi bolsillo siempre llevo al menos 5 monedas de \$1. En este caso, el enunciado ``todas las monedas del bolsillo de Abraham son de \$1'' sería verdadero, pero no lo consideramos una ley porque es una generalización accidental.

Sin embargo, al afirmar que todos los trozos de cobre se dilatan al calentarse, queremos decir algo más que ``no hay un pedazo de cobre que no se dilate al calentarse''. Se quiere capturar cierto tipo de conexión entre el que algo sea de cobre con que se dilate al calentarse. Esta conexión entre dos propiedades se ha tratado de esclarecer en términos lógicos, causales o de dependencia física.

Este tipo de conexión está relacionado con aspectos metafísicos de la ciencia. Algunos han tratado de capturar la conexión en términos causales, en términos de contrafácticos y algunos han sido escépticos acerca de esta conexión. En particular, Nagel discute en la siguiente sección la conexión lógica entre ambas propiedades. Hay quienes sostienen que esta conexión es de carácter necesario en tanto es suficiente saber el efecto para saber qué lo causó, asumiendo que tenemos una ley que describe este comportamiento. Sin embargo, esta postura asume que hay una conexión necesaria, porque de haber conexiónes necesarias, podríamos justificar nuestro conocimiento de las causas a partir de los efectos. Es por ello que Nagel dice que lo que hace falta es una demostración de la necesidad de dicha conexión.

Sin embargo, Nagel mismo autor menciona que no sería deseable que la ciencia proceda en términos de prueba necesaria a la manera en como lo hace la geometría \cite[cfr., p. 53]{Nagel2006} porque se perdería la intuición de que la ciencia es fundamentalmente empírica. Sin embargo, negar dicha conexión necesaria implicaría que la ciencia se ocupa sólo de verdades contingentes  y que las llamadas ``leyes'' que asumimos como verdaderas serían sólo leyes en apariencia. Por lo que tenemos un dilema: o bien las leyes son necesarias y perdemos el componente empírico de la ciencia, o bien las leyes no son más que contingentes y las llamadas ``leyes de la naturaleza'' lo son sólo en apariencia.

Creo que hay una manera de salir de este dilema: el antirealismo acerca de las leyes. Creo que Nagel tiene razón al afirmar que perderíamos el componente empírico de las leyes si fueran necesarias a la manera en que son las demostraciones geométricas. Además creo que es en esta afirmación donde está la solución y por qué la ncesidad involucrada es sólo aparente.

Hasta aquí he tratado de motivar que la discusión sobre las leyes está involucrada en muchos de los debates filosóficos sobre la ciencia. Esto particularmente es lo que hace difícil tratar con el problema de las leyes, ya que parece que la postura que tengamos con respecto de qué son las leyes de la naturaleza, nos compromete con una u otra postura acerca de los objetivos de la ciencia, las características epistémicas de la ciencia o con alguna postura metafísica de la ciencia. Por último presenté un dilema que Nagel encuentra con respecto a qué hacer en caso de que las leyes sean o no necesarias. En el siguiente apartado me propongo a analizar uno de los cuernos del dilema.

\section{La necesidad en las leyes}

\noindent Empecemos por explorar la afirmación de Nagel de que el proceder científico perdería su parte empírica si procediera como lo hace la geometría. Esta afirmación suena plausible, sin embargo, cabe recordar que el desarrollo científico dependió en gran parte de la teoría de proporciones. Puesto de manera burda, se investigaba si algo que se quisiera medir podía expresarse en términos de la teoría de proporciones y, por tanto, considerarse una magnitud.

Por ejemplo, Arquímedes y su trabajo acerca del equilibrio de los cuerpos planos donde nos dice cómo el peso es una magnitud en función de la proporción entre las distancias que hay desde el punto de equilibrio. Incluso Galileo en la tercera jornada prueba 6 teoremas acerca del movimiento uniforme y que demuestra que si un móvil con movimiento uniforme recorre dos espacios, esos espacios serán entre sí como las velocidades \cite[p. 215]{galtre}. Todo esto expresado en términos de la teoría de proporciones.

No cabe duda que los resultados son necesarios al ser derivaciones que dependen de la teoría que Euclides desarrolla en el libro V de los elementos\footnote{Podría objetarse que los resultado de Galileo no sólo dependen de la teoría de proporciones, sino también de un elemento externo, a saber, la definición de movimiento uniforme. Sin embargo, en este punto estoy de acuerdo con Duhem \cite{Duhem1976} en que difícilmente podríamos refutar una definición. Si llegamos a encontrar un fenómeno al que no se aplica la definición (por ejemplo un movimiento acelerado), lo único que se puede decir es que simplemente ese no es un movimiento uniforme, pero esto no se podría presentar como evidencia de que la definición de movimiento uniforme es incorrecta.}. Si quisiéramos negar el resultado al que llega Galileo, tendríamos que decir que o bien la teoría de proporciones, o bien algún paso de la prueba son incorrectos. Pero la prueba de Galileo no tiene pasos incorrectos. La otra opción sería afirmar que la teoría de proporciones lleva a malos resultados y que, por ello, no es una buena herramienta. Queda en manos de quien afirme esto mostrar que es una mala herramienta.

Aún más, Galileo se propone a mostrar que las adquisiciones de velocidad son iguales cuando la altura es igual. Para ello, Salviati \cite[p. 232]{galtre} nos dice que si fijamos un clavo en la pared [Punto A] y de este se amarra un grave que tiene un punto máximo [Punto B], y digamos que hay una línea horizontal perpendicular a la línea AB que Cruza desde D hasta C. Si levantamos el grave desde el punto B hasta el punto C, el \textit{momentum} adquirido desde c hasta B será suficiente para que el grave suba desde B hasta D, de esta manera describiendo un arco simétrico CBD. Si Hacemos que la distancia de A a B sea más corta [llamemos a esta nueva línea EB], y levantamos el grave hasta un punto en la línea perpendicular [punto F], el \textit{momentum} del grave hará que la curva descrita pase por FB y suba de nuevo hasta la línea perpendicular en un punto que esté a la misma distancia que F, pero al lado opuesto de la linea EB. De este experimento se concluye que los \textit{momenta} adquiridos son iguales si las alturas son iguales. Salviati menciona además que este experimento no dista mucho de ser una demostración necesaria, lo que sugiere que esimportante para Galielo proceder a la manera geométrica justificando los pasos en la teoría de proporciones.

Estos ejemplos muestran que en ciencia se pueda proceder a la manera geométrica. Este método se parte de ciertos enunciados y podemos llegar a resultados por medio de una herramienta que asegure que dichos resultados son correctos. Pero esto no implica que perdamos un componente empírico. No lo perdemos porque no sólo parece que a Galileo le importa proceder al modo geométrico, sino que busca que estos resultados se apliquen a problemas concretos. Aún más, estos resultados son generalizaciones de la recaudación de datos al experimentar con planos inclinados, como se sugiere en \cite[p. 231]{galtre}.

En el siguiente apartado me dedico a explorar el segundo cuerno del dilema que nos presentó Nagel e intentaré justificar la conclusión de que no es necesaria una teoría general que capture todas las características de lo que es una ``ley de la naturaleza'' para rescatar los criterios epistémicos que distinguen a la ciencia.

\section{La contingencia en las leyes}

\noindent En la sección pasada presenté lo que me parece un contraejemplo al condicional involucrado en el dilema que presenta Nagel. Nagel afirmaba que al hacer que las ciencias procedieran a la manera de pruebas geométricas, implicaba que se perdiera el componente empírico de la ciencia (que es una fuerte intuición acerca del proceder de la ciencia). Los ejemplos de Galileo muestran que a pesar de que la ciencia proceda a la manera de pruebas geométricas, no implica que se pierda el componente empírico de la misma.

Entonces parece que hay que afirmar el otro disyunto, esto es, que las leyes son contingentes. Si esto es correcto, entonces las llamadas ``leyes de la naturaleza'' son tales sólo en apariencia. Sin embargo, la conclusión del apartado anterior --que hay casos  en la que se procede a manera de una prueba y no se pierde el componente empírico-- entra en conflicto con el segundo cuerno del dilema. Supongamos que en efecto las leyes son contingentes, pero lo que Galileo hace es una prueba y esto es un caso en que hay una ley necesaria. Por lo que terminamos en el absurdo de que las leyes son necesarias y contingentes.

Sin embargo, esta contradicción es sólo aparente y surge de tratar de dar una teoría general acerca de las cualidades que definen lo que son las leyes de la naturaleza. Porque no es claro que haya un consenso acerca de a qué se le llama ley de la naturaleza. Dijimos que el término fue originalmente acuñado durante el siglo XVII. Pero como dice Cohen en la introducción a la \textit{Principia} \cite{principia} no había un consenso sobre a qué enunciado se le llama ``ley de la naturaleza''. Por ejemplo, Cohen menciona que Huygens llamó ``hipótesis'' a un enunciado que es similar a la primera ley de Newton. Y tampoco olvidemos que Newton llama ``axiomas o leyes'' a lo que ahora reconocemos como las leyes de Newton. Esto hace que en este caso suceda algo análogo a lo que sucede con el término ``axioma'': es utilizado por diferentes autores de distinta manera y no hay un consenso claro incluso en la época de NEwton donde comenzó a acuñarse el término \cite{heath2015}.

Entonces podemos considerar casos particulares en los que llamamos leyes a casos donde se procede a la manera geométrica y otros casos en los que sólo consideramos un proceso contingente y lo podemos llamar ley de la naturaleza. La contradicción surge cuando se busca una teoría general sobre las llamadas ``leyes de la naturaleza'' que incorpore todos los casos. Sin embargo, al no haber un claro consenso, quizás haya que abrazar la posibilidad de que el término no es más que una etiqueta.

\section{El estatus de las leyes de la naturaleza}

\noindent Afirmar que el término ``ley de la naturaleza'' no es más que una etiqueta no es una afrimación tan controversial como lo parece a primera vista. Esto no es terreno completamente inexplorado. Por ejemplo, Cartwright \citeyear{Cartwright1983} argumenta que las leyes de la física son literalmente falsas y que sólo se cumplen en casos muy concretos en los que se dan las condiciones adecuadas para que ocurra como se describe. Fuera de estos casos, difícilmente veremos a la naturaleza comportarse como describen las leyes de la naturaleza. Cartwright asegura que es esto lo que le da la fuerza explicativa a la teoría. Un ejemplo que utiliza es el de la ley de gravitación universal. Cartwright asegura que esta ley es falsa ya que siempre hay otras fuerzas actuando y que es sólo cuando se agrega la cláusula de ``siempre y cuando no haya otras fuerzas interactuando'' que dicha ley es verdadera. Pero al hacer que la ley sea verdadera perdemos poder explicativo.

Elgin presenta un punto semejante al de Cartwright. Elgin \citeyear{Elgin2004} señala que estamos en un dilema cuando hablamos de verdad en ciencia: o bien hacemos más laxos nuestros compromisos con la verdad, o bien aceptamos que la ciencia es deshonesta y epistémicamente deficiente. Elgin nos presenta varios casos en los que los científicos aceptan ``falsedades'' porque tienen virtudes cognitivas, en contraste con tener una descripción literalmente verdadera. En algunas ocasiones nos interesa interpretar datos o dar cuenta de fenómenos aún cuando no sea una descripción literalmente verdadera. Una manera de clarificar esto es el caso que nos presenta la autora sobre la ley de Snell. La ley de Snell nos dice que el ángulo de refracción de la luz, cuando un rayo de luz pasa de un medio a otro, es el ángulo de incidencia multiplicado por el índice de refracción del primer medio y esto es igual al ángulo de refracción multiplicado por el índice de refracción del segundo medio. Elgin señala que esta ley es útil, pero es falsa porque no se cumple para todos los casos, sino sólo para aquellos casos en que los medios de propagación son isotrópicos.

En este caso Elgin nos dice que si bien la ley es falsa, la utilizamos porque tiene virtudes cognitivas que perderíamos si buscáramos que la ley fuera verdadera (por ejemplo, acotando el cuantificador sólo a los casos isotrópicos), estas virtudes dependen de su falsedad y de qué tanto se desvía de una descripción literalmente verdadera de la realidad. Al darnos cuenta de dicha desviación, aprendemos algo sobre los medios por los que atraviesa la luz.

Pero si lo que dicen Cartwright y Elgin es correcto, entonces una teoría de la verdad correspondentista no es un buen candidato para una teoría de la verdad en ciencia. Podemos defender que una teoría de la verdad correspondentista es adecuada, sin embargo, los seres humanos somos cognitivamente deficientes y el error está en nuestras capacidades cognitivas. Haciendo que el problema no sea la correspondencia entre los enunciados y aquello que describe, sino que el problema está sólo a nivel de los enunciados que usamos. Debido a que nos valemos de la mejor explicación que hay disponible, el mundo está dispuesto a ser descubierto, pero los humanos aún no llegamos a una buena descripción. Es sólo en el futuro cuando tengamos una mejor teoría que ésta será una descripción literalmente verdadera. Sólo en ese caso, una teoría correspondentista saldrá ilesa de este embrollo.

Detengámonos un momento en las afirmaciones de Cartwright y Elgin. Ambas autoras hacen de la verdad un objetivo secundario de la ciencia y que parecn hacer del ``poder explicativo'' el valor epistémico fundamental\footnote{Haría falta revisar exactamente a qué se refieren con poder explicativo. Si se refieren a que estos enunciados ayudan a explicar cómo suceden los fenómenos, entonces tendrían que describir correctamente el comportamiento de los fenómenos. Pero esto sólo lo podemos asegurar si la descripción es verdadera. Volviendo de nuevo al problema del que trataban de deshacerse.}. Pero dudo que cualquier persona que haya dedicado su vida a la investigación se refiera a su trabajo como ``aproximadamente verdadero'' o ``literalmente falso'', por lo que aún hay algo que explicar acerca de cómo la verdad juega un papel en la investigación científica.

\section{El papel de la verdad en la investigación}

\noindent En un artículo reciente \cite{Pritchard2019}, Pritchard argumenta que el paso de tener a la verdad como un valor epistémico secundario no resuelve los problemas que pretende. Pritchard explora la tesis de que la verdad es el valor epistémico por excelencia y desarrolla un argumento que trata de resolver los problemas que parece tener la afirmación ``la verdad es el único valor epistémico fundamental''.

Una razón para decir que la verdad no es fundamental, o bien no es el único valor importante, parece ser el hecho de que cualquiera que valore la verdad, valorará todas las verdades por igual. Por poner un ejemplo burdo, seguro hay una respuesta correcta a la pregunta ¿cuántos ladrillos hay en las paredes de mi casa?, pero sería ocioso perseguir esta respuesta. Debido a que hay una respuesta verdadera a la pregunta y debido a que no parece que sea valioso perseguirla, por tanto no tiene valor. Pero si no tiene valor en absoluto, entonces que algo sea verdad no es suficiente para la investigación. Por lo que parece claro que la verdad sólo es una parte en la búsqueda del conocimiento y hay que tomar en cuenta otros valores\footnote{El problema de tratar a todas las verdades como igual es un problema al con el que Elgin intenta motivar su teoría, véase el primer capítulo de \cite{Elgin2017}}.

Sin embargo, este problema aparece cuando sólo tomamos en cuenta que es una proposición la que está en juego. Digamos que tengo dos respuestas correctas a dos preguntas diferentes, llamémoslas P y Q. Una de las preguntas es trivial y la otra es ``de más peso''. Si lo único que estuviera en juego fuesen P y Q, entonces tendríamos el problema mencionado antes. Pero no es claro por qué cualquiera que acepte que la verdad es un valor fundamental debería aceptar esto. Podría argumentarse que no importa sólo que una proposición sea verdadera, sino que las proposiciones vinculadas a ella también lo sean. Si el resultado permite obtener más verdades, entonces será un resultado más valioso. Pritchard apela a la teoría de virtudes epistémicas\footnote{Las virtudes epistémicas son capacidades de los agentes que se desarrollan con el hábito. Entre algunas podemos nombrar: que el agente sea observador, que presete atención a la evidencia, que tenga buena memoria, etc. Esto hace que las otras virtudes que probablemente juegan un papel como valores epistémicos, estén dependan del agente y no de valores como los que describen las tesis de Cartwright y Elgin.} y hace que este último punto dependa del agente y no de los valores que perseguimos en la búsqueda de conocimiento, lo que hace que podamos asumir que el valor epistémico fundamental es la verdad.

Este desarrollo que hace Pritchard podría ayudar a explicar cómo es que la verdad juega un papel en la investigación científica, independiente al hecho de que haya o no leyes de la naturaleza, que, como se dijo en la sección 1.3, probablemente no sea algo más que una etiqueta.



%\section{Conclusiones}

%\noindent En este trabajo argumenté que podemos deshacernos del debate acerca de las leyes de la naturaleza al echar un vistazo a cómo se desarrolló históricamente el término. Concluí que el término podría no ser más que una etiqueta. Sin embargo, el hecho de que el concepto de ``leyes de la naturaleza'' no sea más que una etiqueta, no casa con el hecho de que la investigación científica busca descripciones verdaderas acerca de fenómenos de la naturaleza. Exploré dos propuestas que niegan que el valor epistémico fundamental sea el de la verdad, pero argumenté que esto no es congruente con la práctica científica. Por último exploré una alternativa que presenta Pritchard que puede rescatar el hecho de que el valor fundamental en la búsqueda del conocimiento sea el de la verdad y que parece rescatar nuestras intuiciones acerca de la práctica y objetivos de la ciencia, sin ser demasiado restrictiva como para necesitar buscar una teoría de la verdad y una teoría de los valores epistémicos alternativas\footnote{Este último punto es quizás el más endeble del trabajo y necesita un mayor desarrollo, pero a primera vista es una alternativa prometedora para dar cuenta de la práctica científica y los objetivos de la ciencia}.
