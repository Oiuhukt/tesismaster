%El autor de este texto es Oscar Abraham Olivetti Alvarez

\chapter*{Introducción}


\noindent Nosotros queremos en esta tesis defender una noción eoclógica de la adecuación (también usaremos \emph{fitness } a lo largo del trabajo). Para hacer esto, primero queremos criticar la distinción de Mayr entre causas últimas y causas próximas. Para hacer esta crítica, repasamos algunos de los problemas que han atacado a esta distinción \cite{Laland2011, Uller2019}. Con esta crítica a la mano, nos disponemos a defender el modelo manipulabilista de Woodward \cite{Woodward2000, Woodward2003}

queremos utilizar el modelo manipulabilista de Woodward para aplciarlo  al modelo manipulabilista de Woodward como una alternativa a otros modelos de explicación. Nuestro objetivo es  luego queremos aplicar este modelo para defender una versión del fitness ecológico. Creemos que utilizando el modelo manipulabilistad e Woodward podemos resolver el problema de individuar los problemas de ``diseño''  que tiene la noción de \emph{fitness} ecológico \cite{Bouchard2004, sep-fitness} y que nos permite distinguir entre un proceso de evolución por selección natural de uno debido a la derivas génica. Esta solución la esbozamos en términos de la replicabilidad de un evento natural en un ambiente controlado de laboratorio. Para lorgar este objetivo, primero queremos defender que el modelo de Woodward es una buena alternativa para modelar explicaciones causales en biología. Luego, debido a que se asume que hay contingencia\footnote{Esta asunción no está libre de ser controvertida. En el artículo de Powell \citeyear{Powell2012}, el autor discute que hay algunos filósofos que toman el fenómeno de la evolución de caracteres análogos como evidencia a favor de que hay cierto grado de predicción en la teoría evolutiva.} involucrada en el proceso de selección \cite{Scriven1959, Mayr1998}, nosotros ofrecemos un argumento en contra del determinismo defendido por las autoras en \cite{Graves1999}. Por último queremos aplicar el modelo manipulabilista de Woodward al problema del \emph{fitness}.

Parte de la investigación científica se centra en buscar relaciones causales entre fenómenos y es una intuición común que usamos relaciones causales para explicar fenómenos. Cuando hablamos de \textit{inferencias causales} nos referimos a la justificación de dichas relaciones. Hablar de causalidad en biología evolutiva puede ser problemático. Este problema surge a partir de la distinción que Mayr \cite{Mayr1998} traza entre causas próximas y causas últimas. Las causas próximas apelan al medioambiente inmediato de los organismos, mientras que las causas últimas apelan a explicaciones por selección natural. Las causas próximas, nos dice Mayr, están fuera del ámbito de explicación del biólogo evolutivo.

Esta distinción trazada por Mayr, sin embargo, está lejos de ser controvertida. En \cite{Laland2011} los autores argumentan que esta distinción no es tajante y que las causas próximas sí que juegan un papel en la evolución de los organismos. Los autores mencionan que, por ejemplo, el fenómeno de la plasticidad fenotípica puede llevar a que un caracter sea propenso a ser seleccionado, y la plasticidad fenotípica es un fenómeno que se relaciona con el medioambiente próximo. Por tanto, la selección natural y las causas próximas no necesariamente son excluyentes.

Este proyecto toma como motivación el hecho de que las causas próximas sí están relacionadas con explicaciones evolutivas. Es por ello que creemos que podemos dar un modelo de explicación causal que funcione para modelar explicaciones causales en biología evolutiva. En este trabajo queremos hacer tres cosas. La primera es proponer al modelo manipulabilista de Woodward  \cite{Woodward2000, Woodward2003} como un modelo de explicación causal apropiado para trabajar explicaciones causales, en términos de causas proximas. El primer capítulo se encarga de esto. Es en el primer capítulo identificamos algunos problemas que tienen tanto el modelo Nomológico-deductivo, como el modelo de Relevancia estadística. Por último presentamos al modelo de Woodward como una alternativa a estos dos modelos de explicación. Hay que aclarar que el modelo de Woodward no está libre de problemas. Nosotros respondemos a algunos de los problemas del modelo de Woodward, tratando de hacerlo una alternativa viable para trabajar explicación causal en biología debido a que no necesita de leyes para su formulación y que resuelve el problema de contar como causales a correlaciones que tiene el modelo de relevancia estadística.

Sin duda en este trabajo falta la mención de dos alternativas que tienen como finalidad modelar la explicación. El primero es el modelo unificacionista que formularon Friedman \citeyear{Friedman1974} y Kitcher \cite{Kitcher2002}. la segunda gran mención es el modelo pragmático de van Fraassen. Dejamos de lado el modelo de Kitcher porque en su formulación las explicaciones causales no son necesarias. Dichjop de otra manera, Kitcher cree que es epistémicamente problemático el concepto de causa y que este puede ser reducido a los patrones de explicación. En el modelo de Kitcher las relaciones causales son dependientes de las explicaciones, en sus propias palabras ``Lo que emerge en el límite de este proceso es nada menos que la estructura causal del mundo''\footnote{What emerges in the limit of this process is nothing less than the causal structure of the world.}. Dejamos de lado el modelo de van Fraassen porque es compatible con el modelo de explicación manipulabilista de Woodward, en donde la relación de reelevancia es la de estar causalmente relacionados.

Continuando con la presentación, lo segundo que buscamos hacer en esta tesis es dar un argumento a favor del indeterminismo causal. A Mayr le preocupaba que las causas próximas estuvieran relacionadas con las explicaciones evolutivas, porque podía haber determinismo involucrado. Mayr \cite{Mayr1998} en un punto señala que el biólogo funcional se ocupa de un sólo individuo y sus relaciones medioambientales, además menciona que el biólogo funcional ``[...] si aisla suficientemente el fenómeno estudiado de las complejidades del organismo, puede alcanzar el ideal de un experimento en física o en química'' (p. 83). Pero si es verdad que la línea trazada entre biología funcional y biología evolutiva no es tajante como señalan Laland y compañía, entonces puede haber determinismo involucrado.

Pero no quisiéramos que el determinismo se cuele en los fenómenos biológicos. Este compromiso está motivado por la asimetría entre explicar y predecir que hay en biología evolutiva. Esta asimetría, argumenta Scriven \citeyear{Scriven1959}, nos dice que si bien podemos explicar un fenómeno en biología evolutiva, con la misma información no pudimos haber predicho cómo sería este organismo. Nosotros estamos de acuerdo con lo que argumenta Scriven, por lo que creemos que hay que tratar el problema del determinismo.

Un determinista podría argumentar que este fenómeno es reflejo de nuestra incapacidad cognitiva para evaluar todas las variables involucradas. En \cite{Graves1999}, las autoras argumentan a a favor de la tesis de que el indeterminismo en biología es epistémico y, en analogía con las leyes de la mecánica clásica, la biología evolutiva puede acercarse asintóticamente al determinismo. En contra de esto, nosotros argumentamos que la analogía con las leyes de la física es una mala analogía. Esto lo hacemos con base en decir que las leyes de la física clásica ostentan ser necesarias, en tanto que están estructuradas como consecuencias lógicas de axiomas. Esto significa que las leyes de la física no se acercan al determinismo por razones metafísicas, sino por razones lógicas.

Llegados a este punto hemos puesto en duda la distinción de Mayr entre causas próximas y causas últimas, asimismo hemos defendido al modelo manipulabilista de Woodward como una alternativa a otros modelos de explicación. Además, hemos dado un argumento a favor del indeterminismo causal, por lo que no hay problema en hablar de causalidad en biología evolutiva, en términos de causas próximas. Por último, lo que queremos hacer en esta tesis es aplicar el modelo de Woodward como modelo de explicación causal para apenas esbozar una solución al problema filosófico en torno al \emph{fitness}.

De manera clásica definimos al \emph{fitness} o adecuación, como una medida de descendencia. Los organismos que dejan más descendencia están mejor adecuados a su medio ambiente. Sin embargo, al definir así la adecuación de los organismos lo que obtenemos es una trivialidad: que los organismos que dejan más descendencia son los que dejan más descendencia. Dicho de otra manera, si el organismo más apto es el que deja más descendencia y aptitud está definido en términos de dejar más descendencia, entonces \emph{fitness} es tautológico \cite{Paul1992, sep-fitness}.

Bouchard y Rosenberg en \cite{Bouchard2004} sugiere que para librar a la selección natural de esta carga tautológica hay que definir al \emph{fitness} de otra manera: incorporando una noción de \emph{fitness} ecológico. Esta noción integra el tanto el medioambiente en el que se desarrollan los organismos y cómo estos resuelven problemas de ``diseño''. Esto en principio nos libera de la carga tautológica ya que aptitud se define en términos de resolución de problemas. Son los organismos que mejor resuelven estos problemas los que dejan más descendencia. Esto nos permite leer causalmente el concepto de \emph{fitness} a la vez que podemos medirlo en términos de sus consecuencias, a saber, en el número de descendientes.

Sin embargo, la noción de diseño requiere un desarrollo ulterior, además de que no es claro cómo individuar los problemas de diseño a los que se refieren Bouchard y Rosenberg en \citeyear{Bouchard2004}. Para solventar esto, nosotros apelamos a los modelos experimentales utilizados para reunir evidencia. Supongamos, por ejemplo, que insertamos a un depredador en el medio ambiente. Pasa el tiempo y observamos que los organismos que son depredados han desarrollado extremidades más largas que les permiten moverse más rápido. Si en un diseño experimental intervenimos insertando el mismo depredador y obtenemos los mismos resultados, entonces podemos individuar exactamente cuál problema de ``diseño'' están resolviendo los organismos, al mismo tiempo que obtenemos una explicación del fenómeno acorde con el modelo de Woodward: tienen extremidades más largas porque así pueden escapar con más facilidad. Si todo esto es correcto, obtenemos una noción más clara de \emph{fitness}.

Llegados a este punto, hemos defendido al modelo manipulabilista de Woodward como una alternativa a otros modelos de explicación. Hemos deslindado el determinismo de la causalidad y lo último que buscamos hacer es aplicar el modelo de manipulabilista de Woodward al problema de la adecuación. Creemos que al aplicar este modelo manipulabilista de Woodward podemos apoyar una noción de \emph{fitness} ecológico, que nos ayuda a individuar los problemas de diseño que se mencionan y que además nos permite leer causalmente el enunciado de que los más adecuados son los que sobreviven.



%El autor de este texto es Oscar Abraham Olivetti Alvarez
