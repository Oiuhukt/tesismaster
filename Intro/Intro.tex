%El autor de este texto es Oscar Abraham Olivetti Alvarez

\chapter*{Introducción}


\noindent Nosotros queremos en esta tesis apoyar una noción ecológica de la adecuación utilizando el modelo de Woodward. Para hacer esto, primero buscamos exponer en qué consiste el problema clásico de la adecuación, esto es, que la definición clásica es tautológica \cite{Paul1992, sep-fitness}. Repasamos tres nociones que se han propuesto para solucionar este problema y nos apoyamos en el modelo de Woodward para defender una noción de adecuación ecológica. La noción ecológica de adecuación soluciona el problema de la tautología al decir que los organismos son más adecuados cuando mejor resuelven problemas de diseño. Esto, sin embargo, no está libre de problemas. Por un lado no es claro cómo individuar los problemas de diseño. Además de que la misma noción de diseño es metafórica y poco clara.

Nosotros defenderemos el modelo de Woodward porque creemos que puede solventar estos problemas con la noción de adecuación ecológica. Para hacer esto, nosotros apelamos a un ambiente de laboratorio en el cual seamos capaces de intervenir en las variables que creemos son lo que ha llevado a cierto organismo a tener las características que tiene. Si En el ambiente controlado podemos aislar la variable que ha llevado a la modificación de los organismos, entonces decimos que es esa variable a la que responde el organismo y así está resolviendo este problema de diseño específico. Todo esto cuenta como una inbtervención en el senbtido definido por WOodwardd y como todo esto está relacionado de acuerdo a su modelo de explicación causal, entonces obtenemos además una explicación del fenómeno. Lo que en principio nos permite individuar los problemas de diseño es esta intervención en el laboratorio. De esta manera resolviendo así el problema de la noción ecológica. Decimos además que aceptar esta solución tiene la ventaja de permitirnos distinguir entre un proceso de evolución por selección natural y uno por deriva génica.

Para lograr nuestro objetivo primero queremos exponer el problema clásico de la adecuación y ver cómo aplicar el modelo de Woodward puede ser fructífero para usar una caracterización ecológica de la adecuación. De manera clásica definimos la adecuación como una medida de descendencia. Los organismos que dejan más descendencia están mejor adecuados a su medio ambiente. Sin embargo, al definir así la adecuación de los organismos lo que obtenemos es una trivialidad: que los organismos que dejan más descendencia son los que dejan más descendencia. Dicho de otra manera, si el organismo más apto es el que deja más descendencia y aptitud está definido en términos de dejar más descendencia, entonces adecuación es tautológico .

Bouchard y Rosenberg en \cite{Bouchard2004} sugieren que para librar a la adecuación de esta carga tautológica hay que definir la adecuación de otra manera: incorporando una noción de adecuación ecológica. Esta noción, de manera general, nos dice que el organismo más adecuado es aquél que mejor resuelve problemas de diseño. Esto en principio nos libera de la carga tautológica ya que aptitud se define en términos de resolución de problemas. Son los organismos que mejor resuelven estos problemas los que dejan más descendencia. Esto nos permite leer causalmente el concepto de adecuación, esto es, que el hecho de que resuelvan mejor los problemas de diseño impuestos por el medio ambiente, causa que tengan más descendencia.

Esta noción ecológica, como mencionamos anteriormente, requiere un desarrollo ulterior porque no es claro cómo individuar los problemas de diseño a los que se refieren Bouchard y Rosenberg en su artículo \citeyear{Bouchard2004}. Para solventar esto, nosotros apelamos a los modelos experimentales utilizados por los biólogos. Supongamos, por ejemplo, que insertamos a un depredador en el medio ambiente. Pasa el tiempo y observamos que los organismos que son depredados han desarrollado extremidades más largas que les permiten moverse más rápido. Si en un diseño experimental intervenimos insertando el mismo depredador y obtenemos los mismos resultados, entonces podemos individuar exactamente cuál problema de ``diseño'' están resolviendo los organismos, al mismo tiempo que obtenemos una explicación del fenómeno acorde con el modelo de Woodward: tienen extremidades más largas porque así pueden escapar con más facilidad. Si todo esto es correcto, obtenemos una noción más clara de qué problema de diseño resuelven los organismos.

Creemos que utilizando el modelo manipulabilista de Woodward podemos resolver el problema de individuar los problemas de ``diseño''  que tiene la noción de \emph{fitness} ecológico \cite{Bouchard2004, sep-fitness} y que nos permite distinguir entre un proceso de evolución por selección natural de uno debido a la deriva génica. Esta solución la esbozamos en términos de la replicabilidad de un evento natural en un ambiente controlado de laboratorio. Para lograr este objetivo, primero queremos exponer el problema y esbozar la solución pretendida, después exploraremos la distinción de Mayr entre causas próximas y causas últimas porque nuestro esbozo de solución depende de que haya causas próximas en las explicaciones evolutivas. Después de hacer esto pretendemos defender al modelo de Woodward frente a otros modelos de explicación. Luego, debido a que se asume que hay contingencia\footnote{Esta asunción no está libre de ser controvertida. En el artículo de Powell \citeyear{Powell2012}, el autor discute que hay algunos filósofos que toman el fenómeno de la evolución de caracteres análogos como evidencia a favor de que hay cierto grado de predicción en la teoría evolutiva.} involucrada en el proceso de selección \cite{Scriven1959, Mayr1998}, nosotros ofrecemos un argumento en contra del determinismo defendido Graves y compañía \cite{Graves1999}.

El primer paso para este proyecto es criticar la distinción trazada por Mayr entre causas próxzimas y causas últimas. Esta distinción trazada por Mayr está lejos de ser controvertida. En su artículo \cite{Laland2011} los autores argumentan que esta distinción no es tajante y que las causas próximas sí que juegan un papel en la evolución de los organismos. Los autores mencionan que, por ejemplo, el fenómeno de la plasticidad fenotípica puede llevar a que un caracter sea propenso a ser seleccionado, y la plasticidad fenotípica es un fenómeno que se relaciona con el medioambiente próximo. Por tanto, la selección natural y las causas próximas no necesariamente son excluyentes.

Este proyecto toma como motivación el hecho de que las causas próximas sí están relacionadas con explicaciones evolutivas. Es por ello que creemos que podemos dar un modelo de explicación causal que funcione para modelar explicaciones causales en biología evolutiva. Lo segundo que queremos hacer es proponer al modelo manipulabilista de Woodward  \cite{Woodward2000, Woodward2003} como un modelo de explicación causal apropiado para trabajar explicaciones causales, en términos de causas proximas. El segundo capítulo se encarga de esto. Es en el segundo capítulo que identificamos algunos problemas que tienen tanto el modelo Nomológico-deductivo, como el modelo de Relevancia estadística. Por último presentamos al modelo de Woodward como una alternativa a estos dos modelos de explicación. Hay que aclarar que el modelo de Woodward no está libre de problemas. Nosotros respondemos a algunos de los problemas del modelo de Woodward, tratando de hacerlo una alternativa viable para trabajar explicación causal en biología debido a que no necesita de leyes para su formulación y que resuelve el problema de contar como causales a correlaciones que tiene el modelo de relevancia estadística.

Sin duda en este trabajo falta la mención de dos alternativas que tienen como finalidad modelar la explicación. El primero es el modelo unificacionista que formularon Friedman \citeyear{Friedman1974} y Kitcher \cite{Kitcher2002}. La segunda gran falta es el modelo pragmático de van Fraassen. Dejamos de lado el modelo de Kitcher porque en su formulación las explicaciones causales no son necesarias. Dicho de otra manera, Kitcher cree que es epistémicamente problemático el concepto de causa y que este puede ser reducido a los patrones de explicación. En el modelo de Kitcher las relaciones causales son dependientes de las explicaciones, en sus propias palabras ``Lo que emerge en el límite de este proceso es nada menos que la estructura causal del mundo''\footnote{What emerges in the limit of this process is nothing less than the causal structure of the world.}. Dejamos de lado el modelo de van Fraassen porque es compatible con el modelo de explicación manipulabilista de Woodward, en donde la relación de relevancia de estar causalmente relacionados.

Continuando con la presentación, lo segundo que buscamos hacer en esta tesis es dar un argumento a favor del indeterminismo causal. A Mayr le preocupaba que las causas próximas estuvieran relacionadas con las explicaciones evolutivas, porque podía haber determinismo involucrado. Mayr \cite{Mayr1998} en un punto señala que el biólogo funcional se ocupa de un sólo individuo y sus relaciones medioambientales, además menciona que el biólogo funcional ``[...] si aísla suficientemente el fenómeno estudiado de las complejidades del organismo, puede alcanzar el ideal de un experimento en física o en química'' (p. 83). Pero si es verdad que la línea trazada entre biología funcional y biología evolutiva no es tajante como señalan Laland y compañía, entonces puede haber determinismo involucrado.

Pero no quisiéramos que el determinismo se cuele en los fenómenos biológicos. Este compromiso está motivado por la asimetría entre explicar y predecir que hay en biología evolutiva. Esta asimetría, argumenta Scriven \citeyear{Scriven1959}, nos dice que si bien podemos explicar un fenómeno en biología evolutiva, con la misma información no pudimos haber predicho cómo sería este organismo. Nosotros estamos de acuerdo con lo que argumenta Scriven, por lo que creemos que hay que tratar el problema del determinismo.

Un determinista podría argumentar que este fenómeno es reflejo de nuestra incapacidad cognitiva para evaluar todas las variables involucradas. En el artículo de Graves y colaboradores \citeyear{Graves1999}, las autoras argumentan a a favor de la tesis de que el indeterminismo en biología es epistémico y que a pesar de que el nivel fundamental es indeterminista, la física clásica trata  a su objeto de estudio como determinista, por lo que no es descabellado pensar que los procesos biológicos también puedan ser deterministas. Nosotros decimos que esta analogía con la mecánica clásica tiene problemas. Para esto, nosotros argumentamos que la analogía con las leyes de la física es una mala analogía. Esto lo hacemos con base en decir que las leyes de la física clásica ostentan ser necesarias, en tanto que están estructuradas como consecuencias lógicas de axiomas. Esto significa que las leyes de la física no se acercan al determinismo por razones metafísicas, sino por razones lógicas\footnote{Esto por supuesto puede significar que las áreas altamente matematizadas de la biología puedan ostentar el mismo tipo de determinismo que las leyes de la mecánica clásica}.

Llegados a este punto hemos puesto en duda la distinción de Mayr entre causas próximas y causas últimas, asimismo hemos defendido al modelo manipulabilista de Woodward como una alternativa a otros modelos de explicación. Además, hemos dado un argumento a favor del indeterminismo causal, resolviendo la asimetría de la que argumentan tanto Scriven como Mayr.

Un par de aclaraciones restan por hacer. No se hará una defensa de si la selección natural apela a explicaciones en términos de individuos o en términos de poblaciones. Lo único que nos interesa es el problema de individuar cuáles son los problemas de diseño que tiene la noción de adecuación ecológica. Sin duda cómo respondemos a esto tiene consecuencias en los debates antes mencionados, pero creemos que podemos aislar este problema lo suficiente para que sea independiente de estos problemas.  



%El autor de este texto es Oscar Abraham Olivetti Alvarez
