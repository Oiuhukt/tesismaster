%El autor de este texto es Oscar Abraham Olivetti Alvarez

\chapter*{Introducción}


\noindent El objeto de estudio de mi trabajo son las inferencias causales en biología evolutiva. Parte de la investigación científica se centra en buscar relaciones causales entre fenómenos. Cuando hablo de \textit{inferencias causales} me refiero a la justificación de dichas relaciones. En términos de un argumento, son aquellos argumentos en los que en la conclusión aparece una relación causal entre dos fenómenos.

La pregunta que quiero responder en mi trabajo es \textit{¿es el modelo de Woodward adecuado para tratar el fenómeno de la causalidad dentro de la biología evolutiva?} La pregunta es importante porque hay una tensión en torno a cómo interpretar la teoría de la selección natural. Específicamente el concepto de \textit{fitness}. Este problema está reflejado en \cite{Bouchard2004} y \cite{Walsh2002}. Los primero sostienen que la selección natural es a nivel de individuos y causal, mientras que los segundo se adhieren a la tesis de que es a nivel de poblaciones y puramente estadística.

Millstein en \cite{Millstein2006} trata de dar una salida al problema argumentando que la selección natural es un proceso a nivel de poblaciones, pero es un proceso causal. Sin embargo, Millstein no se avoca en su artículo a decirnos qué cuenta como proceso causal y qué no. Por lo que su solución es incompleta.

Diría que esto se debe, tal como se menciona en \cite{Uller}, a que los biólogos se han preocupado poco por el problema de la causalidad y este estudio ha sido relegado a los filósofos de la ciencia. Debido a que cierto compromiso con causalidad indica qué evidencia sería apta para una hipótesis, es importante aclarar cómo se utiliza información para justificar hipótesis causales.

En mi trabajo quiero defender la hipótesis de que el modelo de Woodward \cite{Woodward2000-WOOEAI} puede aclarar el uso de causalidad en biología.  Más aún, al comprometernos con una caracterización particular de la causalidad, pretendo mostrar que el debate entre Walsh y Rosenberg es espurio.

La causalidad es un concepto ampliamente usado para describir cómo se relacionan el medio ambiente y los organismos que habitan en él. La hipótesis es que el modelo de Woodward puede hacer más claro en términos metodológicos cómo se opera al experimentar cuando se tienen hipótesis en biología que tiene como conclusión una relación causal. Este modelo tiene la virtud de que esclarecemos los compromisos ontológicos y metodológicos que se tienen al trabajar con hipótesis de selección natural. Además de que parece encajar bien con cómo se experimenta en evolución del desarrollo \cite{Brakefield2014}

El primer capítulo es el menos original y trata sobre cómo los modelos de explicación no han hecho justicia a cómo se trabaja en ciencias como la biología. Además pretendo esquematizar el modelo de Woodward y motivar que es un buen modelo porque ejemplifica y esclarece cómo se trabaja en biología evolutiva.

El segundo capítulo trata el tema de las leyes, que es una de las partes fundamentales del argumento de Woodward. Woodward nos dice que las leyes no explican ya que la explicación está relacionada con entender los cambios. Sin embargo, no me parece suficiente tomar la tesis de que la explicación está relacionada con los cambios para desechar de una vez por todas que las leyes no explican. Para ello, quiero hacer un argumento histórico de por qué las leyes no son regularidades en el mundo, sino generalizaciones útiles.

Por último, en mi tercer capítulo quiero presentar tanto las posturas de Rosenberg \& Brouchard [BR] y la postura de Walsh, Lewens \& Ariew [LAW]. Después quiero presentar el consenso de Millstein y esclarecer cómo es que el modelo intervencionista de Woodward ayuda a iluminar el debate. Una virtud de este modelo  es que ilustra y empata con la metodología utilizada cuando se prueban hipótesis en biología evolutiva.

La pregunta es importante porque en lo referente a la biología evolutiva del desarrollo, se considera que el medio ambiente en el que habita el organismo tiene influencia en la selección de caracteres, por ejemplo, el tamaño de órganos y extremidades.

Esta influencia puede ser descrita en términos de una relación causal. Describir esta influencia de uno u otro modo, cambia la evidencia pertinente para justificar el fenómeno a investigar y este cambio está lejos de ser trivial. El modelo de Woodward me parece adecuado por el énfasis que hace en la intervención, lo que casa bien con el hecho de que agentes concretos puedan realizar experimentos de laboratorio que sirvan para justificar las hipótesis que pueden generar al observar el entorno
natural. El modelo de Woodward sirve además para no tomar a la causalidad como primitivo y evitar la discusión metafísica acerca de la naturaleza de la relación y los \emph{relata}, mientras que permite
centrarnos en la justificación y el uso de información causal

Pearson y su definición clkara de correlación sin hablar de causalidad.

Esta tesis trata de causalidad en Biología.



La causalidad en biología es una tema en el que hay un creciente interés por parte de los biólogos. Hay una supusoción bastante distrubída que afirm que el objetivo de la ciencia es buscar relaciones causales entre fenómenos. Los biólogos han hablado de causalidad en términos genéticos ya desde qu



%El autor de este texto es Oscar Abraham Olivetti Alvarez
